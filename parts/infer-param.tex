\section{Parametrization inference}\label{sec:infer-K}

Given the IG inferred from a PH as presented in the previous section, one can find the discrete parameters that model the behavior of the studied PH using the method presented in the following.
It relies on an exhaustive enumeration of all predecessors of each component in order to find attractor processes and returns a possibly incomplete parametrization, given the exhaustiveness of the cooperations.
The last step consists of the enumeration of all compatible complete parametrizations given this
set of inferred parameters, the PH dynamics and some biological constraints on parameters.

\subsection{Parameters inference}

This subsection presents some results related to the inference of independent discrete parameters from a given PH.
These results are equivalent to those presented in \cite{PMR10-TCSB}, with notation adapted to be shared with the previous section.
In addition, we introduce the well-formed PH for parameter inference property (\pref{pro:wf-ph-K}),
which implies that the inferred IG does not contain any unsigned interactions, and thus can be seen as the
regular IG $(\Gamma, E)$,
and that any processes in $\levels{b}{a}$ (resp. $\ulevels{b}{a}$) share the same behavior
regarding $a$.

\begin{property}[Well-formed PH for parameter inference]\label{pro:wf-ph-K}
A PH is well-formed for parameter inference if and only if
it is well-formed for IG inference, and
the IG $(\Gamma, E)$ inferred by \pref{pps:inference-IG}
verifies the following property:
\begin{align*}
  \begin{split}
  \forall a \in \Gamma, \forall b\in \GRNreg{a}&,
          \forall (i,j\in\levels{b}{a} \vee i,j\in\ulevels{b}{a}), \\
  & \quad \forall c \in \Gamma, ( (b\neq a\wedge c=a) \vee (c\in\PHpredec{a} \wedge b\in\PHdirectpredec{c})), \\
  & \qquad
                          \PHfrappe{b_i}{c_k}{c_l}\in\PHa \Leftrightarrow
                                  \PHfrappe{b_j}{c_k}{c_l}\in\PHa
  \end{split}
\end{align*}
\end{property}

Let $K_{a,\omega}$ be the parameter we want to infer for a given component $a \in \Gamma$
and $\omega \subset \GRNreg{a}$ a set of its regulators.
This inference, as for the IG inference, relies on the search of focal processes of the component for the given configuration of its regulators.

For each sort $b \in \GRNreg{a}$, we define a context $C^b_{a,\omega}$ in \pref{eq:param_context} that contains all processes representing the influence of the resources in the configuration modeled by $\omega$.
The context of a cooperative sort $\upsilon$ that regulates $a$ is given in
\pref{eq:param_context_coop} as the set of focal processes matching the current configuration.
$C_{a,\omega}$ refers to the union of all these contexts (\pref{eq:K-ctx}).
\begin{align}
  \label{eq:param_context}
  \forall b\in\Gamma,~
  C_{a,\omega}^b & \DEF \begin{cases}
    \levels{b}{a} & \text{if $b \in \omega$,}\\
    \ulevels{b}{a} & \text{if $b \notin \omega$,}\\
    L_b            & \text{otherwise;}\\
  \end{cases}
  \\
  \label{eq:param_context_coop}
  \forall \upsilon \in \PHpredec{a}\setminus\Gamma,~
  C_{a,\omega}^\upsilon & \DEF \{
  \upsilon(\sigma) \mid \sigma \in \textstyle\prod_{c\in\PHdirectpredec{\upsilon}}C_{a,\omega}^c \}
  \\
  C_{a,\omega} & \DEF \textstyle\bigcup_{b\in\PHpredec{a}} C^b_{a,\omega}
  \label{eq:K-ctx}
\end{align}

The parameter $K_{a,\omega}$ specifies to which values $a$ eventually evolves as long as the context
$C_{a,\omega}$ holds, which is precisely the definition of the $\focals$ function
(\pref{def:focals} in \pref{ssec:focal}),
where the focals reachability property can be derived from \pref{pro:wf-ph-K} and
\pref{eq:param_context_coop}.
Hence $K_{a,\omega} = \focals(a,C^a_{a,\omega},C_{a,\omega})$ if this latter is a non-empty interval
(\pref{pps:param_K}).

\begin{proposition}[Parameter inference]
\label{pps:param_K}
Let $(\PHs, \PHl, \PHh)$ be a Process Hitting well-formed for parameter inference, and $\IG = (\Gamma, E)$ the inferred IG.
Let $A$ (resp. $B$) $\subseteq \Gamma$ be the set of regulators that activate (resp. inhibit) a sort
$a$.
If $\focals(a,C^a_{a,\omega},C_{a,\omega})=\segm{a_i}{a_j}$ is a non-empty interval, then $K_{a,\omega} = \segm{i}{j}$.
\end{proposition}

\begin{example}
\label{ex:infer-param-runningPH-1}
Applied to the PH in \pref{fig:runningPH-1}, we obtain, in particular,
$K_{b,\emptyset} = \segm{0}{1}$ and
$K_{a,\{b,c\}} = \segm{2}{2}$,
while $K_{a,\{b\}}$ and $K_{a,\{c\}}$ can not be inferred.
\end{example}

\begin{example}
Regarding the PH in \pref{fig:runningPH-2}, we obtain
$K_{b,\emptyset} = \segm{0}{1}$,
$K_{a,\{b,c\}} = \segm{2}{2}$,
$K_{a,\{b\}} = \segm{1}{1}$,
and $K_{a,\{c\}} = \segm{1}{1}$.
For this PH, all parameters can be inferred, and the obtained parametrization is the one of \pref{fig:runningBRN}(right).
\end{example}

Given the \pref{pps:param_K}, we see that in some cases, the inference of the targeted parameter is impossible.
This can be due to a lack of cooperation between regulators: when two regulators independently hit a component, their actions can have opposite effects, leading to either an indeterministic evolution or to oscillations.
Such an indeterminism is not possible in a BRN as in a given configuration of regulators, a component can only have an interval attractor, and eventually reaches a steady-state.
In order to avoid such inconclusive cases, one has to ensure that no such behavior is allowed by
either removing undesired actions or using cooperative sorts to prevent opposite influences between
regulators.

\subsection{Admissible parametrizations}\label{ssec:admissible-K}

When building a BRN, one has to find the parametrization that best describes the desired behavior of the studied system.
Complexity is inherent to this process as the number of possible parametrizations for a given IG is exponential w.r.t. the number of components.
However, the method of parameters inference presented in this section gives some information about necessary parameters given a certain dynamics described by a PH.
This information thus drops the number of possible parametrizations, allowing to find the desired behavior more easily.

We first delimit the validity of a parameter (\pref{pro:K-valid}) in order to ensure that any
transition in the resulting BRN is allowed by the studied PH.
This is verified by the existence of a hit making the concerned component bounce into the direction
of the value of the parameter in the matching context.
Thus, assuming \pref{pro:wf-ph-K} holds, any transition in the inferred BRN corresponds to at least
one transition in the PH, proving the correctness of our inference.
We remark that any parameter inferred by \pref{pps:param_K} satisfies this property.

\begin{property}[Parameter validity]\label{pro:K-valid}
A parameter $K_{a,\omega}$ is valid w.r.t. the PH if and only if the following equation is verified:
\begin{align*}
  \forall a_i\in C^a_{a,\omega}, a_i \notin K_{a,\omega} \Longrightarrow
    (& \exists \PHfrappe{c_k}{a_i}{a_j}\in\PHa, c_k \in C^c_{a,\omega} \\
     & \wedge a_i < K_{a,\omega} \Rightarrow j > i \wedge  a_i > K_{a,\omega} \Rightarrow j <i )
\end{align*}
\end{property}

Then, we use some additional biological constraints on Thomas' parameters given in
\cite{BernotSemBRN}, that we sum up in the following three properties:

\begin{property}[Extreme values assumption]
Let $\IG = (\Gamma, E)$ be an IG. A parametrization $K$ on $\IG$ satisfies the \emph{extreme values assumption} if and only if:
\label{pro:param_enum_extreme}
\begin{align*}
  \forall b \in \Gamma, \GRNreg{b} \neq \emptyset \Longrightarrow \exists \omega \subset \GRNreg{b}, 0 \in K_{b,\omega} \wedge \exists \omega' \subset \GRNreg{b}, l_b \in K_{b,\omega'}
\end{align*}
\end{property}

\begin{property}[Activity assumption]
\label{pro:param_enum_activity}
Let $\IG = (\Gamma, E)$ be an IG. A parametrization $K$ on $\IG$ satisfies the \emph{activity assumption} if and only if:
\begin{align*}
  \forall b \in \Gamma, \forall a \in \GRNreg{b}, \exists \omega \subset \GRNreg{b}, K_{b,\omega} \neq K_{b,\omega \cup \{ a \}}
\end{align*}
\end{property}

\begin{property}[Monotonicity assumption]
\label{pro:param_enum_monotonicity}
Let $\IG = (\Gamma, E)$ be an IG. A parametrization $K$ on $\IG$ satisfies the \emph{monotonicity assumption} if and only if:
\begin{align*}
  \forall b \in \Gamma,
  \forall A^+ \subset \{ a \in \Gamma \mid \GRNedge{a}{+}{t}{b} \in E_+ \}&,
  \forall A^- \subset \{ a \in \Gamma \mid \GRNedge{a}{-}{t}{b} \in E_- \},\\
  K_{b,\omega \cup A^-} & \leqsegm K_{b,\omega \cup A^+}
\end{align*}
\end{property}
