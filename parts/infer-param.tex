\section{Parametrization inference}\label{sec:infer-K}

Given the IG inferred from a PH as presented in the previous section, one can find the discrete parameters that model the behavior of the studied PH using the method presented in the following.
It relies on an exhaustive enumeration of all predecessors of each component in order to find attractor processes and returns a possibly incomplete parametrization, given the exhaustiveness of the cooperations.
The last step consists of the enumeration of all compatible complete parametrizations given this
set of inferred parameters, the PH dynamics and some biological constraints on parameters.

\subsection{Parameters inference}

This subsection presents some results related to the inference of independent discrete parameters from a given PH.
These results are equivalent to those presented in \cite{PMR10-TCSB}, with notation adapted to be shared with the previous section.
In addition, we introduce the well-formed PH for parameter inference property (\pref{pro:wf-ph-K}),
which implies that the inferred IG does not contain any unsigned interactions, and thus can be seen as the
regular IG $(\Gamma, E)$,
and that any processes in $\levels{b}{a}$ (resp. $\ulevels{b}{a}$) share the same behavior
regarding $a$.

\begin{property}[Well-formed PH for parameter inference]\label{pro:wf-ph-K}
A PH is well-formed for parameter inference if and only if
it is well-formed for IG inference, and
the IG $(\Gamma, E)$ inferred by \pref{pps:inference-IG}
verifies the following property:
\begin{align*}
  \begin{split}
  \forall a \in \Gamma, \forall b\in \GRNreg{a}&,
          \forall (i,j\in\levels{b}{a} \vee i,j\in\ulevels{b}{a}), \\
  & \quad \forall c \in \Gamma, ( (b\neq a\wedge c=a) \vee (c\in\PHpredec{a} \wedge b\in\PHdirectpredec{c})), \\
  & \qquad
                          \PHfrappe{b_i}{c_k}{c_l}\in\PHa \Leftrightarrow
                                  \PHfrappe{b_j}{c_k}{c_l}\in\PHa
  \end{split}
\end{align*}
\end{property}

Let $K_{a,\omega}$ be the parameter we want to infer for a given component $a \in \Gamma$
and $\omega \subset \GRNreg{a}$ a set of its regulators.
This inference, as for the IG inference, relies on the search of focal processes of the component for the given configuration of its regulators.

For each sort $b \in \GRNreg{a}$, we define a context $C^b_{a,\omega}$ in \pref{eq:param_context} that contains all processes representing the influence of the resources in the configuration modeled by $\omega$.
The context of a cooperative sort $\upsilon$ that regulates $a$ is given in
\pref{eq:param_context_coop} as the set of focal processes matching the current configuration.
$C_{a,\omega}$ refers to the union of all these contexts (\pref{eq:K-ctx}).
\begin{align}
  \label{eq:param_context}
  \forall b\in\Gamma,~
  C_{a,\omega}^b & \DEF \begin{cases}
    \levels{b}{a} & \text{if $b \in \omega$,}\\
    \ulevels{b}{a} & \text{if $b \notin \omega$,}\\
    L_b            & \text{otherwise;}\\
  \end{cases}
  \\
  \label{eq:param_context_coop}
  \forall \upsilon \in \PHpredec{a}\setminus\Gamma,~
  C_{a,\omega}^\upsilon & \DEF \{
  \upsilon(\sigma) \mid \sigma \in \textstyle\prod_{c\in\PHdirectpredec{\upsilon}}C_{a,\omega}^c \}
  \\
  C_{a,\omega} & \DEF \textstyle\bigcup_{b\in\PHpredec{a}} C^b_{a,\omega}
  \label{eq:K-ctx}
\end{align}

The parameter $K_{a,\omega}$ specifies to which values $a$ eventually evolves as long as the context
$C_{a,\omega}$ holds, which is precisely the definition of the $\focals$ function
(\pref{def:focals} in \pref{ssec:focal}),
where the focals reachability property can be derived from \pref{pro:wf-ph-K} and
\pref{eq:param_context_coop}.
Hence $K_{a,\omega} = \focals(a,C^a_{a,\omega},C_{a,\omega})$ if this latter is a non-empty interval
(\pref{pps:param_K}).

\begin{proposition}[Parameter inference]
\label{pps:param_K}
Let $(\PHs, \PHl, \PHh)$ be a Process Hitting well-formed for parameter inference, and $\IG = (\Gamma, E)$ the inferred IG.
Let $A$ (resp. $B$) $\subseteq \Gamma$ be the set of regulators that activate (resp. inhibit) a sort
$a$.
If $\focals(a,C^a_{a,\omega},C_{a,\omega})=\segm{a_i}{a_j}$ is a non-empty interval, then $K_{a,\omega} = \segm{i}{j}$.
\end{proposition}

\begin{example}
\label{ex:infer-param-runningPH-1}
Applied to the PH in \pref{fig:runningPH-1}, we obtain, in particular,
$K_{b,\emptyset} = \segm{0}{1}$ and
$K_{a,\{b,c\}} = \segm{2}{2}$,
while $K_{a,\{b\}}$ and $K_{a,\{c\}}$ can not be inferred.
\end{example}

\begin{example}
Regarding the PH in \pref{fig:runningPH-2}, we obtain
$K_{b,\emptyset} = \segm{0}{1}$,
$K_{a,\{b,c\}} = \segm{2}{2}$,
$K_{a,\{b\}} = \segm{1}{1}$,
and $K_{a,\{c\}} = \segm{1}{1}$.
For this PH, all parameters can be inferred, and the obtained parametrization is the one of \pref{fig:runningBRN}(right).
\end{example}

Given the \pref{pps:param_K}, we see that in some cases, the inference of the targeted parameter is impossible.
This can be due to a lack of cooperation between regulators: when two regulators independently hit a component, their actions can have opposite effects, leading to either an indeterministic evolution or to oscillations.
Such an indeterminism is not possible in a BRN as in a given configuration of regulators, a component can only have an interval attractor, and eventually reaches a steady-state.
In order to avoid such inconclusive cases, one has to ensure that no such behavior is allowed by
either removing undesired actions or using cooperative sorts to prevent opposite influences between
regulators.

\subsection{Admissible parametrizations}\label{ssec:admissible-K}

When building a BRN, one has to find the parametrization that best describes the desired behavior of the studied system.
Complexity is inherent to this process as the number of possible parametrizations for a given IG is exponential w.r.t. the number of components.
However, the method of parameters inference presented in this section gives some information about necessary parameters given a certain dynamics described by a PH.
This information thus drops the number of possible parametrizations, allowing to find the desired behavior more easily.

We first delimit the validity of a parameter (\pref{pro:K-valid}) in order to ensure that any
transition in the resulting BRN is allowed by the studied PH.
This is verified by the existence of a hit making the concerned component bounce into the direction
of the value of the parameter in the matching context.
Thus, assuming \pref{pro:wf-ph-K} holds, any transition in the inferred BRN corresponds to at least
one transition in the PH, proving the correctness of our inference.
We remark that any parameter inferred by \pref{pps:param_K} satisfies this property.

\begin{property}[Parameter validity]\label{pro:K-valid}
A parameter $K_{a,\omega}$ is valid w.r.t. the PH if and only if the following equation is verified:
\begin{align*}
  \forall a_i\in C^a_{a,\omega}, a_i \notin K_{a,\omega} \Longrightarrow
    (& \exists \PHfrappe{c_k}{a_i}{a_j}\in\PHa, c_k \in C^c_{a,\omega} \\
     & \wedge a_i < K_{a,\omega} \Rightarrow j > i \wedge  a_i > K_{a,\omega} \Rightarrow j <i )
\end{align*}
\end{property}

Then, we use some additional biological constraints on Thomas' parameters given in
\cite{BernotSemBRN}, that we sum up in the following three properties:

\begin{property}[Extreme values assumption]
Let $\IG = (\Gamma, E)$ be an IG. A parametrization $K$ on $\IG$ satisfies the \emph{extreme values assumption} if and only if:
\label{pro:param_enum_extreme}
\begin{align*}
  \forall b \in \Gamma, \GRNreg{b} \neq \emptyset \Longrightarrow \exists \omega \subset \GRNreg{b}, 0 \in K_{b,\omega} \wedge \exists \omega' \subset \GRNreg{b}, l_b \in K_{b,\omega'}
\end{align*}
\end{property}

\begin{property}[Activity assumption]
\label{pro:param_enum_activity}
Let $\IG = (\Gamma, E)$ be an IG. A parametrization $K$ on $\IG$ satisfies the \emph{activity assumption} if and only if:
\begin{align*}
  \forall b \in \Gamma, \forall a \in \GRNreg{b}, \exists \omega \subset \GRNreg{b}, K_{b,\omega} \neq K_{b,\omega \cup \{ a \}}
\end{align*}
\end{property}

\begin{property}[Monotonicity assumption]
\label{pro:param_enum_monotonicity}
Let $\IG = (\Gamma, E)$ be an IG. A parametrization $K$ on $\IG$ satisfies the \emph{monotonicity assumption} if and only if:
\begin{align*}
  \forall b \in \Gamma,
  \forall A^+ \subset \{ a \in \Gamma \mid \GRNedge{a}{+}{t}{b} \in E_+ \}&,
  \forall A^- \subset \{ a \in \Gamma \mid \GRNedge{a}{-}{t}{b} \in E_- \},\\
  K_{b,\omega \cup A^-} & \leqsegm K_{b,\omega \cup A^+}
\end{align*}
\end{property}



\subsection{Answer Set Programming implementation concepts}

\rewrite{Should this subsection be turned into a new section?}

\newcommand{\ti}[1]{\texttt{\textit{#1}}}
\newcommand{\aspil}[1]{\texttt{#1}}
\newcommand{\asp}[1]{\begin{itemize} \item[] \aspil{#1} \end{itemize}}

\newcommand{\atom}{\mathbf}
%\newcommand{\predicate}{\mathbf}
\newcommand{\predicate}[1]{#1}
\newcommand{\la}{\leftarrow}
\newcommand{\var}[1]{#1}
\newcommand{\nota}{\neg}

\newcommand{\paramlabel}{\predicate{param\_label}}
\newcommand{\paramres}{\predicate{param\_resource}}
\newcommand{\component}{\predicate{component}}
\newcommand{\componentlevels}{\predicate{component\_levels}}
\newcommand{\param}{\predicate{param}}
\newcommand{\inferedparam}{\predicate{infered\_param}}
\newcommand{\lessactive}{\predicate{less\_active}}
\newcommand{\paraminf}{\predicate{param\_inf}}



Answer Set Programming (ASP) is a logic programming paradigm \cite{Baral03},
which has been chosen to address the enumeration of all admissible parametrizations.
The motivations are following:
\begin{itemize}
  \item ASP efficiently tackles the inherent complexity of the models used, thus allowing a fast execution of the formal tools defined in this paper,
  \item ASP is convenient to enumerate a large set of possible answers,
  \item and it allows to easily constrain the answers according to some properties.
\end{itemize}
We now synthesize some key points to better make the reader understand our ASP implementation with the enumeration example.

\subsubsection{Simple rules}\label{sssec:simple_rules}
ASP is based on a set of rules of of the form:
\begin{align*}
  \underbrace{{\ }\atom{H}_{\ }}_{head} \la \underbrace{\atom{A}_1, \atom{A}_2, \dots, \atom{A}_n, \nota \atom{B}_1, \nota \atom{B}_2, \dots, \nota \atom{B}_m}_{body}.
\end{align*}
where the $body$ is a series of atoms ($\atom{A}_i$) and negations of atoms ($\nota \atom{B}_i$).
In the case of \emph{simple rules} (as opposed to the \emph{cardinality rules} of \pref{sssec:cardinality_rules}), the $head$ is also an atom ($\atom{H}$).
Such a rule states that if all atoms $\atom{A}_1, \atom{A}_2, \dots, \atom{A}_n$ are true
and all atoms $\atom{B}_1, \atom{B}_2, \dots, \atom{B}_m$ are not true (negation by failure), then $\atom{H}$ has to be true.
Solving an ASP program means finding an \emph{answer set}, that is a minimal set of true atoms that respect all the rules.
Several answer sets can be solution to the same ASP program and the solver can be directed to enumerate them all.

An atom is composed of a predicate and a series of arguments (possibly empty).
For example, the following atom:
\begin{align*}
  \predicate{p}(x_1, x_2, \dots, x_r)
\end{align*}
is composed of the predicate $\predicate{p}$ and $r$ arguments: $x_1, x_2, \dots, x_r$.
Each argument can be either an value or a variable,
a value being a representation of a piece of data (component name, expression level, …)
and a variable can is used to represent any existing value which respects the rules.
In this paper, a variable is always denoted by a single capital letter (e.g.~$\var{A}$, $\var{P}$, …)
while values are either numerical or consist of a single lowercase letter (e.g.~$a$, $b$, $c$, $1$, $2$, …).

A rule with no $body$ part is called a fact, and its $head$ atom has to belong to all answer sets.
For instance, the information describing the studied model (the original PH model and the inferred IG and parameters) are expressed in ASP using facts.
In particular, the predicate $\component$ allows to define all components belonging to the inferred IG.
Thus, an atom $\component(a, m)$ states that $a$ is a component of the IG and that $l_a = m$.

\begin{example}
Consider \pref{ex:infer-param-runningPH-1}: the inferred IG contains 3 components, and to state the existence of each of them, the following facts are used:
\begin{align*}
  &\component(a, 2). \\
  &\component(b, 1). \\
  &\component(c, 1).
\end{align*}
These three atoms are built with the predicate $\component$.
Furthermore, $a$, $b$, $c$, $1$ and $2$ are values.
\end{example}

To describe the sets of all expression levels of each component (\ie the set $\segm{0}{l_a}$ for each $a \in \Gamma$),
one can use atoms of the form $\componentlevels(a, k)$ to state that $k \in \segm{0}{l_a}$.
Variables here come in handy to enumerate each possible value $k$ for each component $a$:
during the solving, any rule containing variables is duplicated in order to replace each variable by all the possible values it could represent.
The following rule, for example, contains three variables ($\var{A}$, $\var{K}$ and $\var{M}$) and enumerates the set of possible expression levels of each component in the system:
\begin{align*}
  \componentlevels(A, K) \la \component(\var{A}, \var{M}), 0 \leq K \leq M.
\end{align*}
where the notation “$\leq$” stands for a shortcut in ASP which has the same meaning as the mathematical operator.

\begin{example}
Regarding \pref{ex:infer-param-runningPH-1}, the previous rule will make the following set of atoms belong to every answer set:
\begin{align*}
  &\{&\componentlevels(b, 0)
  &&\componentlevels(b, 1) \\
  &&\componentlevels(c, 0)
  &&\componentlevels(c, 1) \\
  &&\componentlevels(a, 0)
  &&\componentlevels(a, 1) \\
  &&\componentlevels(a, 2) &&&&\}
\end{align*}
\end{example}

\subsubsection{Cardinality rules}
\label{sssec:cardinality_rules}
As an extension of simple rules, \emph{cardinality rules} turn out to be convenient to enumerate a set of answer sets.
The head of a cardinality rule specifies a set of atoms $H$ and two integers $min$ and $max$, and is denoted:
\begin{align*}
  min\ \{\ H\ \}\ max \la \atom{A}_1, \atom{A}_2, \dots, \atom{A}_n, \nota \atom{B}_1, \nota \atom{B}_2, \dots, \nota \atom{B}_m.
\end{align*}
Given such a rule, as many answer sets as possible are created, so that each answer $S$ set verifies:
\begin{align*}
  min \leq |S \cap H| \leq max
\end{align*}
and every atom $\atom{H}_i \in S \cap H$ respects the simple rule:
\begin{align*}
  \atom{H}_i \la \atom{A}_1, \atom{A}_2, \dots, \atom{A}_n, \nota \atom{B}_1, \nota \atom{B}_2, \dots, \nota \atom{B}_m.
\end{align*}
In other words, all answer sets contain a subset of $H$ whose cardinality goes from $min$ to $max$.
The set of atoms $H = \{ \atom{H}_1, \atom{H}_2, \dots, \atom{H}_p \}$ is often defined as: $H = \atom{P} : \atom{Q}$,
which is a shorthand for “the set of atoms of the form $\atom{P}$ for which $\atom{Q}$ is true”.
\rewrite{Reformulate?}

Cardinality rules turn out to be convenient to enumerate all possible parametrizations by creating multiple answer sets.
For functional purposes, a unique label is assigned to every possible set of resources of a given component.
Thus, we denote $\omega_p$ the set of resources of a given component $a$ labeled by $p$,
and naturally, $K_{a,\omega_p}$ is the related parameter.
%and in the following we note $K_{a,\omega_p}$ the parameter of component $a$ whose set of resources $\omega$ is assigned to the label $p$.
We note that labeling the sets of resources of a component is obviously equivalent to labeling its parameters.
Then, suppose that:
\begin{itemize}
  \item $\paramlabel(a, p)$ states that $p$ is a valid label for a set of resources of component $a$ (and therefore $K_{a,\omega_p}$ is a valid parameter);
  \item $\param(a, p, i)$ states that: $i \in K_{a, \omega_p}$;
  \item $\inferedparam(a, p)$ states that the parameter inference of $K_{a, \omega_p}$ was conclusive (\pref{pps:param_K}).
\end{itemize}
It is thereby possible to enumerate the possible values of all parameters for which \pref{pps:param_K} was not conclusive, with the following cardinality rule:
\begin{align*}
  & 1\ \{\ \param(\var{A}, \var{P}, \var{I}) : \componentlevels(\var{A}, \var{I})\ \}\ \infty\ \la \\
  & \qquad\qquad\qquad \paramlabel(\var{A}, \var{P}), \nota \inferedparam(\var{A}, \var{P}).
\end{align*}
Indeed, this rule applies to any possible parameter $\var{P}$ of any component $\var{A}$ ($\paramlabel$) whose value is still unknown ($\nota \inferedparam$),
and states that any expression level $\var{I}$ of this component ($\componentlevels$) can belong to the value of the parameter ($\param$).
Furthermore, the lower bound is $1$, which forces each enumerated parameter to contain at least one value,
but no upper bound is specified ($\infty$) for the size of each parameter (which is bounded by the number of possible expression levels of each gene).
%each parameter contains at least $1$ value, as stated by the lower bound, and has no upper bound ($\infty$) but the number of expression levels of its component.
In other worlds, this cardinality rule creates as many answer sets as there are \emph{candidate} parametrizations
so that if $K_{a, \omega_p}$ could not be inferred by \pref{pps:param_K}, then
$K_{a, \omega_p} \subset \segm{0}{l_a} \wedge K_{a, \omega_p} \neq \emptyset$
(thus completely disregarding the notion of admissible parametrizations given in \pref{ssec:admissible-K} or the fact that parameters have to be intervals).


%so that
%inferred parameters with \pref{pps:param_K} keep their value and
%parameters that could not be inferred take any possible value amongst the expression levels of the related component: $K_{a, \omega_p} \subset \segm{0}{l_a} \wedge K_{a, \omega_p} \neq \emptyset$.

\subsubsection{Constraints}\label{sssec:constraints}
%Finally, constraints can be used to filter answer sets that model parametrizations which are impossible or do not respect the assumptions of \pref{ssec:admissible-K}.
Finally, a constraint is a rule with no $head$ part:
\begin{align*}
  \la \atom{A}_1, \atom{A}_2, \dots, \atom{A}_n, \nota \atom{B}_1, \nota \atom{B}_2, \dots, \nota \atom{B}_m.
\end{align*}
A constraint is satisfied only if its $body$ is not satisfied,
which thus allows to invalidate answer sets containing some unwanted combinations of atoms.
In the scope of parameters enumeration, for example, constraints are especially useful to filter parametrizations
that contain non-interval parameters or that do not respect the assumptions of \pref{ssec:admissible-K}.
Indeed, suppose that:
\begin{itemize}
  \item $\lessactive(a, p, q)$ states that $\omega_p$ is a set of resources of $a$ with (loosely) less activators and more inhibitors than $\omega_q$;
  \item $\paraminf(a, p, q)$ states that: $K_{a,\omega_p} \leqsegm K_{a,\omega_q}$.
\end{itemize}
Then, the monotonicity assumption (\pref{pro:param_enum_monotonicity}) is formulated as the following constraint:
\begin{align*}
  \la \lessactive(\var{A}, \var{P}, \var{Q}), \nota \paraminf(\var{A}, \var{P}, \var{Q}).
\end{align*}
Indeed, this constraint removes all parametrization results where parameters $K_{\var{A},\omega_\var{P}}$ and $K_{\var{Z},\omega_\var{Q}}$ exist
such that $\var{A}$ is less activated by the set of resources $\omega_\var{P}$ than $\omega_\var{Q}$,
but $K_{\var{A},\omega_\var{Q}} \ltsegm K_{\var{A},\omega_\var{P}}$,
thus violating the monotonicity assumption.
Of course, other assumptions can be formulated in the same way.

\begin{example}
In the scope of \pref{ex:infer-param-runningPH-1}, $K_{a,\{b\}}$ and $K_{a,\{c\}}$ could not be inferred by \pref{pps:param_K}.
The enumeration thus produces 36 parametrizations, in which these parameters can take all possible values
(as, in this case, the assumptions of \pref{ssec:admissible-K} bring no new constraint):
\begin{align*}
  (K_{a,\{b\}} ; K_{a,\{c\}}) \in \{ \segm{0}{0} , \segm{1}{1} , \segm{2}{2} , \segm{0}{1} , \segm{1}{2} , \segm{0}{2} \}^2
\end{align*}
We note that $(\segm{1}{1} ; \segm{1}{1})$ belongs to the enumerated set of admissible parametrizations.
Therefore, this enumeration allows, from the model in \pref{fig:runningPH-1}, to find the behavior of the model completed with a cooperative sort described in \pref{fig:runningPH-2}.
\end{example}



\begin{comment}
---------------

In particular, the existence of a parameter (whether its value has been inferred or not) is stated by the atom $\paramlabel$.
For functional purposes, we assign a unique label to every possible set of resources $\omega$ of a given component,
and in the following we note $K^p_{a,\omega}$ the parameter of component $a$ whose set of resources $\omega$ is assigned to the label $p$.
In other words, $\paramlabel(a, p)$ states that a parameter $K^p_{a,\omega}$ exists.
Consider \pref{ex:infer-param-runningPH-1}: the inferred IG contains 7 parameters, and to state the existence of each of them, the following facts are used:
\begin{align*}
  &\paramlabel(a,1).
  &\paramlabel(b,1).\\
  &\paramlabel(a,2).
  &\paramlabel(b,2).\\
  &\paramlabel(a,3).\\
  &\paramlabel(a,4).
  &\paramlabel(c,1).
\end{align*}

A set (such as sets of resources $\omega$) can be defined in ASP with a series of atoms. \rewrite{Find a more precise sentence.}
For example, for a given parameter $K^p_{a,\omega}$, expressing $x \in \omega$ is done by the atom: $\paramres(a, p, x)$.
Then, the sets of resources of \pref{ex:infer-param-runningPH-1} is defined by:
$$\begin{array}{>{\hspace{.8em}}l|>{\phantom{\raisebox{.6em}{xx}}}l}
  \hspace{-.8em}\text{Parameter} & \text{Corresponding ASP fact} \\
\hline
  K^0_{a,\emptyset} & \\
  K^1_{a,\{b\}} & \paramres(a,1,b). \\
  K^2_{a,\{c\}} & \paramres(a,2,c). \\
  K^3_{a,\{b,c\}} & \paramres(a,3,b). \quad \paramres(a,3,c). \phantom{\raisebox{-.8em}{xx}} \\ \hline
  K^0_{b,\emptyset} & \\
  K^1_{b,\{a\}} & \paramres(b,1,a). \phantom{\raisebox{-1em}{xx}} \\ \hline
  K^0_{c,\emptyset} &
\end{array}$$
The absence of an atom $\paramres(a,1,c)$, for example, states that $c$ is not a resource for the parameter $K^1_{a,\{b\}}$ (indeed, $c \notin \{b\}$).



\todo{Continue refactoring.}
\rewrite{Explain variables}

Rules allow more detailed declarations than facts as they have a body (right-hand part below) containing constraints and allowing to use variables, while facts only have a head (left-hand part).
For instance, in order to define the set of expression levels of a component, we declare:
$$\atom{component\_levels}(\var{X}, 0..\var{M}) \la \atom{component}(\var{X}, \var{M}).$$
where the atom $\atom{component}(X, M)$ stands for the existence of a component $X$ with a maximum level $M$.
Considering this declaration, any possible answer for the atom $\atom{component\_levels}$ will be found by binding all possible values of its two terms with all existing $\atom{component}$ facts:
the existence of an answer $\atom{component\_levels}(a, k)$ will depend on the existence of a term $a$, which is bound with $\var{X}$, and an integer $k$, constrained by: $0 \leq k \leq \var{M}$.

Cardinalities are convenient to enumerate all possible parametrizations by creating multiple answer sets.
A cardinality (denoted hereafter with curly brackets) gives any number of possible answers for some atoms between a lower and upper bounds.
For example,
\begin{align*}
  & 1\ \{ \atom{param}(\var{X}, \var{P}, \var{I}) : \atom{component\_levels}(\var{X}, \var{I}) \} \la \\
  & \qquad\qquad \paramlabel(\var{X}, \var{P}), \nota \atom{infered\_param}(\var{X}, \var{P}).
\end{align*}
where $\atom{param}(\var{X}, \var{P}, \var{I})$ stands for: $\var{I} \in K^\var{P}_{\var{X},\omega}$,
means that any parameter of component $\var{X}$ and label $\var{P}$ must contain at least one level value ($\var{I}$) in the possible expression levels of $\var{X}$.
Indeed, the lower bound is $1$, forcing at least one element in the parameter, but no upper bound is specified, allowing up to any number of answers.
The body (right-hand side) of the rule also checks for the existence of a parameter of $\var{X}$ with label $\var{P}$,
and constrains that the parametrization inference was not conclusive for the considered parameter (“$\nota$” stands for negation by failure: $\nota \atom{L}$ becomes true if $\atom{L}$ is not true).
Such a constraint gives multiple results as any set of $\atom{param}$ atoms satisfying the cardinality will lead to a new global set of answers.
In this way, we enumerate all possible parametrizations which respects the results of parameters inference,
but completely disregarding the notion of admissible parametrizations given in \pref{ssec:admissible-K}.

We rely on integrity constraints to filter only admissible parametrizations.
An integrity constraint is a rule with no head, that makes an answer set unsatisfiable if its body turns out to be true.
Hence, if we suppose that:
\begin{itemize}
  \item the $\atom{less\_active}(a, p, q)$ atom means that $K^p_{a,\omega}$ stands for a configuration with less activating regulators than $K^q_{a,\omega'}$, % (\ie $A \subset A'$),
  \item the $\atom{param\_inf}(a, p, q)$ atom means: $K^p_{a,\omega} \leqsegm K^q_{a,\omega'}$,
\end{itemize}
then the monotonicity assumption (\pref{pro:param_enum_monotonicity}) is formulated as the following integrity constraint:
$$\la \atom{less\_active}(\var{X}, \var{P}, \var{Q}), \nota \atom{param\_inf}(\var{X}, \var{P}, \var{Q}).$$
which removes all parametrization results where parameters $K^\var{P}_{\var{X},\omega}$ and $K^\var{Q}_{\var{X},\omega'}$ exist such that $\var{X}$ is less activated by $\omega$ than $\omega'$ %$A \subset A'$
but $K^\var{Q}_{\var{X},\omega'} \ltsegm K^\var{P}_{\var{X},\omega}$,
thus violating the monotonicity assumption.
Of course, other assumptions can be formulated in the same way.
\end{comment}



This subsection succinctly described how ASP programs come in handy to represent a model and solve all steps of Thomas' modeling inference.
It finds a particularly interesting application in the enumeration of parameters: all possible parametrizations are generated in separate answer sets, and integrity constraints are formulated to remove those that do not fit the assumptions of admissible parametrizations,
thus reducing the number of interesting parametrizations to be considered in the end.
