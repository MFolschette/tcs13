\section{Answer Set Programming implementation concepts}\label{sec:impl}

\newcommand{\ti}[1]{\texttt{\textit{#1}}}
\newcommand{\aspil}[1]{\texttt{#1}}
\newcommand{\asp}[1]{\begin{itemize} \item[] \aspil{#1} \end{itemize}}

%\newcommand{\atom}{\mathbf}
\newcommand{\atom}[1]{#1}
%\newcommand{\predicate}{\mathbf}
\newcommand{\predicate}[1]{#1}
\newcommand{\la}{\leftarrow}
\newcommand{\var}[1]{#1}
\newcommand{\nota}{\neg}

\newcommand{\paramlabel}{\predicate{param\_label}}
\newcommand{\paramres}{\predicate{param\_resource}}
\newcommand{\component}{\predicate{component}}
\newcommand{\componentlevels}{\predicate{component\_levels}}
\newcommand{\param}{\predicate{param}}
\newcommand{\inferedparam}{\predicate{infered\_param}}
\newcommand{\lessactive}{\predicate{less\_active}}
\newcommand{\paraminf}{\predicate{param\_inf}}



Answer Set Programming (ASP) is a logic programming paradigm \cite{Baral03, Baral10},
which has been chosen to address the enumeration of all admissible parametrizations.
The motivations are following:
\begin{itemize}
  \item ASP efficiently tackles the inherent complexity of the used models, thus enabling us fast execution of the formal tools defined in this paper,
  \item ASP can efficiently enumerate a large set of possible answers,
  \item and it is easy to constrain the answers according to some properties.
\end{itemize}
We now synthesize some key points to better make the reader understand our ASP implementation with the enumeration example.



\subsection{Simple rules and answer sets}\label{sssec:simple_rules}
ASP is based on a set of rules of of the form:
\begin{align*}
  \underbrace{{\ }\atom{H}_{\ }}_{head} \la \underbrace{\atom{A}_1, \atom{A}_2, \dots, \atom{A}_n, \nota \atom{B}_1, \nota \atom{B}_2, \dots, \nota \atom{B}_m}_{body}.
\end{align*}
where the $body$ is a series of atoms ($\atom{A}_i$) and negations of atoms ($\nota \atom{B}_i$).
In the case of \emph{simple rules} (as opposed to the \emph{cardinality rules} of \pref{sssec:cardinality_rules}), the $head$ is also an atom ($\atom{H}$).
Such a rule, whose formal semantics is defined below,
schematically states that if all atoms $\atom{A}_1, \atom{A}_2, \dots, \atom{A}_n$ are true
and all atoms $\atom{B}_1, \atom{B}_2, \dots, \atom{B}_m$ are not true (negation by failure), then $\atom{H}$ has to be true.

An atom is composed of a predicate and a series of arguments (possibly empty).
For example, the following atom:
\begin{align*}
  \predicate{p}(x_1, x_2, \dots, x_r)
\end{align*}
is composed of the predicate $\predicate{p}$ and $r$ arguments: $x_1, x_2, \dots, x_r$.
Each argument is either a \emph{constant}, which is a representation of a piece of data (component name, expression level, …),
or a \emph{variable}, which is in fact a shorthand for any possible constants (variables are detailed below).
In this paper, constants are either numerical or consist of a single lowercase letter (e.g.~$a$, $b$, $c$, $1$, $2$, …)
while variables are always denoted by a single capital letter (e.g.~$\var{A}$, $\var{P}$, $\var{Q}$, …).
We do not use function symbols as arguments in this work.

\begin{example}\label{ex:asp-atom}
Consider a PH model such as \pref{ex:infer-param-runningPH-1}:
in order to state the existence of each component, we use a predicate called $\component$ with two arguments:
\begin{align*}
  &\component(x, n).
\end{align*}
where $x$ is the name of the component and $l_x = n$ is its maximum level of expression.
\end{example}

An ASP program is a set of rules as described above.
Solving an ASP program means finding an \emph{answer set}, which is a minimal set of atoms that respect all the rules.
In order to formally define this notion of answer set,
let us define a \emph{definite rule} as a rule with no negation of atoms (noted “$\nota$” above),
and a \emph{definite program} as a program containing only definite rules.
Let $X$ be a set of atoms: a definite rule is satisfied by $X$ if $\atom{H} \in X$,
or if $\exists i \in \segm{1}{n}, A_i \notin X$.
Given this definition of satisfaction, we define the answer set of a definite program $\Pi$
as the (unique) minimal set $X$ of atoms that satisfies all the rules in $\Pi$.

In order to consider the general case,
for all non-definite program $\Pi$ and set $X$ of atoms, we denote $\Pi^X$ the reduct of $\Pi$ w.r.t.~$X$, defined from $\Pi$ by:
\begin{enumerate}
  \item deleting all the rules that have a negation of atom $\nota \atom{B}_i$ in the body where $\atom{B}_i \in X$, and
  \item removing all negations of atoms in the bodies of the remaining rules.
\end{enumerate}
Then, a set of atoms $X$ is an answer set of a non-definite program $\Pi$ if it is the answer set of $\Pi^X$, which is a definite program.
We note that several answer sets can be solution to the same non-definite program, and in practice the solver can be directed to enumerate them all.

%\begin{example}
%  As an example not related to the current work, consider the following ASP program:
%  \begin{align*}
%    a &\la \nota b.\\
%    b &\la \nota a.
%  \end{align*}
%  This program has the two following answer sets: $\{a\}$ and $\{b\}$.
%\end{example}



\subsection{Facts}
A rule with no $body$ part is called a \emph{fact}, and its $head$ atom consequently has to belong to all answer sets.
For instance, the information describing the studied model (the original PH model and the inferred IG and parameters) are expressed in ASP using facts.
In particular, the predicate $\component$ allows to define all components belonging to the inferred IG.
Thus, an atom $\component(a, m)$ states that $a$ is a component of the IG and that $l_a = m$.

\begin{example}\label{ex:asp-component}
In order to define the 3 components of \pref{ex:infer-param-runningPH-1}, we use the following program:
\begin{align*}
  &\component(a, 2). \\
  &\component(b, 1). \\
  &\component(c, 1).
\end{align*}
This program contains only facts using the predicate $\component$ defined in \pref{ex:asp-atom},
and $a$, $b$, $c$, $1$ and $2$ are constants.
\end{example}



\subsection{Variables}
To describe the sets of all expression levels of each component (\ie the set $\segm{0}{l_a}$ for each $a \in \Gamma$),
one can use atoms of the form $\componentlevels(a, k)$ to state that $k \in \segm{0}{l_a}$.
Variables here come in handy to enumerate each possible constant $k$ for each component $a$:
during the solving, any rule containing variables is duplicated in order to replace each variable by all the possible constants it could represent.
The following rule, for example, contains three variables ($\var{A}$, $\var{K}$ and $\var{M}$) and enumerates the set of possible expression levels of each component in the system:
\begin{align*}
  \componentlevels(A, K) \la \component(\var{A}, \var{M}), 0 \leq K \leq M.
\end{align*}
where the notation “$\leq$” stands for a shortcut in ASP which has the same meaning as the mathematical operator.

\begin{example}
Regarding \pref{ex:infer-param-runningPH-1}, the previous rule together with the facts of \pref{ex:asp-component}
will give the following answer set:
\begin{align*}
  &\{&&\component(a, 2),
  &&\component(b, 1), \\
  &&&\component(c, 1),
  &&\componentlevels(a, 0), \\
  &&&\componentlevels(a, 1),
  &&\componentlevels(a, 2), \\
  &&&\componentlevels(b, 0),
  &&\componentlevels(b, 1), \\
  &&&\componentlevels(c, 0),
  &&\componentlevels(c, 1). &\}
\end{align*}
\end{example}



\subsection{Cardinality rules}
\label{sssec:cardinality_rules}
As an extension of simple rules, \emph{cardinality rules} turn out to be convenient to enumerate a set of answer sets.
The head of a cardinality rule specifies a set of atoms $H$ and two integers $min$ and $max$, and is denoted:
\begin{align*}
  min\ \{\ H\ \}\ max \la \atom{A}_1, \atom{A}_2, \dots, \atom{A}_n, \nota \atom{B}_1, \nota \atom{B}_2, \dots, \nota \atom{B}_m.
\end{align*}
Given such a rule, as many answer sets as possible are created, so that each answer set $S$ verifies:
\begin{align*}
  min \leq |S \cap H| \leq max
\end{align*}
and every atom $\atom{H}_i \in S \cap H$ respects the simple rule:
\begin{align*}
  \atom{H}_i \la \atom{A}_1, \atom{A}_2, \dots, \atom{A}_n, \nota \atom{B}_1, \nota \atom{B}_2, \dots, \nota \atom{B}_m.
\end{align*}
In other words, all answer sets contain a subset of $H$ whose cardinality goes from $min$ to $max$.
The set of atoms $H = \{ \atom{H}_1, \atom{H}_2, \dots, \atom{H}_p \}$ is often defined as: $H = \atom{P} \mid \atom{Q}$,
which is a shorthand for “the set of atoms of the form $\atom{P}$ for which $\atom{Q}$ is true”.

Cardinality rules turn out to be convenient to enumerate all possible parametrizations by creating multiple answer sets.
For functional purposes, a unique label is assigned to every possible set of resources of a given component.
Thus, we denote $\omega_p$ the set of resources of a given component $a$ labeled by $p$,
and naturally, $K_{a,\omega_p}$ is the related parameter.
%and in the following we note $K_{a,\omega_p}$ the parameter of component $a$ whose set of resources $\omega$ is assigned to the label $p$.
We note that labeling the sets of resources of a component is obviously equivalent to labeling its parameters.
Then, suppose that:
\begin{itemize}
  \item $\paramlabel(a, p)$ states that $p$ is a valid label for a set of resources of component $a$ (and therefore $K_{a,\omega_p}$ is a valid parameter);
  \item $\param(a, p, i)$ states that: $i \in K_{a, \omega_p}$;
  \item $\inferedparam(a, p)$ states that the parameter inference of $K_{a, \omega_p}$ was conclusive (\pref{pps:param_K}).
\end{itemize}
It is thereby possible to enumerate the possible values of all parameters for which \pref{pps:param_K} was not conclusive, with the following cardinality rule:
\begin{align*}
  & 1\ \{\ \param(\var{A}, \var{P}, \var{I}) \mid \componentlevels(\var{A}, \var{I})\ \}\ \infty\ \la \\
  & \qquad\qquad\qquad \paramlabel(\var{A}, \var{P}), \nota \inferedparam(\var{A}, \var{P}).
\end{align*}
Indeed, this rule applies to any possible parameter $\var{P}$ of any component $\var{A}$ ($\paramlabel$) whose value is still unknown ($\nota \inferedparam$),
and states that any expression level $\var{I}$ of this component ($\componentlevels$) can belong to the value of the parameter ($\param$).
Furthermore, the lower bound is $1$, which forces each enumerated parameter to contain at least one value,
but no upper bound is specified ($\infty$) for the size of each parameter
(which is already bounded by the number of possible expression levels of the related component).
%each parameter contains at least $1$ value, as stated by the lower bound, and has no upper bound ($\infty$) but the number of expression levels of its component.
In other worlds, this cardinality rule creates as many answer sets as there are \emph{candidate} parametrizations
so that if $K_{a, \omega_p}$ could not be inferred by \pref{pps:param_K}, then
$K_{a, \omega_p} \subset \segm{0}{l_a} \wedge K_{a, \omega_p} \neq \emptyset$
(thus completely disregarding the notion of admissible parametrizations given in \pref{ssec:admissible-K} or the fact that parameters have to be intervals).



\begin{example}\label{ex:cardinality}
In the scope of \pref{ex:infer-param-runningPH-1}, $K_{a,\{a,b\}}$ and $K_{a,\{a,c\}}$ could not be inferred by \pref{pps:param_K}.
The previous cardinality rule allows to produce 49 parametrizations, in which these two parameters can take all possible values:
%The enumeration thus produces 36 parametrizations, in which these parameters can take all possible values:
%(as, in this case, the assumptions of \pref{ssec:admissible-K} bring no new constraint):
\begin{align*}
  (K_{a,\{a,b\}} ; K_{a,\{a,c\}}) \in \{ \segm{0}{0} , \segm{1}{1} , \segm{2}{2} , \segm{0}{1} , \segm{1}{2} , \segm{0}{2}, \{ 0, 2 \} \}^2
\end{align*}
and all the other parameters keep their inferred values.
%These parametrizations take the form of sets of $\param$ atoms.
Note that $\{ 0, 2 \}$ belongs to the set of candidate parametrizations
as no rule specifying that a parameter has to be an interval has been defined yet.
\end{example}



\subsection{Constraints}\label{sssec:constraints}
Finally, a constraint is a rule with no $head$ part:
\begin{align*}
  \la \atom{A}_1, \atom{A}_2, \dots, \atom{A}_n, \nota \atom{B}_1, \nota \atom{B}_2, \dots, \nota \atom{B}_m.
\end{align*}
A constraint is satisfied only if its $body$ is not satisfied,
which thus allows to invalidate answer sets containing some unwanted combinations of atoms.
In the scope of parameters enumeration, for example, constraints are especially useful to filter parametrizations
that contain non-interval parameters, or that do not respect the assumptions of \pref{ssec:admissible-K}.
Indeed, suppose that:
\begin{itemize}
  \item $\lessactive(a, p, q)$ states that $\omega_p$ is a set of resources of $a$ with (loosely) less activators and more inhibitors than $\omega_q$;
  \item $\paraminf(a, p, q)$ states that: $K_{a,\omega_p} \leqsegm K_{a,\omega_q}$.
\end{itemize}
Then, the monotonicity assumption (\pref{pro:param_enum_monotonicity}) is formulated as the following constraint:
\begin{align*}
  \la \lessactive(\var{A}, \var{P}, \var{Q}), \nota \paraminf(\var{A}, \var{P}, \var{Q}).
\end{align*}
Indeed, this constraint removes all parametrization results where parameters $K_{\var{A},\omega_\var{P}}$ and $K_{\var{A},\omega_\var{Q}}$ exist
such that $\var{A}$ is less activated by the set of resources $\omega_\var{P}$ than it is by $\omega_\var{Q}$,
but $K_{\var{A},\omega_\var{Q}} \ltsegm K_{\var{A},\omega_\var{P}}$,
thus violating the monotonicity assumption.
Of course, other assumptions can be formulated in the same way.

\begin{example}
The set of candidate values given in \pref{ex:cardinality} can be filtered using constraints.
For example, a constraint was written in order to obtain only interval parameters
(thus avoiding parameters with “gaps” such as $\{ 0, 2 \}$).
Applying such a constraint would reduce the set of possible values for the parameters $K_{a,\{a,b\}}$ and $K_{a,\{a,c\}}$ to:
\begin{align*}
  (K_{a,\{a,b\}} ; K_{a,\{a,c\}}) \in \{ \segm{0}{0} , \segm{1}{1} , \segm{2}{2} , \segm{0}{1} , \segm{1}{2} , \segm{0}{2} \}^2
\end{align*}

Then, due to the fact that the parameters $K_{a,\{b\}} = \segm{1}{1}$ and $K_{a,\{c\}} = \segm{1}{1}$ could be inferred,
the monotonicity assumption (\pref{pro:param_enum_monotonicity}) removes all parametrizations with: $0 \in K_{a,\{a,b\}} \vee 0 \in K_{a,\{a,c\}}$.
In the end, the remaining possible values for the two parameters that were not inferred are:
\begin{align*}
  (K_{a,\{a,b\}} ; K_{a,\{a,c\}}) \in \{ \segm{1}{1} , \segm{2}{2} , \segm{1}{2} \}^2
\end{align*}
All these candidates respect the other properties of \pref{ssec:admissible-K} and no more candidates are filtered.
We thus obtain the results of \pref{ex:enum-param-runningPH-1}.
\end{example}



\medskip

This subsection succinctly described how ASP programs come in handy to represent a model and solve complex problems on it.
%all steps of Thomas' modeling inference.
It finds a particularly interesting application in the enumeration of parameters:
all possible parametrizations are generated in separate answer sets,
and integrity constraints are formulated to remove those that do not fit the assumptions of admissible parametrizations,
thus reducing the number of candidate parametrizations to be considered in the end.
However, all steps of the inference presented in this paper (\pref{sec:infer-IG} \& \ref{sec:infer-K})
were implemented in and benefited from this programming paradigm, although in significantly different ways.
