\section{Interaction Graph Inference}\label{sec:infer-IG}

In order to infer a complete BRN, one has to find the Interaction Graph (IG) first, as some
constraints on the parametrization rely on it.
Inferring the IG is an abstraction step which consists in determining the global influence of
components on each of its successors.

This section first introduces the notion of focal processes within a PH
(\pref{ssec:focal}) which is used to characterize well-formed PH for IG inference
in \pref{ssec:wf}, and as well used by the parametrization inference presented in \pref{sec:infer-K}.
Finally, the rules for inferring the interactions between components from a PH are
described in two complementary subsections with a different perspective in the inference.
\pref{ssec:infer-IG-structure} proposes an inference based on the structure of the original PH model
while \pref{ssec:infer-IG-dynamics} uses a dynamics point of view to infer the Interaction Graph.
We consider hereafter a global PH $(\PHs,\PHl,\PHa)$ on which the IG inference is to be
performed.



\subsection{Focal Processes}\label{ssec:focal}

Many of the inferences defined in the rest of this paper rely on the knowledge of \emph{focal
processes} w.r.t. a given context (a set of processes that are potentially present).
When such a context applies, we expect to (always) reach one focal process in a bounded number of
actions.

For $S_a\subseteq L_a$ and a context (set of processes) $\ctx$, let us define as $\PHa(S_a,\ctx)$
the set of actions on the sort $a$ having their hitter in $\ctx$ and target in $S_a$
(\pref{eq:PHa-ctx});
and the digraph $(V, E)$ where arcs are the bounces within the sort $a$ triggered by actions
in $\PHa(S_a,\ctx)$ (\pref{eq:bounce-graph}).
$\focals(a,S_a,\ctx)$ denotes the set of focal processes of sort $a$ in the scope of
$\PHa(S_a,\ctx)$ (\pref{def:focals}).
\begin{align}
\PHa(S_a,\ctx) & \DEF \{ \PHfrappe{b_i}{a_j}{a_k}\in\PHa \mid b_i\in\ctx \wedge a_j\in S_a \}
\label{eq:PHa-ctx}
\\
\begin{split}
E  & \DEF \{(a_j,a_k)\in (S_a \times \PHl_a) \mid 
			\exists\PHfrappe{b_i}{a_j}{a_k}\in \PHa(S_a,\ctx) \}
\\
V & \DEF S_a \cup \{ a_k\in L_a\mid \exists (a_j,a_k)\in E\}
\end{split}
\label{eq:bounce-graph}
\end{align}

\begin{definition}[$\focals(a,S_a,\ctx)$]\label{def:focals}
The set of processes that are focal for processes in $S_a$ in the scope of $\PHa(S_a,\ctx)$
are given by:
%$\focals(a,S_a,\ctx)$ is the set of focal processes of sort $a$ in the context $\ctx$:
\[
\focals(a,S_a,\ctx) \DEF
\begin{cases}
\{ a_i \in V \mid \nexists (a_i,a_j)\in E\} & \text{if the digraph $(V,E)$ is acyclic},\\
\emptyset & \text{otherwise.}\\
\end{cases}
\]
\end{definition}

We note $\PHl(\ctx)$ the set of states $s\in L$ such that $\forall a\in\PHsort(\ctx), \PHget{s}{a}\in\ctx$,
where $\PHsort(\ctx)$ is the set of sorts with processes in $\ctx$.
We say a sequence of actions $h^1,\dots,h^n$ is \emph{bounce-wise} if and only if
$\forall m\in\segm{1}{n-1}, \PHbounce(h^m)=\PHtarget(h^{m+1})$.
From \pref{def:focals}, it derives that:
\begin{enumerate}
\item if $\focals(a,S_a,\ctx)=\emptyset$, there exists a 
state $s\in \PHl(\ctx\cup S_a)$ such that $\forall n\in\mathbb N$ there
exists a bounce-wise sequence of actions $h^1,\dots,h^{n+1}$ in $\PHa(S_a,\ctx)$ 
with $\PHtarget(h^1)\in s$.
\item if $\focals(a,S_a,\ctx)\neq\emptyset$, for all
state $s\in \PHl(\ctx\cup S_a)$,
for any bounce-wise sequence of actions $h^1,\dots,h^n$ in $\PHa(S_a,\ctx)$ where $\PHtarget(h^1)\in
s$,
either
 $\PHbounce(h^n) \in \focals(a,S_a,\ctx)$,
or
$\exists h^{n+1}\in \PHa(a,\ctx)$ such that $\PHbounce(h^n) = \PHtarget(h^{n+1})$.
Moreover $n\leq|\PHa(S_a,\ctx)|$ (i.e. no cycle of actions possible).
\end{enumerate}

It is worth noticing that those bounce-wise sequences of actions may not be successively playable in
a state $s\in L(\ctx\cup S_a)$.
Indeed, nothing impose that the hitters of actions are present in $s$.
In the general case, the playability of those bounce-wise sequences, referred to as \emph{focals
reachability} may be hard to prove.
However, in the scope of this paper, the particular contexts used with $\focals$ ensure this property.
Notably, the rest of this section uses only \emph{strict} contexts (\pref{def:strict-ctx}) which
allow at most one hitter per sort in the bounce-wise sequences (and thus are present in $s$).

\begin{definition}[Strict context for $S_a$]\label{def:strict-ctx}
A context (set of processes) $\ctx$ is strict for $S_a\subseteq L_a$ if and only if
$\{b_i,b_j\} \subset \ctx \wedge b\neq a \Rightarrow i=j$.
\end{definition}

In other words, assuming focals reachability, if $\focals(a,S_a,\ctx)$ is empty, there exists a
sequence of actions that may be played an unbound number of times (cycle);
if it is non-empty, it is ensured that any state in $\PHl(\ctx\cup S_a)$ converges, in a bounded
number of steps, either to a process in $S_a$ that is not hit by processes in $\ctx$, or to a process in
$L_a\setminus S_a$.

\begin{example}
In the PH of \pref{fig:runningPH-1}, we obtain:
\begin{align*}
\focals(a,L_a,\{b_0,c_0\}) &= \{ a_0 \}
&
\focals(a,L_a,\{b_1,c_1\}) &= \{ a_2 \}
\\
\focals(a,L_a,\{b_1,c_0\}) &= \emptyset
&
\focals(a,\{a_1\},\{b_1,c_0\}) &= \{ a_0, a_2 \}
\end{align*}
\end{example}



\subsection{Well-formed Process Hitting for Interaction Graph Inference}\label{ssec:wf}

The inference of an IG from a PH assumes that the PH defines two types of sorts:
the sorts corresponding to BRN components, and the cooperative sorts.
This leads to the characterization of a \emph{well-formed} PH for IG inference.

The identification of sorts modeling components relies on the observation that their processes
represent (ordered) qualitative levels.
Hence an action on such a sort cannot make it bounce to a process at a distance more than one.
The set of sorts satisfying such a condition is referred to as $\Gamma$
(\pref{eq:PH-components}).
Therefore, in the rest of this paper, $\Gamma$ denotes the set of components of the BRN to infer.

\begin{equation}
\Gamma \DEF \{a \in \PHs \mid \nexists \PHfrappe{b_i}{a_j}{a_k} \in \PHa, |j - k| > 1\} \\
\label{eq:PH-components}
\end{equation}

Any sort that does not act as a component should then be treated as a cooperative sort.
As explained in \pref{ssec:PH}, the role of a cooperative sort $\upsilon$ is to compute the current
state of set of cooperating processes.
Hence, for each sub-state $\sigma$ formed by the sorts hitting $\upsilon$, $\upsilon$ should
converge to a focal process.
This is expressed by \pref{pro:wf-cooperative-sort}, where
the set of sorts having an action on a given sort $a$ is given by 
$\PHdirectpredec{a}$ (\pref{eq:ph_direct_predec})
and $\PHproc(\sigma)$ is the set of processes that compose the sub-state $\sigma$.

\begin{equation}
\forall a \in \PHs, \PHdirectpredec{a} \DEF \{b \in \PHs \mid \exists \PHfrappe{b_i}{a_j}{a_k}\in\PHa \}
\label{eq:ph_direct_predec}
\end{equation}

\begin{property}[Well-formed cooperative sort]\label{pro:wf-cooperative-sort}
A sort $\upsilon\in\PHs$ is a well-formed cooperative sort if and only if
each configuration $\sigma$ of its predecessors leads $\upsilon$ to a unique focal process,
denoted by $\upsilon(\sigma)$:
\[
\forall \sigma \in {\textstyle\prod_{
a\in\PHdirectpredec{\upsilon} \wedge a\neq \upsilon}}
\PHl_{a},
\focals(\upsilon,\PHl_\upsilon,\PHproc(\sigma)\cup \PHl_\upsilon) = \{ \upsilon(\sigma) \}\]
\end{property}

Such a property allows a large variety of definitions of a cooperative sort, but
for the sake of simplicity, does not allow the existence of multiple focal processes.
While this may be easily extended to (the condition becomes 
$\focals(\upsilon,\PHl_\upsilon, \PHproc(\sigma)\cup \PHl_\upsilon)\neq\emptyset$), it makes some
hereafter equations a bit more complex to read as they should handle a set of focal processes instead
of a unique focal process.


Finally, \pref{pro:wf-ph} sums up the conditions for a Process Hitting to be suitable for IG
inference.
In addition of having either component sorts or well-formed cooperative sorts, we also require that
there is no cycle between cooperative sorts, and that
sorts being never hit (\ie{} serving as an invariant environment) are components.

\begin{property}[Well-formed Process Hitting for IG inference]\label{pro:wf-ph}
A PH is well-formed for IG inference if and only if the following conditions are verified:
\begin{itemize}
\item 
each sort $a\in\PHs$ either belongs to $\Gamma$, or is a well-formed cooperative sort;
\item 
there is no cycle between cooperative sorts
(the digraph $(\Sigma,\{(a,b)\in(\Sigma\times\Sigma)\mid \exists \PHfrappe{a_i}{b_j}{b_k}\in\PHa
\wedge a\neq b\wedge \{a,b\}\cap\Gamma=\emptyset \})$ is
acyclic);
\item 
sorts having no action hitting them belong to $\Gamma$
($\{ a \in \Sigma\mid \nexists \PHfrappe{b_i}{a_j}{a_k}\in\PHa\} \subset \Gamma$).
\end{itemize}
\end{property}

\begin{example}
In the PH of \pref{fig:runningPH-2}, $bc$ is a well-formed cooperative sort as defined in \pref{pro:wf-cooperative-sort}, because:
\begin{align*}
\focals(bc, \PHl_{bc}, \{b_0, c_0\} \cup \PHl_{bc}) = \{bc_{00}\} && \focals(bc, \PHl_{bc}, \{b_0, c_1\} \cup \PHl_{bc}) = \{bc_{01}\} \\
\focals(bc, \PHl_{bc}, \{b_1, c_0\} \cup \PHl_{bc}) = \{bc_{10}\} && \focals(bc, \PHl_{bc}, \{b_1, c_1\} \cup \PHl_{bc}) = \{bc_{11}\}
\end{align*}
Hence, both \pref{fig:runningPH-1} and \pref{fig:runningPH-2} are well-formed PH for IG inference
with $\Gamma = \{a,b,c\}$.
\end{example}



\subsection{Structure-based Interaction Inference}\label{ssec:infer-IG-structure}

At this point we can divide the set of sorts $\PHs$ into components ($\Gamma$, see \pref{eq:PH-components}) and cooperative sorts
($\PHs \setminus \Gamma$) that will not appear in the IG. 
We define in \pref{eq:ph_predec} the set of predecessors of a sort $a$, that is, the sorts influencing $a$
by considering direct actions and possible intermediate cooperative sorts.
The predecessors of $a$ that are components are the regulators of $a$, denoted $\PHpredecgene{a}$
(\pref{eq:regulators}).
\begin{align}
\begin{split}
\forall a \in \PHs, \PHpredec{a} &\DEF \{b \in \PHs \mid \exists n \in \mathbb{N}^*, \exists
(c^k)_{k \in \segm{0}{n}} \in \PHs^{n+1}, \\
                                   & \quad \quad c^0 = b \wedge c^n = a \\
                                   & \quad \quad \wedge \forall k \in \segm{0}{n-1},
                   c^k \in \PHdirectpredec{c^{k+1}} \cap (\PHs\setminus\Gamma)\}
\end{split}
\label{eq:ph_predec}
\\
\forall a\in \PHs, \PHpredecgene{a} & \DEF \PHpredec{a} \cap \Gamma
\label{eq:regulators}
\end{align}

Given a set $g$ of components and a configuration (\ie a sub-state) $\sigma$, $\ctx_g(\sigma)$
refers to the set of processes hitting $a$ regulated by any sort in $g$ (\pref{eq:ctx-sigma}).
If $g=\{b\}$, we simple note $\ctx_b(\sigma)$.
This set is composed of the active processes of sorts in $g$, and the focal process (assumed
unique) of the cooperative sorts $\upsilon$ hitting $a$ that have a predecessor in $g$.
The evaluation of the focal process of $\upsilon$ in context $\sigma$, denoted $\upsilon(\sigma)$,
relies on \pref{pro:wf-cooperative-sort}, which gives its value when all the direct predecessors of
$\upsilon$ are defined in $\sigma$.
When a predecessor $\upsilon'$ is not in $\sigma$, we extend the evaluation by recursively computing
the focal value of $\upsilon'$ is $\sigma$, as stated in \pref{eq:cooperative-eval},
where $\uplus$ stands for a union of states (which are sets of processes).
Because there is no cycle between cooperative sorts, this recursive evaluation of $\upsilon(\sigma)$
always terminates.
\begin{align}
\forall g\subset \Gamma,
  \ctx_g(\sigma) & \DEF \{ \sigma[b] \mid b\in g \} \cup \{ \upsilon(\sigma) \mid
\upsilon\in\PHdirectpredec{a} \setminus \Gamma \wedge g\cap \PHpredecgene{\upsilon} \neq \emptyset \}
\label{eq:ctx-sigma}
\\
\upsilon(\sigma) & \DEF
\upsilon(\sigma \uplus \state{\upsilon'(\sigma) \mid 
  \upsilon'\in\PHdirectpredec{\upsilon} \wedge
  \upsilon'\in\PHs\setminus\Gamma })
\label{eq:cooperative-eval}
\end{align}

This first inference mainly focuses on the presence of actions betweens two sorts to conclude on the presence of a regulation.
We aim at inferring that $b$ activates (inhibits) $a$ if there exists a configuration where increasing
the level of $b$ makes possible the increase (decrease) of the level of $a$.
To do so, the sets of components cooperating together to hit $a$, called groups of influence of $a$, are studied.
Such groups are given by the set $X(a)$ that is the set of connected components in the graph linking two regulators
$b$ and $c$ of $a$ if they use a common cooperative sort to have an influence on $a$ (\pref{eq:influence-groups}).
\begin{equation}
X(a) = \mathcal C\left( (\PHpredecgene{a}, \{ \{b,c\} \mid
        \exists \upsilon\in \PHdirectpredec{a} \setminus \Gamma,
        \{b,c\} \subset \PHpredecgene{\upsilon} \}) \right)
\label{eq:influence-groups}
\end{equation}

This reasoning can then be straightforwardly applied when $b \neq a$ (\pref{eq:edges-inference-b}).
Indeed, to study the influence of any component $b$ on $a$, it is sufficient to study the (unique) group of influence
$g \in X(a)$ which $b$ belongs to. %, that is, the (unique) set in $X_a$ containing $b$, denoted $X_a(b)$.
%The reasoning mentioned in the previous paragraph can then be straightforwardly applied when $b \neq a$ (\pref{eq:edges-inference-b}),
%by considering this 
Indeed, given a configuration $\sigma\in\prod_{c\in g \cup \{ a \}} L_c$, 
$\focals(a,\{a_i\},\ctx_b(\sigma))$ gives the bounces that a given process $a_i$ can make in the
context $\ctx_b(\sigma)$.
We note $\sigma\{b_i\}$ the configuration $\sigma$ where the process of sort $b$ has been replaced
by $b_i$.
If there exists $b_i,b_{i+1}\in L_b$ such that one bounce in 
$\focals(a, \{\sigma[a]\}, \ctx_b(\sigma\{b_i\}))$
has a lower (resp. higher) level that one bounce in
$\focals(a, \{\sigma[a]\}, \ctx_b(\sigma\{b_{i+1}\}))$, then
$b$ as positive (resp. negative) influence on $a$ with a maximum threshold $t=i+1$.

Finally, we infer that $a$ has a self-influence if its current level can have an impact on its own
evolution in a given configuration $\sigma$ of any group of influence on $a$ (\pref{eq:edges-inference-a}).
Given $a_i,a_{i+1}\in L_a$, we pick $a_j\in\focals(a,\{a_i\},\ctx_g(\sigma\{a_i\}))$ and
$a_k\in\focals(a,\{a_{i+1}\},\ctx_g(\sigma\{a_{i+1}\}))$.
If $k=j+1$, we can not conclude as there is no difference in the evolution of both levels.
If $k\neq j+1$ and $k-j\neq 0$, then $a_i$ and $a_{i+1}$ have divergent evolutions: we infer an
influence of sign of $k-j$ at threshold $i+1$.
We note that some aspects of this inference are arbitrary and may impact the number of parameters to
infer in the next section.
In particular, in some cases, the use of intervals for Thomas' parameters drops the requirement of
inferring a self-activation.


\pref{pps:inference-edges} details the inference of all existing influences between components occurring
with a threshold $t$.
These influences are split into positive and negative ones, and represent possible edges in the final IG.
We do not consider the cases where a component has no visible influence on another.
\begin{proposition}[Influences inference]\label{pps:inference-edges}
We define the set of positive (resp. negative) influences $\hat{E}_+$ (resp. $\hat{E}_-$) for any
$a\in\Gamma$ by:
\begin{align}
\begin{split}
\forall b\in\PHpredecgene{a}, b &\neq a, \\
b\xrightarrow t a \in \hat{E}_s & \Longleftrightarrow
  \exists g \in X(a), b \in g, 
  \exists \sigma\in\textstyle\prod_{c\in g \cup \{ a \}} \PHl_c, \exists b_i,b_{i+1}\in \PHl_b,\\
& \qquad\qquad
        \exists a_{j}\in\focals(a, \{\sigma[a]\}, \ctx_b(\sigma\{b_i\})), \\
& \qquad\qquad
        \exists a_{k}\in\focals(a, \{\sigma[a]\}, \ctx_b(\sigma\{b_{i+1}\})), \\
& \qquad\qquad\qquad
                        s = \f{sign}(k - j) \wedge t = i+1
\end{split}
\label{eq:edges-inference-b}
\\
\begin{split}
a \xrightarrow t a \in \hat{E}_s & \Longleftrightarrow
\exists g \in X(a), \sigma \in \textstyle\prod_{b\in g \cup \{ a \}} \PHl_b,
      \exists a_i,a_{i+1}\in \PHl_a, \\
& \qquad\qquad
        \exists a_{j}\in\focals(a, \{a_i\}, \ctx_g(\sigma\{a_i\})), \\
& \qquad\qquad
        \exists a_{k}\in\focals(a, \{a_{i+1}\},  \ctx_g(\sigma\{a_{i+1}\})), \\
& \qquad\qquad\quad
      k \neq j+1
        \wedge s = \f{sign}(k - j) \wedge t = i+1
\end{split}
\label{eq:edges-inference-a}
\end{align}
where $s \in \{ +, - \}$,
$\f{sign}(n) = + \EQDEF n > 0$,
$\f{sign}(n) = - \EQDEF n < 0$,
and $\f{sign}(0) \DEF 0$.
\end{proposition}

We are now able to infer the edges of the final IG by considering positive and negative influences
(\pref{pps:inference-IG}).
We infer a positive (resp. negative) edge if there exists a corresponding influence with the same
sign. If an influence is both positive and negative, we infer an unsigned edge. In the end, the
threshold of each edge is the minimum threshold for which an influence has been found.
%As unsigned edges represent ambiguous interactions, no threshold is inferred.
\begin{proposition}[Interaction Graph inference]\label{pps:inference-IG}
We infer $\IG = (\Gamma,E)$ using \pref{pps:inference-edges} as follows:
\begin{align*}
E_+ &= \{ \GRNedge{a}{+}{t}{b} \mid \nexists a \xrightarrow{t'} b \in \hat{E}_-
  \wedge t = \min \{ l \mid a \xrightarrow{l} b \in \hat{E}_+\}\} \\
E_- &= \{ \GRNedge{a}{-}{t}{b} \mid \nexists a \xrightarrow{t'} b \in \hat{E}_+
  \wedge t = \min \{l \mid a \xrightarrow{l} b \in \hat{E}_-\}\} \\
E_\pm &= \{ \GRNedge{a}{\uns}{t}{b} \mid \exists a \xrightarrow{t'} b \in \hat{E}_+ \wedge \exists a \xrightarrow{t''} b \in \hat{E}_- \\
  & \qquad\qquad\qquad \wedge t = \min \{l \mid a \xrightarrow{l} b \in \hat{E}_- \cup \hat{E}_+\}\}
\end{align*}
\end{proposition}



\begin{example}
The IG inference from the PH of \pref{fig:runningPH-2} gives
$\hat{E}_+ = \{b \xrightarrow{1} a, c \xrightarrow{1} a\}$ and 
$\hat{E}_- = \{a \xrightarrow{2} b\}$, corresponding to the IG of \pref{fig:runningBRN}.
No self-influence are inferred. %($X(a) = \{ \{b,c\} \}$, $X(b)=\{ \{a\}\}$, and $X(c)=\emptyset$).
\rewrite{Check this example}
\end{example}



\todo{Insist on the absence of non-functional edges in the inferred IG (contrary to biologists' IGs)}



\subsection{Dynamics-based Interaction Inference}\label{ssec:infer-IG-dynamics}

This second inference is complementary to the one presented in the previous subsection.
It is based on an approach considering the global dynamics of a component rather than independent dynamics of each regulator,
%in that, it is analogous to standard IG inferences from discrete maps \cite{RiCo07}.
and is inspired from the definition of Interaction Graph built from a Transition Graph as described in~\cite{Richard2010378}.
Contrary to the previously presented method, we focus on a configuration (or: sub-state) $\sigma$
on the whole set $\reg{a}$ of predecessors of a component $a$ (rather than on a subset of cooperating regulators).

For a given configuration $\sigma$ of a component $a$ and its regulators,
$\irB_a(\sigma)$ is the set of processes towards which $a$ can bounce (\pref{eq:possible-bounces}).
This set is defined using the set $\irF_a(\sigma)$ of action hitting $\PHget{\sigma}{a}$ in $\sigma$ (\pref{eq:possible-actions})
and the set of processes $\ctx$ hitting any process of $a$ (\pref{eq:ctx-sigma}).
If $a$ cannot be hit by any action in $\sigma$, then $\irB_a(\sigma) = \{ \PHget{\sigma}{a} \}$.
%
\begin{align}
\begin{split}\label{eq:possible-actions}
  \forall \sigma &\in \textstyle\prod_{c\in \reg{a} \cup \{ a \}} \PHl_c, \sigma' = \ctx_{\reg{a}}(\sigma),\\
  & \irF_a(\sigma) \DEF \{ a_k \in \PHl_a \mid \exists b \in \PHs, \exists \PHfrappe{b_i}{a_j}{a_k} \in \PHa,\\
  & \qquad\qquad\qquad\qquad\qquad \PHget{\sigma'}{b} = b_i \wedge \PHget{\sigma'}{a} = a_j \}\\
\end{split}
\end{align}
%
\begin{align}
\begin{split}\label{eq:possible-bounces}
  \forall \sigma &\in \textstyle\prod_{c\in \reg{a} \cup \{ a \}} \PHl_c,
%    \sigma' = \ctx_{\reg{a}}(\sigma),\\
  \irB_a(\sigma) \DEF 
  \begin{cases}
    \irF_a(\sigma)
      & \text{ if } \irF_a(\sigma) \neq \emptyset\\
    \{ \PHget{\sigma}{a} \}
      & \text{ if } \irF_a(\sigma) = \emptyset
  \end{cases}
\end{split}
\end{align}

The main idea of the inference detailed in \pref{pps:inference-edges-dynamics} is close to the one of \pref{ssec:infer-IG-structure}:
the presence of a positive (negative) influence of a component $b$ on $a$ denotes the fact that
there exists a state in which increasing the level of $b$ tends to make the future level of $a$ rise (drop) (\pref{eq:edges-inference-richard-pos} \& \eqref{eq:edges-inference-richard-neg}).
the main difference lies in the fact that this inference focuses on local evolutions (the possibility for $a$ to bounce away from its original state)
rather than global evolutions (by considering focal processes denoting the overall influence).
Therefore, an influence is inferred if the possible bounces of $a$ in $\sigma$ depend on the level of $b$.
%
\begin{proposition}[Influences inference]\label{pps:inference-edges-dynamics}
We define the set of positive (resp. negative) influences $\hat{E}_+$ (resp. $\hat{E}_-$) for any $a\in\Gamma$ by:
\begin{align}
\begin{split}\label{eq:edges-inference-richard-pos}
  \forall b\in\PHpredecgene{a}, \qquad\qquad\quad\\
  b\xrightarrow{t+1} a \in \hat{E}_+ \Longleftrightarrow\ &
    \exists \sigma \in \textstyle\prod_{c\in \reg{a} \cup \{ a \}} \PHl_c, \{ b_t, b_{t+1} \} \subset \PHl_b \wedge b_t \in \sigma,\\
    &\qquad \exists a_j \in \irB_a(\sigma), \exists a_k \in \irB_a(\sigma\{b_{t+1}\}),\\
    &\qquad a_j \leq \PHget{\sigma}{a} < a_k \warnul{\vee a_j < \PHget{\sigma}{a} \leq a_k \quad^*}
\end{split}\\
\begin{split}\label{eq:edges-inference-richard-neg}
  \forall b\in\PHpredecgene{a}, \qquad\qquad\quad\\
  b\xrightarrow{t+1} a \in \hat{E}_- \Longleftrightarrow\ &
    \exists \sigma \in \textstyle\prod_{c\in \reg{a} \cup \{ a \}} \PHl_c, \{ b_t, b_{t+1} \} \subset \PHl_b \wedge b_t \in \sigma,\\
    &\qquad \exists a_j \in \irB_a(\sigma), \exists a_k \in \irB_a(\sigma\{b_{t+1}\}),\\
    &\qquad a_j > \PHget{\sigma}{a} \geq a_k \warnul{\vee a_j \geq \PHget{\sigma}{a} > a_k \quad^*}
\end{split}
\end{align}
\end{proposition}
%
\todo{\underline{$^*$Keep these large conditions?}}

Finally it is possible to infer the final IG by using the same reasoning as \pref{pps:inference-IG}:
a positive (negative) edge in inferred if there exist positives (negative) influences only, otherwise an unsigned edge in inferred.
We point out that the main use of this alternative inference is the fact that the conditions used to infer each influence
in \pref{eq:edges-inference-richard-pos} \& \eqref{eq:edges-inference-richard-neg}
allows to apply some results about the links between the structure of an IG and the dynamics of the model.
These results consist mainly in Thomas' conjectures, that focus on the presence of positive and negative circuits in the IG, and state that:
\begin{itemize}
  \item the presence of a positive circuit in the IG is a necessary condition for several fixed points,
  \item the presence of a negative circuit in the IG is a necessary condition for sustained oscillations.
\end{itemize}
The sign of a circuit is defined by the number of negative edges it contains;
given the nature of unsigned edges, the presence of such an edge in a circuit would count for the presence of both a positive and negative circuit.
These conjectures have been demonstrated in~\cite{Richard2010378} \towrite{Other references! (multi-stationarity \& boolean models)}.

\todo{Example}
