\subsection{Thomas' modeling}
We concisely present the Thomas' modeling of BRNs dynamics, merely inspired by
\cite{Richard06,BernotSemBRN}.
In order to enlarge the class of Thomas' models compatible with PH dynamics (w.r.t. the presented
inference), we extend the classical formalism by setting parameters to intervals of values instead of single
values, and briefly discuss this addition.

Thomas' formalism lies on two complementary descriptions of the system. First, the
\emph{Interaction Graph} (IG) models the structure of the system by defining the components'
mutual influences.
The \emph{parametrization} then specifies the levels to which tends a component when a given
configuration of its regulators applies.

The IG is composed of nodes that represent components, and edges labeled with a threshold that stand
for either positive or negative regulations (\pref{def:ig}).
For such a regulation to take place, the expression level of its head component has to be higher than its threshold; otherwise, the opposite influence is expressed.
The uniqueness of these regulations makes the following sections simpler.
%Therefore, for any component $b$, a predecessor $a$ of $b$ such that we have $a \xrightarrow{t} b$
%can be either an activator or an inhibitor of $a$, according to the sign of the interaction and if the expression level of $b$ if above or below the threshold $t$.
We call $\levelsA{a}{b}$ (resp. $\levelsI{a}{b}$) the levels of $a$ where it is an
activator (resp. inhibitor) of $b$ (\pref{def:levels});
$l_a$ denotes the maximum level of $a$.
%For the sake of simplicity and to establish a parallel with PH, if $a$ represents a component, we call $a_i$ its $i^\text{th}$ expression level.

\begin{definition}[Interaction Graph]
\label{def:ig}
An \emph{Interaction Graph} (IG) is a triple $(\Gamma, E_+, E_-)$ where $\Gamma$ is a finite number of \emph{components},
and $E_+$ (resp. $E_-$) $\subset \{a \xrightarrow{t} b \mid a, b \in \Gamma \wedge t \in [1; l_a]\}$
is the set of positive (resp. negative) \emph{regulations} between two nodes, labeled with a \emph{threshold}.

A regulation from $a$ to $b$ is uniquely referenced:
if $a \xrightarrow{t} b \in E_+$ (resp. $E_-$),
$\nexists a \xrightarrow{t'} b \in E_+ \text{ (resp. $E_-$)}, t' \neq t$
and $\nexists a \xrightarrow{t'} b \in E_-\text{ (resp. $E_+$)}, t' \in \mathbb{N}$.
\end{definition}

\begin{definition}[Effective levels ($\levels$)]\label{def:levels}
Let $(\Gamma,E_+,E_-)$ be an IG and $a, b \in \Gamma$ two of its components:
\begin{itemize}
  \item if $a \xrightarrow{t} b \in E_+$, $\levelsA{a}{b} \DEF [t; l_a]$ and
    $\levelsI{a}{b} \DEF [0; t-1]$;
  \item if $a \xrightarrow{t} b \in E_-$, $\levelsA{a}{b} \DEF [0; t-1]$ and
    $\levelsI{a}{b} \DEF [t; l_a]$;
  \item otherwise, $\levelsA{a}{b} \DEF \levelsI{a}{b} \DEF \emptyset$.
\end{itemize}
\end{definition}

For all component $a \in \Gamma$, 
$\GRNreg{a} \DEF \{ b\in\Gamma\mid \exists b\xrightarrow t a\in E_+\cup E_- \}$
is the set of its regulators.
We allow any number of levels for the components, without considering the number of outgoing edges,
as the number of processes in a PH sort is not constrained in any way.
%$E_+(a) = \{b \in \Gamma \mid \exists b \xrightarrow{t} a \in
%E_+\}$ (resp. $E_-(a) = \{b \in \Gamma \mid \exists b \xrightarrow{t} a \in E_-\}$) denotes the set
%of its positive (resp. negative) regulators; and
%$\GRNreg{a} = E_+(a)\cup E_-(a)$ the set of its regulators.

\begin{example*}
\pref{fig:runningBRN}(left) represents an Interaction Graph $(\Gamma,E_+,E_-)$ with
$\Gamma = \{a, b, c\}$,
$E_+ = \{b \xrightarrow{1} a, c \xrightarrow{1} a\}$ and
$E_- = \{a \xrightarrow{2} b\}$;
hence $\GRNreg{a} = \{b, c\}$.
%This IG can represent the same behavior as the PH given in \pref{fig:runningPH-2}.
\end{example*}

\begin{figure}[t]
\begin{minipage}{0.4\linewidth}
\centering
\scalebox{1.2}{
\begin{tikzpicture}[grn]
\path[use as bounding box] (-0.3,-0.75) rectangle (2.5,1.5);
\node[inner sep=0] (a) at (2,0) {a};
\node[inner sep=0] (b) at (0,0) {b};
\node[inner sep=0] (c) at (2,1.2) {c};
%\path
%  node[elabel, below=-1em of a] {$0..2$}
%  node[elabel, below=-1em of b] {$0..1$}
%  node[elabel, below=-1em of c] {$0..1$};
\path[->]
  (b) edge[bend right] node[elabel, below=-2pt] {$+1$} (a)
  (c) edge node[elabel, right=-2pt] {$+1$} (a)
  (a) edge[bend right] node[elabel, above=-5pt] {$-2$} (b);
\end{tikzpicture}
}
\end{minipage}
\begin{minipage}{0.6\linewidth}
\centering
\begin{align*}
K_{a,\{b,c\},\emptyset} &= [2 ; 2] & K_{b,\{a\},\emptyset} &= [0 ; 1] \\
K_{a,\{b\},\{c\}} &= [1 ; 1] & K_{b,\emptyset,\{a\}} &= [0 ; 0] \\
K_{a,\{c\},\{b\}} &= [1 ; 1] &&\\
K_{a,\emptyset,\{b,c\}} &= [0 ; 0] & K_{c,\emptyset,\emptyset} &= [0 ; 1]
\end{align*}
\end{minipage}
\caption{\label{fig:runningBRN}
(left)
IG example.
%Components are represented by nodes labeled with a name and possible expression levels.
Regulations are represented by the edges labeled with their sign and threshold.
For instance, the edge from $b$ to $a$ is labeled $+1$, which stands for: $b \xrightarrow{1} a \in
E_+$.
%and means that if the level of expression of $b$ is equal to (i.e. above) 1, then $b$ activates $a$,
%otherwise, $b$ inhibits $a$.
(right)
Example parametrization of the left IG.
}
\end{figure}

A \emph{state} $s$ of an IG $(\Gamma, E_+, E_-)$ is an element in $\prod_{a \in \Gamma} [0;l_a]$.
$\GRNget{s}{a}$ refers to the level of component $a$ in $s$.
The specificity of Thomas' approach lies in the use of discrete \emph{parameters} to represent the
focal level interval towards which the component will evolve in each configuration of its regulators
(\pref{def:param}).
Indeed, for each possible state of a BRN, all regulators of a component $a$ can be divided into
\emph{activators} and \emph{inhibitors}, given their type of interaction and expression level,
referred to as the \emph{resources} of $a$ in this state (\pref{def:resources}).


\begin{definition}[Discrete parameter $K_{a,A,B}$ and Parametrization $K$]\label{def:param}
For a given component $a \in \Gamma$ and $A$ (resp. $B$) $\subset \GRNreg{a}$ a set of its activators (resp. inhibitors) such that $A \cup B = \GRNreg{a}$ and $A \cap B = \emptyset$,
the discrete \emph{parameter} $K_{a,A,B} = [i; j]$ is a non-empty interval towards which $a$ will tend
in the states where its activators (resp. inhibitors) are the regulators in set $A$ (resp. $B$).
The complete map $K$ of discrete parameters for $\IG$ is called a \emph{parametrization} of $\IG$.
\end{definition}
%A consequence of this definition is that $0 \leq i_1 \leq i_2 \leq l_a$.
%We also denote: $j < K_{a,A,B} \Leftrightarrow j < i_1$ and $j > K_{a,A,B} \Leftrightarrow j> i_2$.

\begin{definition}[Resources $\GRNres{a}{s}$]\label{def:resources}
For a given state $s$ of a BRN, we define the \emph{activators} $A$ and \emph{inhibitors} $B$ of $a$ in $s$ as $\GRNres{a}{s} = A,B$, where:
\begin{align*}
  A &= \{b \in \Gamma \mid \GRNget{s}{b} \in \levelsA{b}{a}\} \\
  B &= \{b \in \Gamma \mid \GRNget{s}{b} \in \levelsI{b}{a}\}
\end{align*}
We also denote: $\GRNallres{a} = \{(A;B) \mid \exists s \in \textstyle\prod_{a \in \Gamma} [0;l_a], \GRNres{a}{s} = A,B\}$
\end{definition}

%\begin{example*}
%\pref{fig:runningBRN}(right) gives a Parametrization of the IG of \pref{fig:runningBRN}(left).
%\end{example*}

At last, \pref{def:dynamics} gives the asynchronous dynamics of a BRN using Thomas' parameters.
From a given state $s$, a transition to another state $s'$ is possible provided that only one component $a$ will evolve of one level towards $K_{a,\GRNres{a}{s}}$.

\begin{definition}[Asynchronous dynamics]\label{def:dynamics}
Let $s$ be a state of a BRN using Thomas' parameters $(\IG, K)$ where $\IG = (\Gamma, E_+, E_-)$.
The state that succeeds to $s$ is given by the indeterministic function $f(s)$:
\begin{align*}
  f(s)  & = s' \Leftrightarrow \exists a \in \Gamma,
    \GRNget{s'}{a} = f^a(s) \wedge
    \forall b \in \Gamma, b \neq a, \GRNget{s}{b} = \GRNget{s'}{b}
    \quad\text{, with}\\
  f^a(s) & =
  \begin{cases}
    \GRNget{s}{a} + 1 & \text{if } \GRNget{s}{a} < K_{a,A,B} \\% \GRNres{a}{s}} \\
    \GRNget{s}{a} & \text{if } \GRNget{s}{a} \in K_{a,A,B} \\ %\GRNres{a}{s}}\\
    \GRNget{s}{a} - 1 & \text{if } \GRNget{s}{a} > K_{a,A,B} % \GRNres{a}{s}}
  \end{cases}
\quad\text{, where $A,B=\GRNres{a}{s}$.}
\end{align*}
\end{definition}

While the use of intervals as parameter values does not add expressivity in boolean
networks, it allows to specify a larger range of dynamics in the general case (w.r.t. the above
definitions).
Indeed, assume that $K_{a,A,B} = [i ; i+2]$;
we aim at obtaining three different parameters $K_{a,A_1,B_1} = i$,  $K_{a,A_2,B_2} = i+1$,
$K_{a,A_3,B_3} = i+2$.
The only possible modification in resources is to add $a$ as a self-regulator.
However, because resources have a boolean definition (a component is either an activator or an inhibitor of
$a$), it is not possible to differentiate the 3 values.
We also remark that the use of intervals makes optional some explicit auto-activations in the IG
(as for $b$ in \pref{fig:runningBRN}, for instance).

\begin{example*}
In the BRN that consists of the IG and parametrization of \pref{fig:runningBRN}, the following
transitions are possible given the semantics defined in \pref{def:dynamics}:
$\GRNetat{a_0, b_1, c_1} \rightarrow \GRNetat{a_1, b_1, c_1} \rightarrow \GRNetat{a_2, b_1, c_1} \rightarrow
\GRNetat{a_2, b_0, c_1} \rightarrow \GRNetat{a_1, b_0, c_1}$,
ending in a steady state,
where $a_i$ is the component $a$ at level $i$.
As $K_{b,\{a\},\emptyset} = [0 ; 1]$, no auto-regulation on $b$ is needed to prevent its evolution when $a$ is not at level $2$.
\end{example*}

