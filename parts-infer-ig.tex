% vim:spell spelllang=en:
\section{Interaction Graph Inference from Process Hitting}\label{sec:infer-IG}

The Interaction Graph (IG) is an abstract representation of the direct qualitative influences,
positive and/or negative, between the components of the system.
As discussed in \pref{sec:intro}, the IG allows to efficiently characterize global dynamical
properties for the concrete system, such as the capability for multi-stationarity or oscillation.

\todo{The reviewer says about the following paragraph: “The inference of an
IG from a dynamics is interesting and important in biology. However the
motivations expressed here are not well argued. In particular I don't find
this “common”. Some bibliographical references are required to consolidate
the arguments or the sentence has to be rephrased.”}

In a typical biological network modeling process, a prior IG is generally the starting point for
the formal system specification.
However, it is common that the prior IG actually refers to interactions that reveal to be 
non-effective with respect to the dynamics.
Hence, deriving the IG directly from the dynamical models lead to more concise IGs, enhancing the
conclusiveness of static analyses based upon this abstract representation.

We consider hereafter a global PH $(\PHs,\PHl,\PHa)$ and a split of sorts
$\Gamma \cup \Delta=\PHs$ satisfying \pref{pro:wf-ph} on which the IG inference is to be
performed.
The inference of the IG is described in \pref{ssec:infer-IG}, and is illustrated
on small examples in \pref{ssec:infer-ig-examples}.


\subsection{Inference of Influences}\label{ssec:infer-IG}

\def\fp{\Phi}

We aim at inferring that $b$ activates (inhibits) $a$ if there exists a configuration where increasing
the level of $b$ makes possible the increase (decrease) of the level of $a$,
following the standard IG inference from Boolean and discrete networks \cite{Richard2010378}.

\modLP{
Given a process $a_i$ and a configuration $\sigma$ of a group $g$ of regulators of $a$,
we denote by
$\irB_a^g(\sigma)$ (\pref{def:B-bounce}) the set of processes of sort $a$ towards which $a_i$ can bounce
in the configuration $\sigma$.
$\irB_a^g(\sigma)$ matches with the focal processes of $a$ in the scope of $a_i$
(\pref{eq:B-bounce})
except when no bounce can occur in $\sigma$:
in such a case, we need to ensure that there exists a configuration of the other
groups of regulators of $a$ where $a_i$ is never hit (\pref{eq:B-no-bounce}).
The set of configuration of regulators of $a$ where $a_i$ is never hit is given
by $\fp(a_i)$ (\pref{eq:fp}).
\begin{definition}[$\irB_a^g(\sigma)$]
\label{def:B-bounce}
Given $a\in\PHs$, $g\in X(a)$ and $\sigma\in L(\PHpredecgene{a}\cup\{a\})$
such that $a_i\in\sigma$,
$\irB_a^g(\sigma)\subseteq L_a$ is defined as follows:
\begin{align}
\forall a_j\in L_a, j\neq i,
a_j \in \irB_a^g(\sigma) &\EQDEF
a_j \in \focals(a, \{a_i\}, \ctx_a^g(\sigma))
%\exists b_k\in\ctx_{g}(\sigma): \PHfrappe{b_k}{a_i}{a_j}\in\PHa
\label{eq:B-bounce}
\\
a_i \in \irB_a^g(\sigma) &\EQDEF \exists\sigma'\in\fp(a_i): \sigma\subset\sigma'
\label{eq:B-no-bounce}
\end{align}
where $\fp(a_i) \subseteq  L(\PHpredecgene{a}\cup\{a\}$ is defined as follows:
\begin{equation}
\fp(a_i) \DEF \{\sigma\in L(\PHpredecgene{a}\cup\{a\}) \mid a_i\in\sigma \wedge
					\focals(a, \{a_i\}, \ctx_a^{\PHpredecgene{a}}(\sigma)) =
					\{a_i\} \}
\label{eq:fp}
\end{equation}
\end{definition}}

\pref{pps:inference-edges} details the inference of all existing influences between components occurring
with a threshold $t$.
\modLP{The inference relies on finding configurations $\sigma$ of group of regulators of $a$
such that the increase of one of the regulator $b\in g$ from $b_t$ to $b_{t+1}$
makes possible the increase (decrease) of $a$ (\pref{eq:edges-inference} when
$b\neq a$ and \pref{eq:edges-inference-auto} when $b=a$)}.
Therefore, these influences are split into positive and negative ones, and represent possible edges in the final IG.
When $a$ has no regulators, it derives that there is a positive self-influence
of $a$ for each level above $1$ (\pref{eq:edges-inference-noreg}).
%
\begin{proposition}[Influences inference]\label{pps:inference-edges}
We define the set of positive (resp. negative) influences $\hat{E}_+$ (resp.
$\hat{E}_-$) by
\modLP{
\begin{align}
\begin{split}
\forall a\in\Gamma, \forall b\in\PHpredecgene{a}, b\neq a, & \forall s \in \{ +, - \}, \forall t < l_b\\
  b \xrightarrow{t+1} a \in \hat{E}_s \EQDEF & \exists g \in X(a), 
  	b \in g, \exists \sigma \in L(g\cup\{a\}), \\
    \qquad & \exists a_j \in \irB_a^g(\sigma\{b_t\}), \exists a_k \in \irB_a^g(\sigma\{b_{t+1}\}), \\
	\qquad & k > j \wedge s=+  \vee j < t \wedge s=-
\end{split}
\label{eq:edges-inference}
\\
\begin{split}
 \forall a\in\Gamma: a\in\PHpredecgene{a}, & \forall s \in \{ +, - \}, \forall t < l_a\\
  a \xrightarrow{t+1} a \in \hat{E}_s \EQDEF & \exists g \in X(a), 
  	 \exists \sigma \in L(g\cup\{a\}), \\
    \qquad & \exists a_j \in \irB_a^g(\sigma\{a_t\}), \exists a_k \in \irB_a^g(\sigma\{a_{t+1}\}), \\
	\qquad & k\geq t+1 \wedge j \leq t \wedge s=+ \\
	\qquad & \vee k<t+1 \wedge j>t \wedge s=-
\end{split}
\label{eq:edges-inference-auto}
\\
\forall a\in\Gamma: \PHpredecgene{a} = \emptyset, &
	\forall t\in\{0,\cdots,l_a-1\},
	a \xrightarrow{t+1} a \in \hat{E}_+
\label{eq:edges-inference-noreg}
\end{align}
}
\end{proposition}

We are now able to infer the edges of the final IG by considering positive and negative influences
(\pref{pps:inference-IG}).
We infer a positive (resp. negative) edge if there only exist corresponding influences with the same sign.
If an influence is both positive and negative, we infer an unsigned edge.
In the end, the threshold of each edge is the minimum threshold for which an influence has been found.
%
\begin{proposition}[Interaction Graph inference]\label{pps:inference-IG}
We infer $\IG = (\Gamma,E)$ using \pref{pps:inference-edges} as follows:
\begin{align*}
E_+ &= \{ \GRNedge{a}{+}{t}{b} \in \hat E_+ \mid \nexists a \xrightarrow{t'} b \in \hat{E}_-
  \wedge t = \min \{ l \mid a \xrightarrow{l} b \in \hat{E}_+\}\} \\
E_- &= \{ \GRNedge{a}{-}{t}{b} \in \hat E_- \mid \nexists a \xrightarrow{t'} b \in \hat{E}_+
  \wedge t = \min \{l \mid a \xrightarrow{l} b \in \hat{E}_-\}\} \\
E_\pm &= \{ \GRNedge{a}{\uns}{t}{b} \mid \exists a \xrightarrow{t'} b \in \hat{E}_+ \wedge \exists a \xrightarrow{t''} b \in \hat{E}_- \\
  & \qquad\qquad\qquad \wedge t = \min \{l \mid a \xrightarrow{l} b \in \hat{E}_- \cup \hat{E}_+\}\}
\end{align*}
\end{proposition}


\subsection{Examples}
\label{ssec:infer-ig-examples}

The IG inference of the PH of \pref{fig:runningPH-2} gives the
IG in \pref{fig:BRN-inf1}, containing the following edges:
\begin{align*}
  E_+ &= \{\GRNedgef{b}{+}{1}{a}, \GRNedgef{c}{+}{1}{a}, \GRNedgef{a}{+}{1}{a}, \GRNedgef{b}{+}{1}{b}, \GRNedgef{c}{+}{1}{c}\}\\
  E_- &= \{\GRNedgef{a}{-}{2}{b}\} \qquad\qquad\qquad\qquad\qquad
  E_\uns = \emptyset
\end{align*}
This IG is close to the one in \pref{fig:runningBRN} but not equivalent,
as each component has an additional auto-action.
The auto-actions on $b$ and $c$ are the consequence of a global stability
in some configurations: $c$ never evolves, and neither does $b$ when $a_2$ is not active.
The auto-action on $a$ is mainly caused by its multi-valued nature.

The inference of the PH of \pref{fig:runningPH-1}
(without refinement with cooperative sort) gives the same IG.

\begin{figure}[t]
\centering
\scalebox{1.2}{
\begin{tikzpicture}[grn]
  \path[use as bounding box] (-1.3,-0.75) rectangle (3.5,1.5);
  \node[inner sep=0] (a) at (2,0) {a};
  \node[inner sep=0] (b) at (0,0) {b};
  \node[inner sep=0] (c) at (2,1.2) {c};
  \path[->]
    (b) edge[bend right] node[elabel, below=-2pt] {$+1$} (a)
    (c) edge node[elabel, right=-2pt] {$+1$} (a)
    (a) edge[bend right] node[elabel, above=-5pt] {$-2$} (b)
    (b) edge[in=-15+180, out=15+180, loop] node[elabel, left=-2pt] {+1} (b)
    (c) edge[in=15, out=-15, loop] node[elabel, right=-2pt] {+1} (c)
    (a) edge[in=15, out=-15, loop] node[elabel, right=-2pt] {+1} (a);
\end{tikzpicture}
}
\caption{\label{fig:BRN-inf1}
  Result of the IG inference performed on the PH of \pref{fig:runningPH-2}.
}
\end{figure}

If we add the action $\PHfrappe{a_2}{b_0}{b_1}$ to the PH of \pref{fig:runningPH-2},
then two unsigned edges towards $b$ are inferred instead of the previous signed edges:
\begin{align*}
  E_+ &= \{\GRNedgef{b}{+}{1}{a}, \GRNedgef{c}{+}{1}{a}, \GRNedgef{a}{+}{1}{a}, \GRNedgef{c}{+}{1}{c}\}\\
  E_- &= \emptyset \qquad\qquad\qquad\qquad
  E_\uns = \{\GRNedgef{a}{\uns}{2}{b}, \GRNedgef{b}{\uns}{1}{b}\}
\end{align*}
This is due to the fact that the actions $\PHfrappe{a_2}{b_1}{b_0}$ and $\PHfrappe{a_2}{b_0}{b_1}$
introduce an oscillation only caused by $a$, which cannot be represented in Thomas modeling.
