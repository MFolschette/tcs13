% vim:spell spelllang=en:
\section{From Local Transitions to Global Functions}\label{sec:tr2global}

Process Hitting (PH) and Thomas modelling rely on two different paradigms for the specification of
network dynamics:
Thomas parametrization defines a global function
which associates to each component and each resource configuration a level toward which the component
eventually evolves in the given configuration.
Such a function is always deterministic (whereas the associated asynchronous dynamics can be
indeterministic due to concurrent evolution of components).
In contrast, 
PH models specify a list of local transitions between the levels (processes) of the components (sorts)
that are conditioned with the presence of other processes.
In addition, PH allows the use of intermediate sorts, such as cooperative sorts, that do not refer
to the components of the network.

Hence, identifying Thomas models from PH requires a step that lump a set of transitions that can be
applied in a given context into a single process that correspond to the level toward which the
component evolve.
This step is detailed in \pref{ssec:focal} in which is defined the $\focals$ function associating 
the farthest reachable processes, called \emph{focal processes}, of a given sort for a given context.
It is possible that several different focal processes are identified, indicating an indeterministic
evolution of the sort;
and it is also possible possible that no focal process exist, when in presence of terminal cycles of
transitions.
In those cases, there is no possible Thomas specification for the PH.
In that way, necessary conditions for Thomas model identification from PH are described in
\pref{ssec:dfocals}.

\subsection{Focal Processes}\label{ssec:focal}

Given a sort $a\in\Sigma$ and a context delimited by 
a sub-set of processes of sort $a$ and a sub-state $\sigma$
of the sorts having an action on $a$,
$\focals(a,S_a,\sigma)$ (\pref{def:focals}) is the set of processes of sort $a$ towards which
$a$ will eventually converge in the scope of the context.
The context delimits the set of states $s\in L$ where $s[a]\in S_a$, and for all 
sort $b\in\PHpred a, b\neq a$,
$s[b]=\sigma[b]$.
$\focals(a,S_a,\sigma)$ is obtained from the digraph where $a_i\in S_a$ is connected to $a_j\in L_a$ only if there
exists an action $\PHfrappe{b_k}{a_i}{a_j}$ where either $b_k = a_i$ or $b_k$ is in $\sigma$.
$\focals(a,S_a,\sigma)$ is empty if there is any cyclic terminal connected components;
otherwise, it is exactly the leafs of the digraph.


\begin{definition}[$\focals(a,S_a,\sigma)$]\label{def:focals}
Given a sort $a\in\Sigma$, a sub-set of its processes $S_a\subset L_a$ 
and a sub-state
$\sigma \in {\prod_{b\in\PHdirectpredec{a}, b\neq a} \PHl_{b}}$,
\[
\focals(a,S_a,\sigma) \DEF
\begin{cases}
\emptyset & \text{if }\f{tscc}(V,E)\neq\emptyset\\
\{ a_i \in V \mid \nexists (a_i,a_j)\in E\} & \text{otherwise,}
\end{cases}
\]
where
\begin{align}
E  & \DEF \{(a_i,a_j)\in (S_a \times \PHl_a) \mid 
			\exists\PHfrappe{b_k}{a_i}{a_j}\in \PHa:
				(b_k = a_i \vee \sigma[b]=k) \}
\\
V & \DEF S_a \cup \{ a_j\in L_a\mid \exists (a_i,a_j)\in E\}
\end{align}
and $\f{tscc}(V,E)$ are the non-elementary terminal strongly connected components of the digraph
$(V,E)$:
\[
\f{tscc}(V,E) = \{
W\in\f{scc}(V,E) \mid \card{W}\geq 2\wedge \forall a_i\in W, (a_i,a_j)\in E\Rightarrow a_j\in W\}
\]
with $\f{scc}(V,E)$ the strongly connected components of the digraph $(V,E)$.
\end{definition}

From \pref{def:focals} can be derived 
\pref{pps:no-focals} -- if $\focals(a,S_a,\sigma)$ is empty, there exists a limit cycle in the evolution of
$a$ in the given context;
and \pref{pps:has-focals} -- if $\focals(a,S_a,\sigma)$ is non empty, all the evolutions of $a$ in the scope
of the given context eventually terminate, and the resulting processes are in $\focals(a,S_a,\sigma)$.
In other words, if $\focals(a,S_a,\sigma)$ is empty, there exists a
sequence of actions that may be played an unbound number of times (cycle) in the given context;
if it is non-empty, it is ensured that any state in the given context converges 
in a bounded number of steps to a process in $\focals(a,S_a,\sigma)$.

\begin{proposition}
\label{pps:no-focals}
$\focals(a,S_a,\sigma)=\emptyset$ if and only if
there exists a 
state $s\in L$
where
$s[a]\in S_a$ and
$\forall b\in \PHpred a, b\neq a, s[b] = \sigma[b]$,
such that
$\forall n\in \mathbb N$
there 
exists a sequence of actions $h^1,\dots,h^{n+1}$ in $\PHa$
sequentially playable in $s$ with
$\PHtarget(h^1)\in S_a$ and
$\forall m\in\segm{1}{n-1}, \PHbounce(h^m)=\PHtarget(h^{m+1})$.
\end{proposition}

\begin{proposition}
\label{pps:has-focals}
If $\focals(a,S_a,\sigma)\neq\emptyset$, for all
state $s\in L$
where
$s[a]\in S_a$ and
$\forall b\in \PHpred a, b\neq a, s[b] = \sigma[b]$,
either
$\nexists h\in\PHa$ with $\PHtarget(h)\in S_a$  playable in $s$ and
$s[a]\in\focals(a,S_a,\sigma)$;
or
there exists a sequence of actions $h^1,\dots,h^n$ in $\PHa$ sequentially playable in $s$ with
$\PHtarget(h^1)\in S_a$ and
$\forall m\in\segm{1}{n-1}, \PHbounce(h^m)=\PHtarget(h^{m+1})$
where
 $\PHbounce(h^n) \in \focals(a,S_a,\sigma)$
 and
 where 
$\nexists h^{n+1}\in \PHa$ with $\PHtarget(h^{n+1})=\PHbounce(h^n)$ 
that is playable in $s\play h^1\play \cdots \play h^n$.
\end{proposition}



\begin{example}
In the PH of \pref{fig:runningPH-1}, we obtain:
\begin{align*}
\focals(a,L_a,\state{b_0,c_0}) &= \{ a_0 \}
&
\focals(a,L_a,\state{b_1,c_1}) &= \{ a_2 \}
\\
\focals(a,L_a,\state{b_1,c_0}) &= \emptyset
&
\focals(a,\{a_1\},\state{b_1,c_0}) &= \{ a_0, a_2 \}
\end{align*}
\end{example}

\subsection{Restrictions for Determinism} % or Well-formed Cooperative Sorts
\label{ssec:dfocals}

As discussed at the beginning of this section, and as formally described in the above sub-section, 
in the scope of a given context,
a PH sort may converge to one among several different processes (indeterminism) or may never converge
(infinite loop).

\todo{this statement is too strong if we consider intervals, but correct if we are not}
It results that deriving Thomas specification for dynamics of a PH model
requires that every sort converges toward a unique focal process, i.e., that every sort has a
deterministic behaviour w.r.t. any context, as stated by \pref{pro:deterministic-sort}.

\begin{property}[Deterministic sort]\label{pro:deterministic-sort}
A sort $a\in\PHs$ is a deterministic sort if and only if
each configuration $\sigma$ of its predecessors leads $a$ to a unique focal process,
denoted $a(\sigma)$:
\[
\forall \sigma \in {\textstyle\prod_{
b\in\PHdirectpredec{\upsilon} \wedge b\neq a}}
\PHl_{b},
\focals(a,\PHl_a,\sigma) = \{ a(\sigma) \}\]
\end{property}

\todo{examples}

