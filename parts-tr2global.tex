% vim:spell spelllang=en:
\section{From Local Transitions to Global Functions}\label{sec:tr2global}

Process Hitting (PH) and Thomas modelling rely on two different paradigms for the specification of
network dynamics:
Thomas parametrization defines a global function
which associates to each component and each resource configuration a level toward which the component
eventually evolves in the given configuration.
Such a function is always deterministic (whereas the associated asynchronous dynamics can be
indeterministic due to concurrent evolution of components).
In contrast, 
PH models specify a list of local transitions between the levels (processes) of the components (sorts)
that are conditioned with the presence of other processes.
In addition, PH allows the use of intermediate sorts, such as cooperative sorts, that do not refer
to the components of the network.

Hence, identifying Thomas models from PH requires to lump a set of transitions that can be
applied in a given configuration into a single process that correspond to the level toward which the
component evolve.
This step is detailed in \pref{ssec:focal} in which is defined the $\focals$ function associating 
the farthest reachable processes, called \emph{focal processes}, of a given sort
for a given configuration.
It is possible that several different focal processes are identified, indicating an indeterministic
evolution of the sort;
and it is also possible possible that no focal process exist, when in presence of terminal cycles of
transitions.
In those cases, there is no possible Thomas specification for the PH.
The \pref{ssec:dfocals} discusses on constraints for obtaining deterministic
(local) behaviours in PH.

Finally, \pref{ssec:split-sorts} addresses the separation of PH sorts into
components and intermediate cooperate sorts, the latter having to be masked in Thomas models.
It notably defines the predecessors, (group of) regulators of a
sort, and the extension of a configuration to incorporate the state of the
cooperative sorts.

\subsection{Focal Processes}\label{ssec:focal}

Given a sort $a\in\Sigma$ and a configuration delimited by 
a sub-set of processes of sort $a$ and a sub-state $\sigma$
of the sorts having an action on $a$,
$\focals(a,S_a,\sigma)$ (\pref{def:focals}) is the set of processes of sort $a$ towards which
$a$ will eventually converge in the scope of the configuration.
The configuration delimits the set of states $s\in L$ where $s[a]\in S_a$, and for all 
sort $b\in\PHpred a, b\neq a$,
$s[b]=\sigma[b]$.
$\focals(a,S_a,\sigma)$ is obtained from the digraph where $a_i\in S_a$ is connected to $a_j\in L_a$ only if there
exists an action $\PHfrappe{b_k}{a_i}{a_j}$ where either $b_k = a_i$ or $b_k$ is in $\sigma$.
$\focals(a,S_a,\sigma)$ is empty if there is any cyclic terminal connected components;
otherwise, it is exactly the leafs of the digraph.


\begin{definition}[$\focals(a,S_a,\sigma)$]\label{def:focals}
Given a sort $a\in\Sigma$, a sub-set of its processes $S_a\subset L_a$ 
and a sub-state
$\sigma \in {\prod_{b\in\PHdirectpredec{a}, b\neq a} \PHl_{b}}$,
\[
\focals(a,S_a,\sigma) \DEF
\begin{cases}
\emptyset & \text{if }\f{tscc}(V,E)\neq\emptyset\\
\{ a_i \in V \mid \nexists (a_i,a_j)\in E\} & \text{otherwise,}
\end{cases}
\]
where
\begin{align}
E  & \DEF \{(a_i,a_j)\in (S_a \times \PHl_a) \mid 
			\exists\PHfrappe{b_k}{a_i}{a_j}\in \PHa:
				(b_k = a_i \vee \sigma[b]=k) \}
\\
V & \DEF S_a \cup \{ a_j\in L_a\mid \exists (a_i,a_j)\in E\}
\end{align}
and $\f{tscc}(V,E)$ are the non-elementary terminal strongly connected components of the digraph
$(V,E)$:
\[
\f{tscc}(V,E) = \{
W\in\f{scc}(V,E) \mid \card{W}\geq 2\wedge \forall a_i\in W, (a_i,a_j)\in E\Rightarrow a_j\in W\}
\]
with $\f{scc}(V,E)$ the strongly connected components of the digraph $(V,E)$.
\end{definition}

From \pref{def:focals} can be derived 
\pref{pro:no-focals} -- if $\focals(a,S_a,\sigma)$ is empty, there exists a limit cycle in the evolution of
$a$ in the given configuration;
and \pref{pro:has-focals} -- if $\focals(a,S_a,\sigma)$ is non empty, all the evolutions of $a$ in the scope
of the given configuration eventually terminate, and the resulting processes are in $\focals(a,S_a,\sigma)$.
In other words, if $\focals(a,S_a,\sigma)$ is empty, there exists a
sequence of actions that may be played an unbound number of times (cycle) in the
given configuration;
if it is non-empty, it is ensured that any state in the configuration converges 
in a bounded number of steps to a process in $\focals(a,S_a,\sigma)$.

\begin{property}
\label{pro:no-focals}
$\focals(a,S_a,\sigma)=\emptyset$ if and only if
there exists a 
state $s\in L$
where
$s[a]\in S_a$ and
$\forall b\in \PHpred a, b\neq a, s[b] = \sigma[b]$,
such that
$\forall n\in \mathbb N$
there 
exists a sequence of actions $h^1,\dots,h^{n+1}$ in $\PHa$
sequentially playable in $s$ with
$\PHtarget(h^1)\in S_a$ and
$\forall m\in\segm{1}{n-1}, \PHbounce(h^m)=\PHtarget(h^{m+1})$.
\end{property}

\begin{property}
\label{pro:has-focals}
If $\focals(a,S_a,\sigma)\neq\emptyset$, for all
state $s\in L$
where
$s[a]\in S_a$ and
$\forall b\in \PHpred a, b\neq a, s[b] = \sigma[b]$,
either
$\nexists h\in\PHa$ with $\PHtarget(h)\in S_a$  playable in $s$ and
$s[a]\in\focals(a,S_a,\sigma)$;
or
there exists a sequence of actions $h^1,\dots,h^n$ in $\PHa$ sequentially playable in $s$ with
$\PHtarget(h^1)\in S_a$ and
$\forall m\in\segm{1}{n-1}, \PHbounce(h^m)=\PHtarget(h^{m+1})$
where
 $\PHbounce(h^n) \in \focals(a,S_a,\sigma)$
 and
 where 
$\nexists h^{n+1}\in \PHa$ with $\PHtarget(h^{n+1})=\PHbounce(h^n)$ 
that is playable in $s\play h^1\play \cdots \play h^n$.
\end{property}



\begin{example}
In the PH of \pref{fig:runningPH-1}, we obtain:
\begin{align*}
\focals(a,L_a,\state{b_0,c_0}) &= \{ a_0 \}
&
\focals(a,L_a,\state{b_1,c_1}) &= \{ a_2 \}
\\
\focals(a,L_a,\state{b_1,c_0}) &= \emptyset
&
\focals(a,\{a_1\},\state{b_1,c_0}) &= \{ a_0, a_2 \}
\end{align*}
\end{example}

\subsection{Restrictions for Determinism} % or Well-formed Cooperative Sorts
\label{ssec:dfocals}

As discussed at the beginning of this section, and as formally described in the above sub-section, 
in the scope of a given configuration,
a PH sort may converge to one among several different processes (indeterminism) or may never converge
(infinite loop).

We call a \emph{deterministic sort} \pref{def:deterministic-sort} a sort $a$
that has a single focal process for each possible configuration of its predecessors in the scope of
$L_a$.

\begin{definition}[Deterministic sort]\label{def:deterministic-sort}
A sort $a\in\PHs$ is a deterministic sort if and only if
each configuration $\sigma$ of its predecessors leads $a$ to a unique focal process,
denoted $a(\sigma)$:
\[
\forall \sigma \in L(\PHdirectpredec{\upsilon}),
\focals(a,\PHl_a,\sigma) = \{ a(\sigma) \}\]
\end{definition}

For the inference of the interaction graph and Thomas parameters from a PH model, the following
sections assume that \emph{all the cooperative sorts are deterministic}.
\todo{add more explanation depending on the text of the next sections}


\subsection{Separating components from cooperative sorts}
\label{ssec:split-sorts}

The identification of a BRN from a PH assumes that the PH defines two types of sorts:
the sorts corresponding to BRN components -- noted $\Gamma$ --, and the cooperative
\modLP{(intermediate)} sorts -- noted $\Delta$ --
\modLP{which should not appear in the BRN}.
\modLP{In this subsection, we first give a criteria for identifying the sorts of a PH
as either a component or cooperative sort.
Then, we characterize the \emph{well-formed} PH for IG inference.}

\modLP{The delimitation of sorts modeling components relies on the observation that their processes
represent (ordered) qualitative levels.
Hence an action on such a sort cannot make it bounce to a process at a distance more than one.
This set of sorts in denoted by $\hat\Gamma$ (\pref{eq:PH-components}), whereas the
set of cooperative sorts is denoted by $\hat\CSorts$ (\pref{eq:PH-coop}).}
\begin{align}
\hat\Gamma  &\DEF \{a \in \PHs \mid \nexists \PHfrappe{b_i}{a_j}{a_k} \in \PHa, |j - k| > 1\}
\label{eq:PH-components}\\
\hat\CSorts &\DEF \PHs \setminus \hat\Gamma
\label{eq:PH-coop}
\end{align}

\modLP{Given a PH and a partition of its sorts in components $\Gamma$ and cooperative
sorts $\Delta$,
\pref{pro:wf-ph} establishes conditions for BRN identification:
in addition of having $\Gamma$ compatible with $\hat\Gamma$ and cooperative
sorts being deterministic (\pref{def:deterministic-sort})},
we also require that there is no cycle between cooperative sorts, and that
sorts being never hit (\ie serving as an invariant environment) are components.

\begin{property}[Well-formed Process Hitting for BRN identification]\label{pro:wf-ph}
A PH is well-formed for BRN identification with
$\PHs = \Gamma\cup\CSorts$ only if the following conditions hold:
\begin{enumerate}
\item 
\modLP{$\Gamma\cap\CSorts=\emptyset$, $\Gamma\subseteq\hat\Gamma$ and $\forall \upsilon\in\CSorts$, 
$\upsilon$ is a deterministic sort (\pref{def:deterministic-sort})};
\item 
there is no cycle between cooperative sorts
(the digraph $(\Sigma,\{(a,b)\in(\Sigma\times\Sigma)\mid \exists \PHfrappe{a_i}{b_j}{b_k}\in\PHa
\wedge a\neq b\wedge \{a,b\}\cap\Gamma=\emptyset \})$ is
acyclic);
\item 
sorts having no action hitting them belong to $\Gamma$
($\{ a \in \Sigma\mid \nexists \PHfrappe{b_i}{a_j}{a_k}\in\PHa\} \subset \Gamma$).
\end{enumerate}
\end{property}

\begin{example}
In the PH of \pref{fig:runningPH-2}, $bc$ is a deterministic sort as defined in
\pref{def:deterministic-sort}:
\begin{align*}
\focals(bc, \PHl_{bc}, \state{b_0, c_0}) = \{bc_{00}\} &&
\focals(bc, \PHl_{bc}, \state{b_0, c_1}) = \{bc_{01}\} \\
\focals(bc, \PHl_{bc}, \state{b_1, c_0}) = \{bc_{10}\} && 
\focals(bc, \PHl_{bc}, \state{b_1, c_1}) = \{bc_{11}\}
\end{align*}
Hence, both \pref{fig:runningPH-1} and \pref{fig:runningPH-2} are well-formed PH
for BRN identification with $\Gamma = \{a,b,c\}$ and $\CSorts=\{bc\}$.
\end{example}

\modLP{Assuming a PH and a split $\Gamma$ and $\CSorts$ of its sort that satisfy
\pref{pro:wf-ph}, }
\pref{def:ph-pred} characterizes the set of predecessors of a sort $a$ as the sorts influencing $a$
through direct actions or intermediate cooperative sorts.
The predecessors of $a$ that are components are the regulators of $a$, denoted $\PHpredecgene{a}$
(\pref{eq:regulators}).
\modLP{%
\begin{definition}[$\PHpredec{a}$ and $\PHpredecgene{a}$]
\label{def:ph-pred}
\label{def:ph-reg}
Given $a\in\PHs$, $\PHpredec{a}\subseteq\PHs$ is the smallest set satisfying the following
conditions:
\begin{align}
\PHdirectpredec{a} &\subseteq \PHpredec{a}
\\
\upsilon\in\PHpredec{a}\cap\CSorts \Rightarrow \PHpredec{\upsilon}&\subseteq\PHpredec{a}
\end{align}
The components that are predecessors of $a$ are referred to as $\PHpredecgene{a}$:
\begin{equation}
\PHpredecgene{a} \DEF \PHpredec{a} \cap \Gamma
\label{eq:regulators}
\end{equation}
\end{definition}}

\modLP{%
As described in the introduction of this section, the identification of Thomas
models from PH relies on the inference of focal processes in the
different configurations of the regulators of each component.
In order to reduce the scope of the configurations that need to be enumerated, 
we introduce the notion of \emph{group of regulators} of sort $a$ as a set of
components that have a joint influence on $a$.
More precisely, $b$ and $c$ are in the same group of regulators of $a$ if there exists
an intermediate cooperative sort $\upsilon$ that hits $a$ such that $b$ and $c$
are regulators of $\upsilon$.
Those groups form a partition of $\PHpredecgene{a}$.
We denote $X(a)$ the finest partition in groups of
regulators (\pref{def:influence-groups}), i.e. where each group is minimal.
\begin{definition}[Partition $X(a)$ of $\PHpredecgene{a}$]
\label{def:influence-groups}
Given $a\in\PHs$,
$X(a)$ is the finest partition of $\PHpredecgene{a}$ such that
for any $b,c\in\PHpredecgene{a}$, $b\neq c$, if there exists a cooperative sort
$\upsilon\in\PHdirectpredec{a}\cap\CSorts$ such that 
$\{b,c\}\subseteq\PHpredecgene{\upsilon}$ then
there exists $g\in X(a)$ such that $\{b,c\}\subseteq g$.
\end{definition}
}


\modLP{%
If the cooperative sorts are all deterministic (\pref{def:deterministic-sort}),
it is sufficient to specify the configuration $\sigma$ of a group $g$ of regulators of a sort $a$
to obtain the configuration of all the sorts hitting $a$ directly.
Indeed, because of the absence of cycles between cooperative sorts
(\pref{pro:wf-ph}), one can recursively
evaluate the focal process of each cooperative sort that hit $a$ with respect to
the configuration $\sigma$ of the regulators $g$.
Such an extended configuration is denoted by $\ctx_a^g(\sigma)$, formalized in
\pref{def:ctx-sigma}
\begin{definition}[Configuration extension $\ctx_a^g(\sigma)$]
\label{def:ctx-sigma}
Given $a\in\PHs$, a group $g\in X(a)$ and a configuration 
$\sigma\in L(g)$, $\ctx_a^g(\sigma)$ is the configuration $\sigma$ with the corresponding
processes of the cooperative sorts in $\PHdirectpredec{a}$:
\begin{align}
\ctx_a^g(\sigma) & \DEF \sigma \uplus \state{\upsilon(\sigma) \mid\upsilon \in\PHdirectpredec{a} \cap \CSorts} 
\label{eq:ctx-sigma}
\\
\upsilon(\sigma') & \DEF
\begin{cases}
\focals(\upsilon,L_\upsilon,\sigma') & \text{if
$\PHdirectpredec{\upsilon}\subseteq\f{dom}(\sigma')$}
\\
\upsilon(\sigma' \uplus \state{\upsilon'(\sigma') \mid 
  \upsilon'\in\PHdirectpredec{\upsilon} \cap \CSorts })
  & \text{otherwise.}
\end{cases}
\label{eq:cooperative-eval}
\end{align}
\end{definition}
}


