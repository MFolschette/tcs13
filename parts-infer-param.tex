\section{Parametrization inference}\label{sec:infer-K}

\todo{In this while section: add informal explanations and maybe more examples (see previous section).}

Given the IG inferred from a PH as presented in the previous section, one can find the discrete parameters
that model the behavior of the studied PH using the method presented in the following.
It relies on an exhaustive enumeration of all predecessors of each component in order to find attractor processes
and returns a possibly incomplete parametrization, given the exhaustiveness of the cooperations (\pref{ssec:infer-K}).
The last step consists of the enumeration of all compatible complete parametrizations (\pref{ssec:admissible-K}) given this
set of inferred parameters, the PH dynamics and some biological constraints on parameters.

\todo{(LP) parameters equivalence:
if $\GRNedge{a}{+}{t}{a}$,
$K_{a,R\setminus \{a\}} = t
\equiv
K_{a,R\setminus \{a\}} = u$
for $t \leq u \leq l_a$;
$K_{a,R\cup \{a\}} = t-1
\equiv
K_{a,R\cup \{a\}} = u$
for $0 \leq u < t$.
If $\GRNedge{a}{-}{t}{a}$,
$K_{a,R\cup \{a\}} = t
\equiv
K_{a,R\cup \{a\}} = u$
for $t \leq u \leq l_a$;
$K_{a,R\setminus \{a\}} = t-1
\equiv
K_{a,R\cup \{a\}} = u$
for $0 \setminus u < t$.
- any other equivalence?}


\subsection{Parameters inference}\label{ssec:infer-K}

This subsection presents some results related to the inference of independent discrete parameters from a given PH.
These results are equivalent to those presented in \cite{PMR10-TCSB}, with notation adapted to be shared with the previous section.
In addition, we introduce the well-formed PH for parameter inference property (\pref{pro:wf-ph-K}),
which implies that %the inferred IG does not contain any unsigned interactions, and thus can be seen as the
%regular IG $(\Gamma, E)$,
%and that 
any process in $\levels{b}{a}$ (resp. $\ulevels{b}{a}$) share the same behavior
regarding $a$.

\todo{makes it standalone: repeat the constraints for IG inference that are
actually necessary + check that we can do if if the IG is provided (then what
are the constraints on this IG)}

\begin{property}[Well-formed PH for parameter inference]\label{pro:wf-ph-K}
A PH is well-formed for parameter inference if and only if
it is well-formed for IG inference, and
the IG $(\Gamma, E)$ inferred by \pref{pps:inference-IG}
verifies the following property:
\todo{Define $c(\sigma)$ for $\sigma \in \textstyle\prod_{d \in \PHpredec{c}} \PHl_d$
(\ie the unique focal point)}
\modMF{
\begin{align*}
  \begin{split}
%   \forall a \in \Gamma&, \forall b\in \GRNreg{a},
%           \forall (i,j\in\levels{b}{a} \vee i,j\in\ulevels{b}{a}), \\
%   & \quad \forall c \in \PHs, ( (b\neq a\wedge c=a) \vee (c\in\PHpredec{a}\setminus \Gamma \wedge b\in\PHdirectpredec{c})), \\
%   & \qquad
%                           \PHfrappe{b_i}{c_k}{c_l}\in\PHa \Leftrightarrow
%                                   \PHfrappe{b_j}{c_k}{c_l}\in\PHa
      &
    \forall a \in \Gamma,
    \forall g \in X(a),
    \forall \sigma, \sigma' \in \textstyle\prod_{d \in g} \PHl_d,
    \forall b \in g,
    b \in \GRNreg{c},
    \forall c \in \PHdirectpredec{a} \setminus \Gamma,
      \\&\qquad
    b \in \reg{c},
    \forall (i,j \in \levels{b}{a} \vee i,j \in \ulevels{b}{a}),
%    \forall b_i \in \PHl_b,
    b_i \in \sigma,
%    b_i \in \levels{b}{a},
%    \forall b_j \in \levels{b}{a},
      \\&\qquad\qquad
    \sigma' = \sigma\{b_j\},
    %c_{\sigma} \in \PHl_c \wedge c_{\sigma'} \in \PHl_c,  %facultatif
    \PHfrappe{c(\sigma)}{a_i}{a_j} \in \PHa \Leftrightarrow
    \PHfrappe{c(\sigma')}{a_i}{a_j} \in \PHa
  \end{split}\\
  \begin{split}
    & \forall a \in \Gamma, \forall b \in \GRNreg{a},
      \forall (i,j \in \levels{b}{a} \vee i,j \in \ulevels{b}{a}), \\
    & \qquad \forall c \in \PHs, b \neq a,
      \PHfrappe{b_i}{a_k}{a_l} \in \PHa \Leftrightarrow
      \PHfrappe{b_j}{a_k}{a_l} \in \PHa
  \end{split}
\end{align*}
}
\end{property}

Let $K_{a,\omega}$ be the parameter we want to infer for a given component $a \in \Gamma$
and $\omega \subset \GRNreg{a}$ a set of its regulators.
This inference, as for the IG inference, relies on the search of focal processes of the component for the given configuration of its regulators.

For each sort $b \in \GRNreg{a}$, we define a context $C^b_{a,\omega}$ in \pref{eq:param_context} that contains all processes representing the influence of the resources in the configuration modeled by $\omega$.
The context of a cooperative sort $\upsilon$ that regulates $a$ is given in
\pref{eq:param_context_coop} as the set of focal processes matching the current configuration.
$C_{a,\omega}$ refers to the union of all these contexts (\pref{eq:K-ctx}).
\todo{(LP) update definition to match new $\focals$ (pick only one process for each sort)}
\begin{align}
  \label{eq:param_context}
  \forall b\in\Gamma,~
  C_{a,\omega}^b & \DEF \begin{cases}
    \levels{b}{a} & \text{if $b \in \omega$,}\\
    \ulevels{b}{a} & \text{if $b \notin \omega$,}\\
    L_b            & \text{otherwise;}\\
  \end{cases}
  \\
  \label{eq:param_context_coop}
  \forall \upsilon \in \PHpredec{a}\setminus\Gamma,~
  C_{a,\omega}^\upsilon & \DEF \{
  \upsilon(\sigma) \mid \sigma \in \textstyle\prod_{c\in\PHdirectpredec{\upsilon}}C_{a,\omega}^c \}
  \\
  C_{a,\omega} & \DEF \textstyle\bigcup_{b\in\PHpredec{a}} C^b_{a,\omega}
  \label{eq:K-ctx}
\end{align}

The parameter $K_{a,\omega}$ specifies to which values $a$ eventually evolves as long as the context
$C_{a,\omega}$ holds, which is precisely the definition of the $\focals$ function
(\pref{def:focals} in \pref{ssec:focal}),
where the focals reachability property can be derived from \pref{pro:wf-ph-K} and
\pref{eq:param_context_coop}.
Hence $K_{a,\omega} = \focals(a,C^a_{a,\omega},C_{a,\omega})$ if this latter is a non-empty interval
(\pref{pps:param_K}).

\begin{proposition}[Parameter inference]
\label{pps:param_K}
Let $(\PHs, \PHl, \PHh)$ be a Process Hitting well-formed for parameter inference, and $\IG = (\Gamma, E)$ the inferred IG.
Let $A$ (resp. $B$) $\subseteq \Gamma$ be the set of regulators that activate (resp. inhibit) a sort
$a$.
If $\focals(a,C^a_{a,\omega},C_{a,\omega})=\segm{a_i}{a_j}$ is a non-empty interval, then $K_{a,\omega} = \segm{i}{j}$.
\end{proposition}

\begin{example}
\label{ex:infer-param-runningPH-1}

Applied to the PH in \pref{fig:runningPH-1}, we infer the following parameters:
\begin{align*}
K_{a, \emptyset} &= \segm{0}{0} &
K_{b, \emptyset} &= \segm{0}{0} \\
K_{a, \{a\}} &= \segm{0}{0} &
K_{b, \{a\}} &= \segm{0}{0} \\
K_{a, \{c\}} &= \segm{1}{1} &
K_{b, \{b\}} &= \segm{1}{1} \\
K_{a, \{b\}} &= \segm{1}{1} &
K_{b, \{a,b\}} &= \segm{0}{0} \\
K_{a, \{b,c\}} &= \segm{1}{1} &
K_{c, \emptyset} &= \segm{0}{0} \\
K_{a, \{a,b,c\}} &= \segm{2}{2} &
K_{c, \{c\}} &= \segm{1}{1}
%K_{a, \{a,b\}} = ?
%K_{a, \{a,c\}} = ?
\end{align*}
$K_{a,\{a,b\}}$ and $K_{a,\{a,c\}}$ cannot be inferred,
which is a direct consequence of the lack of cooperation between $b$ and $c$ on $a$.
\end{example}

\begin{example}
Regarding the refined PH of \pref{fig:runningPH-2}, all parameters can be inferred.
We obtain the same value for the inferred parameters of \pref{ex:infer-param-runningPH-1},
together with the following results:
\begin{align*}
  K_{a,\{a,b\}} &= \segm{1}{1} &
  K_{a,\{a,c\}} &= \segm{1}{1}
\end{align*}
\end{example}

Given the \pref{pps:param_K}, we see that in some cases, the inference of the targeted parameter is impossible.
This can be due to a lack of cooperation between regulators:
when two regulators independently hit a component, their actions can have opposite effects, leading to two possible evolutions.
Such an indeterminism is not possible in Thomas modeling as in a given configuration of regulators,
a component can only have an interval attractor, and can therefore evolve in only one direction.
In order to avoid such inconclusive cases, one has to ensure that no such behavior is allowed by
either removing undesired actions or using cooperative sorts to prevent opposite influences between
concurrent regulators.

\subsection{Admissible parametrizations}\label{ssec:admissible-K}

When building a BRN, one has to find the parametrization that best describes the desired behavior of the studied system.
Complexity is inherent to this process as the number of possible parametrizations for a given IG is exponential w.r.t.~the number of components.
However, the method of parameters inference presented in this section gives some information about necessary parameters given a certain dynamics described by a PH.
This information thus drops the number of possible parametrizations, allowing to find the desired behavior more easily.

We first delimit the validity of a parameter (\pref{pro:K-valid}) in order to ensure that any
transition in the resulting BRN is allowed by the studied PH.
This is verified by the existence of a hit making the concerned component bounce into the direction
of the value of the parameter in the matching context.
Thus, assuming \pref{pro:wf-ph-K} holds, any transition in the inferred BRN corresponds to at least
one transition in the PH, proving the correctness of our inference.
We remark that any parameter inferred by \pref{pps:param_K} satisfies this property.

\begin{property}[Parameter validity]\label{pro:K-valid}
A parameter $K_{a,\omega}$ is valid w.r.t. the PH if and only if the following equation is verified:
\begin{align*}
  \forall a_i\in C^a_{a,\omega}, a_i \notin K_{a,\omega} \Longrightarrow
    (& \exists \PHfrappe{c_k}{a_i}{a_j}\in\PHa, c_k \in C^c_{a,\omega} \\
     & \wedge a_i < K_{a,\omega} \Rightarrow j > i \wedge  a_i > K_{a,\omega} \Rightarrow j <i )
\end{align*}
\end{property}

Then, we use some additional biological constraints on Thomas' parameters given in
\cite{BernotSemBRN}, that we sum up in the following three properties:

\begin{property}[Extreme values assumption]
Let $\IG = (\Gamma, E)$ be an IG. A parametrization $K$ on $\IG$ satisfies the \emph{extreme values assumption} if and only if:
\label{pro:param_enum_extreme}
\begin{align*}
  \forall b \in \Gamma, \GRNreg{b} \neq \emptyset \Longrightarrow \exists \omega \subset \GRNreg{b}, 0 \in K_{b,\omega} \wedge \exists \omega' \subset \GRNreg{b}, l_b \in K_{b,\omega'}
\end{align*}
\end{property}

\begin{property}[Activity assumption]
\label{pro:param_enum_activity}
Let $\IG = (\Gamma, E)$ be an IG. A parametrization $K$ on $\IG$ satisfies the \emph{activity assumption} if and only if:
\begin{align*}
  \forall b \in \Gamma, \forall a \in \GRNreg{b}, \exists \omega \subset \GRNreg{b}, K_{b,\omega} \neq K_{b,\omega \cup \{ a \}}
\end{align*}
\end{property}

\begin{property}[Monotonicity assumption]
\label{pro:param_enum_monotonicity}
Let $\IG = (\Gamma, E)$ be an IG. A parametrization $K$ on $\IG$ satisfies the \emph{monotonicity assumption} if and only if:
\begin{align*}
  \forall b \in \Gamma,
  \forall A^+ \subset \{ a \in \Gamma \mid \GRNedge{a}{+}{t}{b} \in E_+ \}&,
  \forall A^- \subset \{ a \in \Gamma \mid \GRNedge{a}{-}{t}{b} \in E_- \},\\
  K_{b,\omega \cup A^-} & \leqsegm K_{b,\omega \cup A^+}
\end{align*}
\end{property}

\begin{example}\label{ex:enum-param-runningPH-1}
The parametrization inferred in \pref{ex:infer-param-runningPH-1} was partial because $K_{a,\{a,b\}}$ and $K_{a,\{a,c\}}$ could not be inferred.
It is however possible to enumerate all complete and admissible parametrizations
compatible with both the inferred parameters, and the properties of this subsection.
This enumeration gives 9 different parametrizations which correspond to the 3 possible values
for each of the two parameters that could not be inferred:
\begin{align*}
  K_{a,\{a,b\}} &\in \{ \segm{1}{1}, \segm{1}{2}, \segm{2}{2} \} \\
  K_{a,\{a,c\}} &\in \{ \segm{1}{1}, \segm{1}{2}, \segm{2}{2} \}
\end{align*}
We note that for all solutions, $0 \notin K_{a,\{a,b\}} \wedge 0 \notin K_{a,\{a,c\}}$.
This is due to the monotonicity assumption (\pref{pro:param_enum_monotonicity}) which especially states that:
\begin{align*}
  K_{a,\{b\}} \leqsegm K_{a,\{a,b\}} \wedge
  K_{a,\{c\}} \leqsegm K_{a,\{a,c\}}
\end{align*}

Finally, we note that $\segm{1}{1}$ belongs to the possible values for both parameters.
Therefore this enumeration allows, from the model in \pref{fig:runningPH-1},
to find the behavior of the model refined with a cooperative sort described in \pref{fig:runningPH-2}.
\end{example}
