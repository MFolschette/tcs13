% vi:spell spelllang=en:
\section{Parametrization inference}\label{sec:infer-K}

\modLP{
Given a PH and an IG (either arbitrary, or inferred following the previous section), this section
addresses the identification of the parametrization for the corresponding Thomas model.}

\modLP{
As described in \pref{ssec:infer-K}, this identification relies on the computation of the focal
processes for each configuration of the parameters.
When there is a unique focal process of a component for a given configuration of its regulators,
the focal process matches with the value of the associated Thomas parameter.}

\modLP{
However, in the general case, as described in \pref{sec:tr2global} and even when the PH is
well-formed for parameter inference (characterized in the next subsection by \pref{pro:wf-ph-K}),
there may be several focal processes (nondeterministic behaviour) or none (cyclic behaviour).  Such
a case occurs notably when the PH encodes the union of several Boolean or discrete networks, as
described in \cite{PMR10-TCSB,PCFMR14-chapterLMBS}.}

\modLP{
Hence, we propose in \pref{ssec:admissible-K} the notion of \emph{compatible} parametrization with respect
ot the PH dynamics: a Thomas model is compatible with a PH if its dynamics is included in the PH
dynamics, \ie all the transitions in the Thomas model are possible transitions in the PH.
This relaxed notion of parameterization compatibility allows to enumerate all parametrizations
compatible with a given PH, \ie all Thomas model whose dynamics is included in the PH dynamics.}


\subsection{Parameters inference}\label{ssec:infer-K}

This subsection addresses the inference of independent discrete parameters from a given PH.
The inference is equivalent to the one in \cite{PMR10-TCSB}.
In addition, we characterize the well-formed PH for parameter inference property (\pref{pro:wf-ph-K}),
which implies that %the inferred IG does not contain any unsigned interactions, and thus can be seen as the
%regular IG $(\Gamma, E)$,
%and that 
any process in $\levels{b}{a}$ (resp. $\ulevels{b}{a}$) share the same behavior regarding $a$.

\begin{property}[Well-formed PH for parameter inference]\label{pro:wf-ph-K}
A PH is well-formed for parameter inference if and only if
it is well-formed for BRN identification (\pref{pro:wf-ph}) and
that the associated IG $(\Gamma, E)$ verifies that:
\modLP{
\begin{align}
\begin{split}
\forall b\rightarrow a&\in E, b\neq a,
	\forall \sigma\in L(\PHpredecgene{a}),\\
    \forall (i,j & \in \levels{b}{a} \vee i,j \in \ulevels{b}{a}),
	\forall a_k \in L_a,
	\\
&	\focals(a,\{a_k\},\ctx_a^{\PHpredecgene{a}}(\sigma\{b_i\}))
	=\focals(a,\{a_k\},\ctx_a^{\PHpredecgene{a}}(\sigma\{b_j\}))
\end{split}
\\
\begin{split}
\forall a\in\Gamma, &\forall b\in\PHpredecgene{a},
b\neq a \wedge b\rightarrow a\notin E,\\
	\forall \sigma&\in L(\PHpredecgene{a}), 
    \forall b_i,b_j \in L_b,
	\forall a_k \in L_a,
	\\
&	\focals(a,\{a_k\},\ctx_a^{\PHpredecgene{a}}(\sigma\{b_i\}))
	=\focals(a,\{a_k\},\ctx_a^{\PHpredecgene{a}}(\sigma\{b_j\}))
\end{split}
\end{align}
}
\end{property}

Let $K_{a,\omega}$ be a Thomas parameter for a given component $a \in \Gamma$ 
with $\omega \subset \GRNreg{a}$ a set of its regulators.
As described in \pref{ssec:thomas}, $K_{a,\omega}$ specifies to which values $a$ eventually evolves
in the configuration matching with $\omega$.

\modLP{The configuration of the PH corresponding to $\omega$ is given as follows.
For each component $b \in \GRNreg{a}$, we define $\sigma^b_{a,\omega}$ (\pref{eq:param_context}) as 
the process of $b$ with the level in 
$\levels{b}{a}$ if $b\in\omega$, or in $\ulevels{b}{a}$ if $b\notin\omega$.
Because of \pref{pro:wf-ph-K}, if several possible levels exist, this process can be chosen
arbitrarily.
If $b$ does not regulate $a$ in the IG, all processes of sort $b$ should have the same action on
$a$, so the process $b_0$ is arbitrarily selected (\pref{eq:param_context_free}).
The configuration $\sigma_{a,\omega}$ corresponding to $\omega$ (\pref{eq:K-ctx})
is then obtained by extending the configuration of the
regulators of $a$ to the (deterministic) cooperative sorts (\pref{def:ctx-sigma} in
\pref{ssec:split-sorts}).
Finally, we denote by $S_a(\sigma)$ (\pref{eq:K-domain}) the set of processes of sort $a$ that are
compatible with a configuration $\sigma$: if $a$ is specified in $\sigma$, then $S_a(\sigma) = \{a_i\}$
where $a_i\in\sigma$; otherwise $S_a(\sigma)$ \modMF{is the set of all processes of sort $a$, that is,} $L_a$.}
\begin{align}
  \label{eq:param_context}
  \forall b\rightarrow a\in E,~
  \sigma_{a,\omega}^b & \DEF \begin{cases}
    \modLP{\min}(\levels{b}{a}) & \text{if $b \in \omega$,}\\
    \modLP{\min}(\ulevels{b}{a}) & \text{if $b \notin \omega$}
  \end{cases}
  \\
  \modLP{
  \forall b\in\PHpredecgene a, b\rightarrow a\notin E,~
  \sigma_{a,\omega}^b} & \modLP{\DEF b_0}
  \label{eq:param_context_free}
  \\
  \sigma_{a,\omega} & \DEF
  \modLP{\ctx_a^{\PHpredecgene{a}}(
  	\state{\sigma^b_{a,\omega} \mid  b\in\PHpredecgene{a}})}
  \label{eq:K-ctx}
  \\
  \modLP{
  \forall a\in\Gamma,~
  S_a(\sigma)} &\modLP{\DEF
	\begin{cases}
	\{a_i\} & \text{if }a_i\in \sigma\\
	L_a & \text{otherwise.}
	\end{cases}}
\label{eq:K-domain}
\end{align}

Therefore, we obtain that
\modLP{$K_{a,\omega} = \focals(a,S_a(\sigma_{a,\omega}),\sigma_{a,\omega})$ if this latter is
a singleton, denoting a deterministic behaviour (\pref{pps:param_K})}.

\begin{proposition}[Parameter inference]
\label{pps:param_K}
Let $(\PHs, \PHl, \PHh)$ be a Process Hitting \modLP{with $\PHs=\Gamma\cup\CSorts$} and an
associated IG $\IG = (\Gamma, E)$ well-formed for parameter inference.
For all $a\in\Gamma$, for any $\omega\subseteq \GRNreg{a}$,
if $\focals(a,S_a(\sigma_{a,\omega}),\sigma_{a,\omega})= \{ a_i \}$, then \modLP{$K_{a,\omega} = i$}.
\end{proposition}

\modMF{
\begin{example}
\label{ex:infer-param-runningPH-2}
When applied to the refined PH of \pref{fig:runningPH-2},
and using the IG of \pref{fig:runningBRN}(left)
which is the result of the IG inference on this PH model
(as explained in \pref{ssec:infer-ig-examples}),
\pref{pps:param_K} infers the parametrization of \pref{fig:runningBRN}(right).
The inference of parameters thus finds back all parameters
of the original model when the cooperations are fully defined.
\end{example}
}

\modMF{
\begin{example}
\label{ex:infer-param-runningPH-1}
Regarding the non-refined PH of \pref{fig:runningPH-1},
and using the same IG of \pref{fig:runningBRN}(left),
the parameters of \pref{fig:runningBRN}(right) are inferred,
except the parameters $K_{a,\{b\}}$ and $K_{a,\{c\}}$
for which \pref{pps:param_K} is not conclusive.
The obtained parametrization is therefore partial.
This is due to the lack of precision in the cooperation between $b$ and $c$,
caused by the absence of the cooperative sort.
\end{example}
}

Given the \pref{pps:param_K}, we see that in some cases, the inference of the targeted parameter is impossible.
This can be due to a lack of cooperation between regulators:
when two regulators independently hit a component, their actions can have opposite effects, leading to two possible evolutions.
Such an indeterminism is not possible in Thomas modeling as in a given configuration of regulators,
a component can only have an interval attractor, and can therefore evolve in only one direction.
In order to avoid such inconclusive cases, one has to ensure that no such behavior is allowed by
either removing undesired actions or using cooperative sorts to prevent opposite influences between
concurrent regulators.

\subsection{Admissible parametrizations}\label{ssec:admissible-K}

When building a BRN, one has to find the parametrization that best describes the desired behavior of the studied system.
Complexity is inherent to this process as the number of possible parametrizations for a given IG is exponential w.r.t.~the number of components.
However, the method of parameters inference presented in this section gives some information about necessary parameters given a certain dynamics described by a PH.
This information thus drops the number of possible parametrizations, allowing to find the desired behavior more easily.

We first delimit the validity of a parameter (\pref{pro:K-valid}) in order to ensure that any
transition in the resulting BRN is allowed by the studied PH.
This is verified by the existence of a hit making the concerned component bounce into the direction
of the value of the parameter in the matching context.
Thus, assuming \pref{pro:wf-ph-K} holds, any transition in the inferred BRN corresponds to at least
one transition in the PH, proving the correctness of our inference.
\modMF{%
Any parameter that does not satisfy \pref{pro:K-valid} is therefore excluded from the enumeration.
}%
We remark that all parameters inferred by \pref{pps:param_K} satisfy this property.

\begin{property}[Parameter validity]\label{pro:K-valid}
A parameter $K_{a,\omega}$ is valid w.r.t.~the PH if and only if the following equation is verified:
\begin{align*}
  \forall a_i \in C^a_{a,\omega}, a_i \neq K_{a,\omega} \Longrightarrow
    (& \exists \PHfrappe{c_k}{a_i}{a_j}\in\PHa, c_k \in C^c_{a,\omega} \\
     & \wedge a_i < K_{a,\omega} \Rightarrow j > i \wedge  a_i > K_{a,\omega} \Rightarrow j <i )
\end{align*}
\end{property}

Then, we use some additional biological constraints on Thomas' parameters given in
\cite{BernotSemBRN}, that we sum up in the following three properties:

\begin{property}[Extreme values assumption]
\label{pro:param_enum_extreme}
Let $\IG = (\Gamma, E)$ be an IG. A parametrization $K$ on $\IG$ satisfies the \emph{extreme values assumption} if and only if:
\begin{align*}
  \forall b \in \Gamma, \GRNreg{b} \neq \emptyset \Longrightarrow
    \exists \omega \subset \GRNreg{b}, K_{b,\omega} = 0 \wedge
    \exists \omega' \subset \GRNreg{b}, K_{b,\omega'} = l_b
\end{align*}
\end{property}

\begin{property}[Activity assumption]
\label{pro:param_enum_activity}
Let $\IG = (\Gamma, E)$ be an IG. A parametrization $K$ on $\IG$ satisfies the \emph{activity assumption} if and only if:
\begin{align*}
  \forall b \in \Gamma, \forall a \in \GRNreg{b}, \exists \omega \subset \GRNreg{b}, K_{b,\omega} \neq K_{b,\omega \cup \{ a \}}
\end{align*}
\end{property}

\begin{property}[Monotonicity assumption]
\label{pro:param_enum_monotonicity}
Let $\IG = (\Gamma, E)$ be an IG. A parametrization $K$ on $\IG$ satisfies the \emph{monotonicity assumption} if and only if:
\begin{align*}
  \forall b \in \Gamma,
  \forall A^+ \subset \{ a \in \Gamma \mid \GRNedge{a}{+}{t}{b} \in E_+ \}&,
  \forall A^- \subset \{ a \in \Gamma \mid \GRNedge{a}{-}{t}{b} \in E_- \},\\
  K_{b,\omega \cup A^-} & \leq K_{b,\omega \cup A^+}
\end{align*}
\end{property}

\begin{example}\label{ex:enum-param-runningPH-1}
The parametrization inferred in \pref{ex:infer-param-runningPH-1} was partial
because \modMF{$K_{a,\{b\}}$ and $K_{a,\{c\}}$} could not be inferred.
It is however possible to enumerate all admissible parametrizations
compatible with both the inferred parameters, and the properties of this subsection.
This enumeration gives 9 different parametrizations
which correspond to the 3 possible values
\modMF{%
for both $K_{a,\{b\}}$ and $K_{a,\{c\}}$:
%\[(K_{a,\{b\}}; K_{a,\{c\}}) \in \{ 0, 1, 2 \} \times \{ 0, 1, 2 \}\]
\[K_{a,\{b\}} \in \{ 0, 1, 2 \} \qquad \text{and} \qquad K_{a,\{c\}} \in \{ 0, 1, 2 \}\]
We note that all these solutions satisfy Properties
\ref{pro:K-valid} to \ref{pro:param_enum_monotonicity}.
All the parametrizations obtained from these combinations are thus admissible.
}

Finally, we note that the value \modMF{$1$} belongs to the possible values for both parameters.
Therefore this enumeration allows, from the model in \pref{fig:runningPH-1},
to find the behavior of the model refined with a cooperative sort described in
\pref{fig:runningPH-2}.
\end{example}
