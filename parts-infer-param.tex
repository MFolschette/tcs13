\section{Parametrization inference}\label{sec:infer-K}

\modLP{
Given a PH and an IG (either arbitrary, or inferred following the previous section), this section
addresses the identification of the parametrization for the corresponding Thomas model.}

\modLP{
As described in \pref{ssec:infer-K}, this identifcation relies on the computation of the focal
processes for each configuration of the parameters.
When there is a unique focal process of a component for a given configuration of its regulators,
the focal process matches with the value of the associated Thomas parameter.}

\modLP{
However, in the general case, as described in \pref{sec:tr2global} and even when the PH is
well-formed for parameter inference (characterized in the next subsection by \pref{pro:wf-ph-K}),
there maybe several focal processes (non-deterministic behaviour) or none (cyclic behaviour).  Such
a case occurs notably when the PH encodes the union of several Boolean or discrete networks, as
described in \cite{PMR10-TCSB,PCFMR14-chapterLMBS}.}

\modLP{
Hence, we propose in \pref{ssec:admissible-K} the notion of \emph{compatible} parametrization with respect
ot the PH dynamics: a Thomas model is compatible with a PH if Thomas dynamics is included in the PH
dynamics, i.e., all the transitions in the Thomas model are possible transitions in the PH.
This relaxed notion of parameterization compatibility allows to enumerate all parametrizations
compatible with a given PH, i.e. all Thomas model whose dynamics is included in the PH dynamics.}


\subsection{Parameters inference}\label{ssec:infer-K}

This subsection addresses the inference of independent discrete parameters from a given PH.
The inference is equivalent to the one in \cite{PMR10-TCSB}.
In addition, we characterize the well-formed PH for parameter inference property (\pref{pro:wf-ph-K}),
which implies that %the inferred IG does not contain any unsigned interactions, and thus can be seen as the
%regular IG $(\Gamma, E)$,
%and that 
any process in $\levels{b}{a}$ (resp. $\ulevels{b}{a}$) share the same behavior regarding $a$.

\begin{property}[Well-formed PH for parameter inference]\label{pro:wf-ph-K}
A PH is well-formed for parameter inference if and only if
it is well-formed for BRN identification (\pref{pro:wf-ph}) and
that the associated IG $(\Gamma, E)$ verifies that:
\modLP{
\begin{align}
\begin{split}
\forall b\rightarrow a&\in E, b\neq a,
	\forall \sigma\in L(\PHpredecgene{a}),\\
    \forall (i,j & \in \levels{b}{a} \vee i,j \in \ulevels{b}{a}),
	\forall a_k \in L_a,
	\\
&	\focals(a,\{a_k\},\ctx_a^{\PHpredecgene{a}}(\sigma\{b_i\})))
	=\focals(a,\{a_k\},\ctx_a^{\PHpredecgene{a}}(\sigma\{b_j\})))
\end{split}
\\
\begin{split}
\forall a\in\Gamma, &\forall b\in\PHpredecgene{a},
b\neq a \wedge b\rightarrow a\notin E,\\
	\forall \sigma&\in L(\PHpredecgene{a}), 
    \forall b_i,b_j \in L_b,
	\forall a_k \in L_a,
	\\
&	\focals(a,\{a_k\},\ctx_a^{\PHpredecgene{a}}(\sigma\{b_i\})))
	=\focals(a,\{a_k\},\ctx_a^{\PHpredecgene{a}}(\sigma\{b_j\})))
\end{split}
\end{align}
}
\end{property}

Let $K_{a,\omega}$ be a Thomas parameter for a given component $a \in \Gamma$ 
with $\omega \subset \GRNreg{a}$ a set of its regulators.
As described in \pref{ssec:thomas}, $K_{a,\omega}$ specifies to which values $a$ eventually evolves
in the configuration matching with $\omega$.

\modLP{The configuration of the PH corresponding to $\omega$ is given as follows.
For each component $b \in \GRNreg{a}$, we define $\sigma^b_{a,\omega}$ (\pref{eq:param_context}) as 
the process of $b$ with the level in 
$\levels{b}{a}$ if $b\in\omega$, or in $\ulevels{b}{a}$ if $b\not\omega$.
Because of \pref{pro:wf-ph-K}, if several possible levels exist, this process can be chosen
arbitrarily.
If $b$ does not regulate $a$ in the IG, all processes of sort $b$ should have the same action on
$a$, so the process $b_0$ is arbitrarily selected (\pref{eq:param_context_free}).
The configuration $\sigma_{a,\omega}$ corresponding to $\omega$ (\pref{eq:K-ctx})
is then obtained by extending the configuration of the
regulators of $a$ to the (deterministic) cooperative sorts (\pref{def:ctx-sigma} in
\pref{ssec:split-sorts}).
Finally, we denote by $S_a(\sigma)$ (\pref{eq:K-domain}) the set of processes of sort $a$ that are
compatible with a configuration $\sigma$: if $a$ is specified in $\ctx$, then $S_a(\sigma) = \{a_i\}$
where $a_i\in\sigma$; otherwise $S_a(\sigma)$ is the all the processes of sort $a$, $L_a$.}
\begin{align}
  \label{eq:param_context}
  \forall b\rightarrow a\in E,~
  \sigma_{a,\omega}^b & \DEF \begin{cases}
    \modLP{\min}(\levels{b}{a}) & \text{if $b \in \omega$,}\\
    \modLP{\min}(\ulevels{b}{a}) & \text{if $b \notin \omega$}
  \end{cases}
  \\
  \modLP{
  \forall b\in\PHpredecgene a, b\rightarrow a\notin E,~
  \sigma_{a,\omega}^b} & \modLP{\DEF b_0}
  \label{eq:param_context_free}
  \\
  \sigma_{a,\omega} & \DEF
  \modLP{\ctx_a^{\PHpredecgene{a}}(
  	\state{\sigma^b_{a,\omega} \mid  b\in\PHpredecgene{a}})}
  \label{eq:K-ctx}
  \\
  \modLP{
  \forall a\in\Gamma,~
  S_a(\sigma)} &\modLP{\DEF
	\begin{cases}
	\{a_i\} & \text{if }a_i\in \ctx\\
	L_a & \text{otherwise.}
	\end{cases}}
\label{eq:K-domain}
\end{align}

Therefore, we obtain that
\modLP{$K_{a,\omega} = \focals(a,S_a(\sigma_{a,\omega}),\sigma_{a,\omega})$ if this latter is
a singleton, denoting a deterministic behaviour (\pref{pps:param_K})}.

\begin{proposition}[Parameter inference]
\label{pps:param_K}
Let $(\PHs, \PHl, \PHh)$ be a Process Hitting \modLP{with $\PHs=\Gamma\cup\CSorts$} 
well-formed for parameter inference, and $\IG = (\Gamma, E)$ the inferred IG.
Let $A$ (resp. $B$) $\subseteq \Gamma$ be the set of regulators that activate (resp. inhibit) a sort
$a$.
If $\focals(a,S_a(\sigma_{a,\omega}),\sigma_{a,\omega})= \{ a_i \}$, then \modLP{$K_{a,\omega} = i$}.
\end{proposition}

\begin{example}
\label{ex:infer-param-runningPH-1}

Applied to the PH in \pref{fig:runningPH-1}, we infer the following parameters:
\begin{align*}
K_{a, \emptyset} &= 0&
K_{b, \emptyset} &= 0 \\
K_{a, \{a\}} &= 0 &
K_{b, \{a\}} &= 0 \\
K_{a, \{c\}} &= 1 &
K_{b, \{b\}} &= 1 \\
K_{a, \{b\}} &= 1 &
K_{b, \{a,b\}} &= 0 \\
K_{a, \{b,c\}} &= 1 &
K_{c, \emptyset} &= 0 \\
K_{a, \{a,b,c\}} &= 2 &
K_{c, \{c\}} &= 1
%K_{a, \{a,b\}} = ?
%K_{a, \{a,c\}} = ?
\end{align*}
$K_{a,\{a,b\}}$ and $K_{a,\{a,c\}}$ cannot be inferred,
which is a direct consequence of the lack of cooperation between $b$ and $c$ on $a$.
\end{example}

\begin{example}
Regarding the refined PH of \pref{fig:runningPH-2}, all parameters can be inferred.
We obtain the same value for the inferred parameters of \pref{ex:infer-param-runningPH-1},
together with the following results:
\begin{align*}
  K_{a,\{a,b\}} &= 1 &
  K_{a,\{a,c\}} &= 1
\end{align*}
\end{example}

Given the \pref{pps:param_K}, we see that in some cases, the inference of the targeted parameter is impossible.
This can be due to a lack of cooperation between regulators:
when two regulators independently hit a component, their actions can have opposite effects, leading to two possible evolutions.
Such an indeterminism is not possible in Thomas modeling as in a given configuration of regulators,
a component can only have an interval attractor, and can therefore evolve in only one direction.
In order to avoid such inconclusive cases, one has to ensure that no such behavior is allowed by
either removing undesired actions or using cooperative sorts to prevent opposite influences between
concurrent regulators.

\subsection{Admissible parametrizations}\label{ssec:admissible-K}

When building a BRN, one has to find the parametrization that best describes the desired behavior of the studied system.
Complexity is inherent to this process as the number of possible parametrizations for a given IG is exponential w.r.t.~the number of components.
However, the method of parameters inference presented in this section gives some information about necessary parameters given a certain dynamics described by a PH.
This information thus drops the number of possible parametrizations, allowing to find the desired behavior more easily.

We first delimit the validity of a parameter (\pref{pro:K-valid}) in order to ensure that any
transition in the resulting BRN is allowed by the studied PH.
This is verified by the existence of a hit making the concerned component bounce into the direction
of the value of the parameter in the matching context.
Thus, assuming \pref{pro:wf-ph-K} holds, any transition in the inferred BRN corresponds to at least
one transition in the PH, proving the correctness of our inference.
We remark that any parameter inferred by \pref{pps:param_K} satisfies this property.

\todo{update notations}
\begin{property}[Parameter validity]\label{pro:K-valid}
A parameter $K_{a,\omega}$ is valid w.r.t. the PH if and only if the following equation is verified:
\begin{align*}
  \forall a_i\in C^a_{a,\omega}, a_i \notin K_{a,\omega} \Longrightarrow
    (& \exists \PHfrappe{c_k}{a_i}{a_j}\in\PHa, c_k \in C^c_{a,\omega} \\
     & \wedge a_i < K_{a,\omega} \Rightarrow j > i \wedge  a_i > K_{a,\omega} \Rightarrow j <i )
\end{align*}
\end{property}

Then, we use some additional biological constraints on Thomas' parameters given in
\cite{BernotSemBRN}, that we sum up in the following three properties:

\begin{property}[Extreme values assumption]
Let $\IG = (\Gamma, E)$ be an IG. A parametrization $K$ on $\IG$ satisfies the \emph{extreme values assumption} if and only if:
\label{pro:param_enum_extreme}
\begin{align*}
  \forall b \in \Gamma, \GRNreg{b} \neq \emptyset \Longrightarrow \exists \omega \subset \GRNreg{b}, 0 \in K_{b,\omega} \wedge \exists \omega' \subset \GRNreg{b}, l_b \in K_{b,\omega'}
\end{align*}
\end{property}

\begin{property}[Activity assumption]
\label{pro:param_enum_activity}
Let $\IG = (\Gamma, E)$ be an IG. A parametrization $K$ on $\IG$ satisfies the \emph{activity assumption} if and only if:
\begin{align*}
  \forall b \in \Gamma, \forall a \in \GRNreg{b}, \exists \omega \subset \GRNreg{b}, K_{b,\omega} \neq K_{b,\omega \cup \{ a \}}
\end{align*}
\end{property}

\begin{property}[Monotonicity assumption]
\label{pro:param_enum_monotonicity}
Let $\IG = (\Gamma, E)$ be an IG. A parametrization $K$ on $\IG$ satisfies the \emph{monotonicity assumption} if and only if:
\begin{align*}
  \forall b \in \Gamma,
  \forall A^+ \subset \{ a \in \Gamma \mid \GRNedge{a}{+}{t}{b} \in E_+ \}&,
  \forall A^- \subset \{ a \in \Gamma \mid \GRNedge{a}{-}{t}{b} \in E_- \},\\
  K_{b,\omega \cup A^-} & \leqsegm K_{b,\omega \cup A^+}
\end{align*}
\end{property}

\begin{example}\label{ex:enum-param-runningPH-1}
The parametrization inferred in \pref{ex:infer-param-runningPH-1} was partial because $K_{a,\{a,b\}}$ and $K_{a,\{a,c\}}$ could not be inferred.
It is however possible to enumerate all complete and admissible parametrizations
compatible with both the inferred parameters, and the properties of this subsection.
This enumeration gives 9 different parametrizations which correspond to the 3 possible values
for each of the two parameters that could not be inferred:
\begin{align*}
  K_{a,\{a,b\}} &\in \{ \segm{1}{1}, \segm{1}{2}, \segm{2}{2} \} \\
  K_{a,\{a,c\}} &\in \{ \segm{1}{1}, \segm{1}{2}, \segm{2}{2} \}
\end{align*}
We note that for all solutions, $0 \notin K_{a,\{a,b\}} \wedge 0 \notin K_{a,\{a,c\}}$.
This is due to the monotonicity assumption (\pref{pro:param_enum_monotonicity}) which especially states that:
\begin{align*}
  K_{a,\{b\}} \leqsegm K_{a,\{a,b\}} \wedge
  K_{a,\{c\}} \leqsegm K_{a,\{a,c\}}
\end{align*}

Finally, we note that $\segm{1}{1}$ belongs to the possible values for both parameters.
Therefore this enumeration allows, from the model in \pref{fig:runningPH-1},
to find the behavior of the model refined with a cooperative sort described in \pref{fig:runningPH-2}.
\end{example}
