% vim:spell spelllang=en:
\section{Introduction}\label{sec:intro}

As regulatory phenomena play a crucial role in biological systems, they need to be studied accurately.
Biological Regulatory Networks (BRNs) consist in sets of either positive or negative mutual effects between the components.
With the purpose of analyzing these systems, they are often modeled as graphs which make it possible to determine the possible evolutions of all the interacting components of the system.
\modOR{Besides continuous models of physicists, generally designed through systems of ordinary differential equations, modeling regulatory networks by means of Boolean networks has become popular in the wake of Kauffman's work \cite{kauffman1969metabolic}.
We based our work on a logical formalism developed by Ren\'e Thomas from 1973 \cite{Thomas73} that generalizes upon Boolean networks in the sense that it allows variables to have more than two values and transitions between states to occur asynchronously.}

In this approach, the different levels of a component, such as concentration or expression levels, are abstractly represented by (positive) integer values and transitions between these levels may be considered as instantaneous.
Hence, qualitative state graphs may be derived from which we are able to formally find out all the possible behaviors expressed as sequences of transitions between these states.
Nevertheless, these dynamics can be precisely established only with regard to some discrete parameters,
hereafter called ``Thomas' parameters'',
which stand for kinds of ``focal points'', \ie the evolutionary tendency from each state and depending on the set of the other currently interacting components.
%resources in this very state, that is, the set of the other currently interacting components.
%Hereafter, we refer to these discrete parameters as Thomas' parameters.

Thomas' modeling has motivated numerous works around the link between the Interaction Graph (IG)
(summarizing the global influences between components) and the possible dynamics (\eg
\cite{RiCo07,RRT08}),
model reduction (\eg \cite{Naldi09}), formal checking of dynamics (\eg \cite{Richard06,Naldi07}), 
and the incorporation of time (\eg \cite{Siebert06,Ahmad08}) and probability
(\eg \cite{Twardziok10-CMSB}) dimensions, to name but a few.

While the formal checking of dynamical properties is often limited to small networks because of the
state graph explosion, an other major drawback of this framework is the difficulty to
specify Thomas' parameters, especially for large networks. \modOR{Other works have also been carried out with the aim of dealing with large systems (\eg \cite{Corblin200991,Bollman2007,Delaplace2012, 10.1371/journal.pone.0007992}).} \modMM{Because of the complexity of such systems, these works generally focus on some restricted properties,} \modOR{such as finding steady states.} 
%\todo{The reviewer proposes to nuance this with several papers:
%\begin{itemize}
%  \item Corblin, F., Tripodi, S., Fanchon, E., Ropers, D., \& Trilling, L. (2009).
%    A declarative constraint-based method for analyzing discrete genetic
%    regulatory networks. Bio Systems, 98(2), 91104.
%  \item Bollman, D., Coln-Reyes, O., \& Orozco, E. (2007). Fixed points in
%    discrete models for regulatory genetic networks. EURASIP journal on
%    bioinformatics \& systems biology, 2007, 97356.
%  \item Delaplace, F., Klaudel, H., Melliti, T., \& Sen, S. (2012). Analysis
%    of modular organisation of interaction networks based on asymptotic
%    dynamics. In D. Gilbert \& M. Heiner (Eds.), CMSB 2012
%  \item Scalable Steady State Analysis of Boolean Biological Regulatory Net-
%    works Ay F, Xu F, Kahveci T (2009) Scalable Steady State Analysis
%    of Boolean Biological Regulatory Networks. PLoS ONE 4(12): e7992.
%\end{itemize}
%This can be in turn be nuanced as some papers only focus on rather specific aspects (like steady-states for instance).}

In order to address the formal analysis of additional dynamical properties \modMM{(\eg state reachabilities)} within very large BRNs, we recently
introduced in \cite{PMR10-TCSB} a new formalism, named the \emph{Process Hitting} (PH), to model
concurrent systems having components with a few qualitative levels.
A PH model describes, in an atomic manner, the possible evolutions of a process (representing one
component at one level) triggered by the hit of at most one other process in the system.
This framework can be seen as a special class of formalisms like Petri Nets or Communicating Finite
State Machines, where the arity and the effect of synchronizations between components are
restricted.
Thanks to the particular structure of interactions within a PH, very efficient static analysis
methods have been developed to over- and under-approximate reachability properties making tractable
the formal analysis of the qualitative dynamics of BRNs with hundreds or thousands of components,
which was impossible with other state-of-the-art approaches \cite{PMR12-MSCS,PAK13-CAV,FPMR13-CS2Bio}.
 
The PH is suitable to model BRNs with different levels of abstraction in the specification of
cooperations (associated influences) between components. \modOR{The concept of cooperation refers to the} \modMM{way two (or more) components jointly influence a third one. In other words, it captures the logical functions stating how} \modOR{various elements coalesce and act together upon an other element among the network.}
\modLP{In particular, PH allows to specify partial (nondeterministic) cooperations that superpose the dynamics of 
candidate fully-specified (deterministic) cooperations.
This permits to model BRNs with a partial knowledge on precise evolution functions for components
by capturing the largest (the most general) dynamics, without a combinatorics enumeration of
compatible Thomas models.}

\medskip

The aim of this paper is to formally establish links between several layers suited for specifying the
qualitative dynamics of BRNs, namely:
the Interaction Graph (IG), referencing the sign of direct regulations between components;
the Process Hitting (PH), which permits to specify the qualitative dynamics of the system with
various degrees of knowledge for joint regulations;
and Thomas' parameters, which completely specify the functions governing the evolution of
components.
Another motivation for the work depicted in this paper is that it constitutes an important step
to go further in the study of large biological regulatory networks systems beyond simple analysis:
it makes it possible to partly control some behaviors.
Indeed, when one wants to modify some behaviors
(\eg so as to avoid the reachability of some states or to create stable states)
we intend to be able to get upstream the results in accordance at the regulatory level.
The outcome of the present work is that we could make such changes on the PH model
and then recover afterwards the BRN corresponding to the transformed behavioral system.
In a more general framework and in an interesting way,
we should be able to synthesize families of BRNs having some featuring behavioral properties.

Firstly, we derive rules that, given a PH model, infer the IG corresponding to the dynamics of the
encoded BRN.
The obtained IG relates only components that have actual influence in the dynamics.
This typically results into IGs that contain less interactions than an hypothetical starting prior
IG, as the specification of component evolutions may reveal non-functional regulations.
Based on the derived IG, various static analysis allows to conclude on global properties
of the system dynamics (see \cite{PR11-SASB} for a short survey).
The most known are the Thomas' conjectures (that have been proved since,
\eg \cite{RRT08,RiCo07,Richard2010378}
for Boolean and discrete networks),
which relate the absence of
positive cycle to the impossibility for multi-stationarity (distinct attractors),
and the absence of negative cycle to the impossibility for oscillatory behaviors.

Second, we tackle the inference of Thomas' parameters that are compatible with a given PH
model, \ie that fully specify the evolution of components while respecting the cooperations
specified in PH.
More formally, the resulting BRN dynamics is ensured to be included in the PH dynamics: any
transition in the (asynchronous) Thomas model exists in the PH model.
In practice, the number of compatible Thomas parameters can be huge for large networks where only a
few cooperations are specified.
This highlights that the PH is suitable to analyze dynamics of large-scale networks
where the knowledge on cooperations is often incomplete:
instead of dealing with a possibly non-tractable number of Thomas models, one can already capture
precise dynamical features using a single PH model which gathers all the plausible dynamics.

Finally, we discuss on an implementation of these inference schemes using Answer Set
Programming (ASP) \cite{Baral03},  which turns out to be effective for these enumerative searches.
The whole method is also applied to
the model of the bacteriophage lambda immunity control with 4 components,
in order to give a detailed application of our method,
and to the biological model of the ERBB receptor-regulated G1/S transition,
which contains 20 components
and allows to show how joint actions can be defined inside a PH model in order
to refine its dynamics and study a more precise class of underlying Thomas models.
%\todo{describe application}

Our work is related to the approach of \cite{Khalis09} which relies on temporal logic, and to \cite{20646302,DBLP:conf/ipcat/CorblinFTCT12} that use constraint programming.
All three of them aim at determining a class of models which are consistent with available partial data on the regulatory structure and dynamical properties.
%These classes are built in order to infer properties common to some studied models.
%In our approach, we intend to focus on the Thomas' parameters inference
Our method is based on a model rather than on constraints, which allows to define some properties on the system structure (such as cooperations).
Furthermore, we claim that we are able to deal with larger biological networks.
Finally, it must be noticed that we are not interested in this paper in the derivation of one PH
from a BRN (which was previously described in \cite{PMR10-TCSB}) but, on the contrary, in finding
out a set of BRNs from one PH.

The contributions presented in this paper significantly extend and improve the preliminary results
introduced in \cite{FPIMR12-CMSB}.
In addition to the improvement of the efficiency and accuracy of the IG inference, we have added support for
\modLP{non-monotonous} regulations, that are regulations being positive or negative depending on a particular
context.
This allows to apply our methodology to a wider class of models.
Furthermore, some parts have been completely rewritten in order to explain more precisely the efficiency of the ASP implementation,
and give a new detailed biological application of our work.

\paragraph{Outline}
\pref{sec:frameworks} recalls the PH and Thomas frameworks;
\pref{sec:infer-IG} defines the IG inference from PH;
\pref{sec:infer-K} details the parameters inference and the enumeration of Thomas parametrizations compatible with a PH;
\pref{sec:impl} discusses the implementation of the latter in ASP.
\pref{sec:examples} illustrates our method on a biological model
and discusses its applicability on large models.

\paragraph{Notations}
$\segm{i}{j}$ is the set of integers $\{ i, i+1, \dots, j \}$.
