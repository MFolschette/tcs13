\subsection{The Process Hitting framework}
\label{ssec:PH}

We recall here the definition and semantics of the Process Hitting (PH), and its usage to model
cooperation between concurrent components.
Two examples of PH modeling a BRN at different abstraction levels are given.
They serve as running examples in the rest of this article.

\medskip

A PH (\pref{def:PH}) gathers a finite number of concurrent \emph{processes}
grouped into a finite set of \emph{sorts}.
A process belongs to a unique sort and is noted $a_i$ where $a$ is the
sort and $i$ the identifier of the process within the sort $a$.
At any time, one and only one process of each sort is present; a state of the PH thus corresponds to the set of such processes.
 
The concurrent interactions between processes are defined by a set of
\emph{actions}.
Actions describe the replacement of a process by another of the same sort
conditioned by the presence of at most one other process in the current
state of the PH.
An action is denoted by $\PHfrappe{a_i}{b_j}{b_k}$ where $a_i,b_j,b_k$ are processes
of sorts $a$ and $b$.
It is required that $b_j\neq b_k$ and that $a=b\Rightarrow a_i=b_j$.
An action $h=\PHfrappe{a_i}{b_j}{b_k}$ is read as ``$a_i$ \emph{hits} $b_j$ to
make it bounce to $b_k$'', and
$a_i,b_j,b_k$ are called respectively \emph{hitter}, \emph{target} and
\emph{bounce} of the action, and can be referred to as
$\PHhitter(h), \PHtarget(h), \PHbounce(h)$, respectively.

\begin{definition}[Process Hitting]\label{def:PH}
A \emph{Process Hitting} is a triple $(\PHs,\PHl,\PHa)$:
\begin{itemize}
\item $\PHs \DEF \{a,b,\dots\}$ is the finite set of \emph{sorts};
\item $\PHl \DEF \prod_{a\in\PHs} \PHl_a$ is the set of states with $\PHl_a = \{a_0,\dots,a_{l_a}\}$
the finite set of \emph{processes} of sort $a\in\Sigma$ and $l_a$ a positive integer with
	$a\neq b\Rightarrow \forall(a_i,b_j)\in\PHl_a\times\PHl_b,a_i\neq b_j$;
\item $\PHa \DEF \{ \PHfrappe{a_i}{b_j}{b_k}, \ldots \mid
					(a,b)\in\PHs^2 \wedge (a_i,b_j,b_k)\in \PHl_a\times\PHl_b\times\PHl_b$ \\
	\hspace*{2cm} $\wedge b_j\neq b_k \wedge a=b\Rightarrow a_i=b_j\}$
			is the finite set of \emph{actions}.
\end{itemize}
$\PHproc$ denotes the set of all processes ($\PHproc \DEF \{ a_i\mid a\in\PHs \wedge a_i\in\PHl_a\}$).
\end{definition}

\noindent
The sort of a process $a_i$ is referred to as $\PHsort(a_i)=a$ and the set of
sorts present in an action $h\in\PHa$ as 
$\PHsort(h) = \{\PHsort(\PHhitter(h)),\PHsort(\PHtarget(h))\}$.
Given a state $s\in \PHl$, the process of sort $a\in\PHs$ present in $s$ is
denoted by $\PHget{s}{a}$, that is the $a$-coordinate of the state $s$.
If $a_i\in \PHl_a$, we define the notation $a_i\in s \EQDEF \PHget{s}{a}=a_i$.

An action $h=\PHfrappe{a_i}{b_j}{b_k} \in\PHa$ is \emph{playable} in $s\in L$
if and only if $\PHget{s}{a}=a_i$ and $\PHget{s}{b}=b_j$.
In such a case, $(s\play h)$ stands for the state resulting from the play of
the action $h$ in $s$, that is 
$\PHget{(s\play h)}{b} = b_k$ and 
$\forall c\in\PHs, c\neq b, \PHget{(s\play h)}{c} = \PHget{s}{c}$.
For the sake of clarity, $((s\play h)\play h')$, $h'\in\PHa$ is abbreviated as
$(s\play h\play h')$.

\begin{example}
\pref{fig:runningPH-1} represents a PH $(\PHs,\PHl,\PHa)$ with
$\PHs = \{a,b,c\}$,
$\PHl_a = \{a_0,a_1,a_2\}$,
$\PHl_b = \{b_0, b_1\}$,
$\PHl_c = \{c_0, c_1\}$, and
\begin{align*}
\PHa & = \{
	\PHfrappe{a_2}{b_1}{b_0},
&&  \PHfrappe{b_0}{a_2}{a_1},
&&	\PHfrappe{c_0}{a_2}{a_1},\\
&&& \PHfrappe{b_0}{a_1}{a_0},
&&	\PHfrappe{c_0}{a_1}{a_0},\\
&&& \PHfrappe{b_1}{a_0}{a_1},
&&	\PHfrappe{c_1}{a_0}{a_1},\\
&&& \PHfrappe{b_1}{a_1}{a_2},
&&	\PHfrappe{c_1}{a_1}{a_2} \}\enspace.
\end{align*}
The action $h=\PHfrappe{b_1}{a_1}{a_2}$ is playable in the state
$s = \state{b_1,a_1,c_0}$; and $s\play h=\state{b_1,a_2,c_0}$.

\begin{figure}[t]
\centering
\scalebox{1.3}{
\begin{tikzpicture}
\TSort{(0,0)}{a}{3}{r}
\TSort{(-3,0.5)}{b}{2}{l}
\TSort{(3,0.5)}{c}{2}{r}

\THit{b_1}{very thick}{a_0}{.west}{a_1}
\THit{b_1}{very thick}{a_1}{.north west}{a_2}
\THit{b_0}{}{a_2}{.west}{a_1}
\THit{b_0}{}{a_1}{.west}{a_0}

\path[bounce, bend left=60]
\TBounce{a_1}{very thick}{a_2}{.south}
\TBounce{a_0}{very thick}{a_1}{.south}
;
\path[bounce, bend right=60]
\TBounce{a_2}{}{a_1}{.north}
\TBounce{a_1}{}{a_0}{.north}
;

\THit{c_1}{very thick}{a_0}{.east}{a_1}
\THit{c_1}{very thick}{a_1}{.north east}{a_2}
\THit{c_0}{}{a_2}{.east}{a_1}
\THit{c_0}{}{a_1}{.east}{a_0}

\path[bounce, bend right=60]
\TBounce{a_1}{very thick}{a_2}{.south east}
\TBounce{a_0}{very thick}{a_1}{.south east}
;
\path[bounce, bend left=60]
\TBounce{a_2}{}{a_1}{.north}
\TBounce{a_1}{}{a_0}{.north east}
;

\THit{a_2}{bend right}{b_1}{.north east}{b_0}
\path[bounce, bend left=80]
\TBounce{b_1}{out=100,in=140}{b_0}{.north}
;

\end{tikzpicture}
}
\caption{\label{fig:runningPH-1}
A Process Hitting (PH) example.
Sorts are represented by labeled boxes, and processes by circles (ticks are
the identifiers of the processes within the sort, for instance, $a_0$ is the
process ticked $0$ in the box $a$).
An action (for instance $\PHfrappe{b_1}{a_1}{a_2}$) is represented by a pair of
directed arcs, having the hit part ($b_1$ to $a_1$) in plain line and the bounce
part ($a_1$ to $a_2$) in dotted line.
Here, actions involving $b_1$ or $c_1$ are in thick lines.
%The current state is represented by the grayed processes:
%$\state{a_0,b_1,c_0,d_0}$.
}
\end{figure}

This PH example actually models a BRN where the component $a$ has three qualitative
levels and components $b$ and $c$ are boolean.
In this BRN, $b$ and $c$ activate $a$, while $a$ inhibits $b$.
The inhibition of $b$ by $a$ is only effective when $a$ is at level $2$ (noted: $a_2$);
in the other cases, $b$ cannot evolve in any direction.
The activation of $a$ by $b$ ($c$) is encoded by the actions making the level of $a$ increase (resp.
decrease) when $b$ ($c$) is present (resp. absent).
It is worth noticing that the activation of $a$ by $b$ ($c$) is independent from $c$ ($b$).
This may express a lack of knowledge on the cooperation between these two regulators:
we thus model an over-approximation of the possible actions.
\end{example}

\paragraph{Modeling cooperation}
As described in \cite{PMR10-TCSB}, the cooperation between several processes to make another process bounce can be
expressed in PH by building a \emph{cooperative sort}.
\pref{fig:PH-cooperativity} shows an example of cooperation between processes $b_1$ and $c_1$ to
make $a_1$ bounce to $a_2$:
a cooperative sort $bc$ is defined with 4 processes (one for each sub-state of the presence of
processes $b_1$ and $c_1$).
For the sake of clarity, the processes of $bc$ are indexed using the sub-state they represent.
Hence, $bc_{01}$ represents the sub-state $\state{b_0,c_1}$, and so on.
Each process of sort $b$ and $c$ hit $bc$ to make it bounce to the process reflecting the status of the sorts $b$
and $c$ (e.g., $\PHfrappe{b_1}{bc_{00}}{bc_{10}}$ and $\PHfrappe{b_1}{bc_{01}}{bc_{11}}$).
Then, it is process $bc_{11}$ which hits $a_1$ to make it bounce to $a_2$ instead of
independent hits from $b_1$ and $c_1$.

\begin{figure}[p]
\centering
\scalebox{1.3}{
\begin{tikzpicture}
\TSort{(0,0)}{b}{2}{t}
\TSort{(0,-3.8)}{c}{2}{b}
\TSort{(4.5,-3)}{a}{3}{r}

\TSetTick{bc}{0}{00}
\TSetTick{bc}{1}{01}
\TSetTick{bc}{2}{10}
\TSetTick{bc}{3}{11}
% \TSetSortLbcel{bc}{$\neg a\wedge b$}
\TSort{(-0.5,-2)}{bc}{4}{b}

\THit{b_1}{very thick,bend right}{bc_0}{.north}{bc_2}
\THit{b_1}{very thick,bend right}{bc_1}{.north}{bc_3}
\THit{b_0}{}{bc_2}{.north west}{bc_0}
\THit{b_0}{}{bc_3}{.north west}{bc_1}

\THit{c_0}{}{bc_1}{.south}{bc_0}
\THit{c_0}{}{bc_3}{.south}{bc_2}
\THit{c_1}{very thick}{bc_0}{.south}{bc_1}
\THit{c_1}{very thick}{bc_2}{.south}{bc_3}

\path[bounce, bend right=25]
\TBounce{bc_2}{}{bc_0}{.north east}
\TBounce{bc_3}{}{bc_1}{.north east}
;
\path[bounce, bend left=80, distance=30]
\TBounce{bc_0}{very thick}{bc_2}{.north}
\TBounce{bc_1}{very thick}{bc_3}{.north}
;
\path[bounce, bend right]
\TBounce{bc_0}{very thick}{bc_1}{.west}
\TBounce{bc_2}{very thick}{bc_3}{.west}
;
\path[bounce, bend left]
\TBounce{bc_3}{}{bc_2}{.east}
\TBounce{bc_1}{}{bc_0}{.east}
;

\THit{bc_3}{}{a_1}{.west}{a_2}
\path[bounce, bend left=40]
\TBounce{a_1}{}{a_2}{.south west}
;

\end{tikzpicture}
}

\caption{\label{fig:PH-cooperativity}
A PH modeling a cooperativity between $b_1$ and $c_1$ to make
$a_1$ bounce to $a_2$.
Actions involving $b_1$ or $c_1$ are in thick lines.
}
\end{figure}
\begin{figure}[p]
\centering
\scalebox{1.3}{
\begin{tikzpicture}
\path[use as bounding box] (-4,-1.9) rectangle (4.5,3.9);

\TSort{(0,0)}{a}{3}{l}
\TSort{(3, 3)}{b}{2}{t}
\TSort{(3,-1)}{c}{2}{b}

\TSetTick{bc}{0}{00}
\TSetTick{bc}{1}{01}
\TSetTick{bc}{2}{10}
\TSetTick{bc}{3}{11}
% \TSetSortLbcel{bc}{$\neg a\wedge b$}
\TSort{(-3,-0.5)}{bc}{4}{l}

\THit{bc_3}{}{a_1}{.north west}{a_2}
\THit{bc_0}{}{a_1}{.south west}{a_0}
\path[bounce]
\TBounce{a_1}{bend left}{a_2}{.south west}
\TBounce{a_1}{bend right}{a_0}{.north west}
;

\THit{b_0}{}{a_2}{.east}{a_1}
\THit{b_1}{}{a_0}{.north east}{a_1}
\path[bounce]
\TBounce{a_2}{bend left}{a_1}{.north east}
\TBounce{a_0}{bend right=20}{a_1}{.south}
;

\THit{c_0}{bend right}{a_2}{.south east}{a_1}
\THit{c_1}{bend right}{a_0}{.east}{a_1}
\path[bounce]
\TBounce{a_2}{bend left=20}{a_1}{.north}
\TBounce{a_0}{bend right=30}{a_1}{.south east}
;

\path[dashed,hit]
	(2,-1.3) edge[bend left=10] (-2.3,-0.7)
	(2.2, 3.3) edge[bend right=10] (-2.3,3)
;

\THit{a_2}{bend left,out=40,in=80}{b_1}{.north west}{b_0}
\path[bounce, bend right]
\TBounce{b_1}{}{b_0}{.east}
;

\end{tikzpicture}
}

\caption{\label{fig:runningPH-2}
PH resulting from the refinement of the one in \pref{fig:runningPH-1} by the
specification of several cooperations.
The actions from $b$ and $c$ to the cooperative sort $bc$ are identical to those defined in
\pref{fig:PH-cooperativity} and are represented here by a single dashed arc.
}
\end{figure}

We note that cooperative sorts are standard PH sorts and do not involve any
special treatment regarding the semantics of related actions.

When the number of cooperating processes is large, it is possible to chain several cooperative sorts
to prevent the combinatoric explosion of the number of processes created within cooperative sorts.
For instance, if $b_1$, $c_1$, and $d_1$ cooperate, one can create a cooperative sort $bc$ with 4
processes reflecting the presence of $b_1$ and $c_1$, and a cooperative sort $bcd$ with 4 processes
reflecting the presence of $bc_{11}$ and $d_1$.  Such constructions are helpful in PH
as the static analysis of dynamics developed in \cite{PMR12-MSCS} does not suffer from the number of
sorts, but on the number of processes within a single sort.

While the construction of cooperation in PH allows to encode any boolean functions
between cooperating processes \cite{PMR10-TCSB}, it is worth noticing they introduce a temporal
shift in their application.
This allows the existence of interleaving of actions leading to a cooperative sort representing a
past sub-state of the presence of the cooperative processes.
The resulting behavior is then an over-approximation
of the realization of an instantaneous cooperation.
However, it is possible to obtain the exact behavior of the cooperations by using the semantics with prioritized actions defined in~\cite{FPMR13-CS2Bio}.
Although this new semantics is not further described here, the inferences given in \pref{sec:infer-IG} \& \ref{sec:infer-K} do ignore the temporal shift induced by the cooperative sorts.

\begin{example}
The PH in \pref{fig:runningPH-2} results from the refinement of the PH in \pref{fig:runningPH-1}
where several cooperations have been specified.
In particular, the bounce to $a_2$ is the result of a cooperation between $b_1$ and $c_1$; and the
bounce to $a_0$ of a cooperation between $b_0$ and $c_0$.
Hence, this PH expresses a BRN where $a$ requires both $b$ and $c$ active to reach its
highest level, and $a$ does not become inactive unless both $b$ and $c$ are inactive.
\end{example}
