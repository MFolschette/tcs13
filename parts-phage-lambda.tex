
\modMF{
\subsection{Detailed example: the bacteriophage lambda immunity control}
\label{ssec:phage-lambda}
}

\newcommand{\cI}{\f{cI}}
\newcommand{\cro}{\f{cro}}
\newcommand{\cII}{\f{cII}}
\newcommand{\N}{\f{N}}

\newcommand{\tcIcrocII}{cI\mbox{-}cro\mbox{-}cII}
\newcommand{\tcIcroN}{cI\mbox{-}cro\mbox{-}N}
\newcommand{\tcIcro}{cI\mbox{-}cro}
\newcommand{\cIcrocII}{\f{\tcIcrocII}}
\newcommand{\cIcroN}{\f{\tcIcroN}}
\newcommand{\cIcro}{\f{\tcIcro}}

\newcommand{\tcrocII}{cro\mbox{-}cII}
\newcommand{\crocII}{\f{\tcrocII}}

In order to illustrate the IG inference developed as \pref{sec:infer-IG},
this first subsection focuses on a model of the lambda phage immunity control,
which has been widely studied mainly for its particular switch mechanism,
allowing this virus to “chose” between lysis and lysogenization.
These two possible responses are characterized by different dynamics that lead to two
separate attractors.

We consider here the model with four genes proposed in \cite{thieffry_dynamical_1995},
which will serve as example to detail the inference method proposed in the previous sections.
This model can be straightforwardly represented under the form of a Process Hitting
with 4 components %($\Gamma = \{ \cI, \cro, \cII, \N \}$)
and 4 cooperative sorts: %($\PHs \setminus \Gamma = \{ \cIcrocII, \cIcroN, \cIcro \}$).
\[\PHs = \{ \underbrace{\cI, \cro, \cII, \N}_{\Gamma},
  \underbrace{\cIcrocII, \cIcroN, \cIcro, \crocII}_{\Delta} \}\]
The four components have respectively 3, 4 2 and 2 expression levels,
which we denote:
\begin{align*}
  \PHl_{\cI} &= \{ \cI_0, \cI_1, \cI_2 \} &
  \PHl_{\cro} &= \{ \cro_0, \cro_1, \cro_2, \cro_3 \} \\
  \PHl_{\cII} &= \{ \cII_0, \cII_1 \} &
  \PHl_{\N} &= \{ \N_0, \N_1 \} \\
\end{align*}

The full Process Hitting model is not represented here due to a lack of space,
but excerpts are given in \pref{fig:phage-lambda-cIcro-N} and \pref{fig:phage-lambda-cIcro-cro}.
The whole set of actions is given in appendix.



\subsubsection{Regulation of $\cI$ on $\N$}

We first study the influence of $\cI$ on $\N$.
The relevant part of the Process Hitting model for this problem is depicted in
\pref{fig:phage-lambda-cIcro-N}.
It involves a third component $\cro$ and the cooperative sort $\cIcro$;
indeed, by studying the full model, we find:
\[\PHpredec{\N} = \{ \cI, \cro, \cIcro \} \qquad
  \PHpredecgene{\N} = \{ \cI, \cro \} \qquad
  X(\N) = \{ \{ \cI, \cro \} \}\]
Thus, we only have to focus on the group of regulators $\{ \cI, \cro \}$
in order to analyze the regulation $\cI \rightarrow \N$.
The idea behind the IG inference developed in \pref{sec:infer-IG}
is to make the level of the considered regulator vary (here, $\cI$)
while keeping constant all the other regulators in the same group (here, only $\cro$)
and to observe the results on the evolution of the observed component (here, $\N$).
In other words, if increasing the sole level of $\cI$ makes $\N$ tend to decrease,
then a negative local influence of $\cI$ on $\N$ is inferred.

\begin{figure}[p]
\centering
\scalebox{1.3}{
\begin{tikzpicture}
  %\path[use as bounding box] (-4,-1.9) rectangle (4.5,3.9);
  
  % Sortes
  \TSetSortLabel{cI}{$\cI$}
  \TSort{(0,3.5)}{cI}{3}{l}
  \TSetSortLabel{cro}{$\cro$}
  \TSort{(0,-1.5)}{cro}{4}{l}
  \TSetSortLabel{N}{$\N$}
  \TSort{(7.5,1.5)}{N}{2}{r}
  
  % Sorte coopérative
  \newcommand{\ttable}[3]{\scriptsize%
    $#1\left\{\begin{array}{@{}l}\cI #2 1\\\cro #3 2\end{array}\right.$}
  \TSetTick{cIcro}{0}{\ttable{FF}{<}{<}}
  \TSetTick{cIcro}{1}{\ttable{FT}{<}{\geq}}
  \TSetTick{cIcro}{2}{\ttable{TF}{\geq}{<}}
  \TSetTick{cIcro}{3}{\ttable{TT}{\geq}{\geq}}
  \TSetSortLabel{cIcro}{$\cIcro$}
  \TSort{(4,0.5)}{cIcro}{4}{r}
  
  % Actions de MàJ
  \THit{cI_1}{}{cIcro_0}{.west}{cIcro_2}
  \THit{cI_1}{}{cIcro_1}{.160}{cIcro_3}
  \THit{cI_2}{}{cIcro_0}{.west}{cIcro_2}
  \THit{cI_2}{}{cIcro_1}{.160}{cIcro_3}
  \path[bounce,bend left]
    \TBounce{cIcro_0}{}{cIcro_2}{.240}
    \TBounce{cIcro_1}{}{cIcro_3}{.south west} ;
  \THit{cI_0}{}{cIcro_2}{.160}{cIcro_0}
  \THit{cI_0}{}{cIcro_3}{.150}{cIcro_1}
  \path[bounce,bend right]
    \TBounce{cIcro_2}{}{cIcro_0}{.north west}
    \TBounce{cIcro_3}{}{cIcro_1}{.120} ;
  \THit{cro_2}{}{cIcro_0}{.210}{cIcro_1}
  \THit{cro_2}{}{cIcro_2}{.200}{cIcro_3}
  \THit{cro_3}{}{cIcro_0}{.210}{cIcro_1}
  \THit{cro_3}{}{cIcro_2}{.200}{cIcro_3}
  \path[bounce,bend left]
    \TBounce{cIcro_0}{}{cIcro_1}{.240}
    \TBounce{cIcro_2}{}{cIcro_3}{.south} ;
  \THit{cro_0}{}{cIcro_3}{.190}{cIcro_2}
  \THit{cro_0}{}{cIcro_1}{.200}{cIcro_0}
  \THit{cro_1}{}{cIcro_3}{.190}{cIcro_2}
  \THit{cro_1}{}{cIcro_1}{.200}{cIcro_0}
  \path[bounce,bend right]
    \TBounce{cIcro_1}{}{cIcro_0}{.north}
    \TBounce{cIcro_3}{}{cIcro_2}{.120} ;
  
  % Actions sortantes
  \THit{cIcro_3}{}{N_1}{.west}{N_0}
  \path[bounce,bend right] \TBounce{N_1}{}{N_0}{.north} ;
  \THit{cIcro_2}{}{N_0}{.north west}{N_1}
  \path[bounce,bend left] \TBounce{N_0}{}{N_1}{.260} ;
  \THit{cIcro_1}{}{N_0}{.west}{N_1}
  \path[bounce,bend left] \TBounce{N_0}{}{N_1}{.230} ;
  \THit{cIcro_0}{}{N_0}{.south west}{N_1}
  \path[bounce,bend left=40] \TBounce{N_0}{}{N_1}{.200} ;
\end{tikzpicture}
}
\caption{\label{fig:phage-lambda-cIcro-N}%
  A portion of the PH model used in this section modeling the phage lambda immunity response.
  This part of the whole model only depicts the interactions towards $\N$ from components
  $\cI$ and $\cro$.
  The processes of the cooperative sort $\cIcro$ are labeled by the configuration they represent,
  given the values of $\cI$ and $\cro$ regarding their respective thresholds.
}
\end{figure}

We first note that in the case where $\cro_3$ is present, we have from \pref{eq:ctx-sigma}:
\begin{align*}
  \ctx_\N^{\{ \cI, \cro \}}(\state{\cI_0, \cro_3, \N_0})
    &= \{ \cI_0, \cro_3, \N_0, \cIcro_{FT} \} \\
  \ctx_\N^{\{ \cI, \cro \}}(\state{\cI_1, \cro_3, \N_0})
    &= \{ \cI_1, \cro_3, \N_0, \cIcro_{TT} \} \\
  \ctx_\N^{\{ \cI, \cro \}}(\state{\cI_2, \cro_3, \N_0})
    &= \{ \cI_2, \cro_3, \N_0, \cIcro_{TT} \}
\end{align*}
We can then deduce the following directions of evolution from \pref{eq:irB}:
\begin{align*}
  \irB_\N^{\{ \cI, \cro \}}(\state{\cI_0, \cro_3, \N_0}) &= \{ \N_1 \} \\
%     \text{because } \state{\cI_0, \cro_3, \N_0} \in \Phi(\N_0) \\
%     \text{(from \pref{eq:B-no-bounce})} \\
  \irB_\N^{\{ \cI, \cro \}}(\state{\cI_1, \cro_3, \N_0}) &= \{ \N_0 \} \\
%     \text{because } \state{\cI_0, \cro_3, \N_0} \in \Phi(\N_0) \\
%     \text{(from \pref{eq:B-bounce})} \\
  \irB_\N^{\{ \cI, \cro \}}(\state{\cI_2, \cro_3, \N_0}) &= \{ \N_0 \}
%     \text{because } \state{\cI_0, \cro_3, \N_0} \in \Phi(\N_0) \\
%     \text{(from \pref{eq:B-bounce})}
\end{align*}
          %      eq:inference-edges-ba
Therefore, \pref{eq:edges-inference} gives: $\cI \xrightarrow{1} \N \in \hat{E}_-$.
The configurations using the other processes of $\cro$ ($\cro_0$, $\cro_1$ and $\cro_2$)
do not add more information on the influences towards $\N$.
Thus, given \pref{pps:inference-IG}, it comes: $\GRNedgef{\cI}{-}{1}{\N} \in E$.

We note that an equivalent work on $\cro$ also gives the second influence on $\N$:
$\GRNedgef{\cro}{-}{2}{\N} \in E$.



\subsubsection{Self-regulation of $\cro$ due to $\cI$}

We now wish to study the self-regulation of $\cro$.
For this, we will show that studying only the actions self-hitting $\cro$ is not enough;
indeed, it is required to also study the interactions with $\cI$.
We have the following results from the whole model:
\[\PHpredec{\cro} = \{ \cI, \cro \} \qquad
  \PHpredecgene{\cro} = \{ \cI, \cro \} \qquad
  X(\cro) = \{ \{ \cI \}, \{ \cro \} \}\]
The interactions of $\cro$ and $\cI$ on $\cro$ have been represented in
\pref{fig:phage-lambda-cIcro-cro}.

\begin{figure}[p]
\centering
\scalebox{1.3}{
\begin{tikzpicture}
  %\path[use as bounding box] (-4,-1.9) rectangle (4.5,3.9);
  
  % Sortes
  \TSort{(0,1)}{cI}{3}{l}
  \TSort{(4,0)}{cro}{4}{r}
  
  % Actions sortantes
  \THit{cro_3}{out=north east,in=south east,selfhit,min distance=50pt}{cro_3}{.south east}{cro_2}
    \path[bounce, bend left] \TBounce{cro_3}{}{cro_2}{.north east} ;
  
  \THit{cI_2}{}{cro_3}{.210}{cro_2}
    \path[bounce,bend right=40] \TBounce{cro_3}{}{cro_2}{.west} ;
  \THit{cI_2}{}{cro_2}{.210}{cro_1}
    \path[bounce,bend right=40] \TBounce{cro_2}{}{cro_1}{.west} ;
  \THit{cI_2}{}{cro_1}{.210}{cro_0}
    \path[bounce,bend right=40] \TBounce{cro_1}{}{cro_0}{.west} ;
  
  \THit{cI_1}{}{cro_0}{.110}{cro_1}
    \path[bounce,bend left] \TBounce{cro_0}{}{cro_1}{.south} ;
  \THit{cI_1}{}{cro_1}{.110}{cro_2}
    \path[bounce,bend left] \TBounce{cro_1}{}{cro_2}{.south} ;
  \THit{cI_1}{}{cro_2}{.110}{cro_3}
    \path[bounce,bend left] \TBounce{cro_2}{}{cro_3}{.south} ;
  
  \THit{cI_0}{}{cro_0}{.140}{cro_1}
    \path[bounce,bend left] \TBounce{cro_0}{}{cro_1}{.240} ;
  \THit{cI_0}{}{cro_1}{.140}{cro_2}
    \path[bounce,bend left] \TBounce{cro_1}{}{cro_2}{.240} ;
  \THit{cI_0}{}{cro_2}{.140}{cro_3}
    \path[bounce,bend left] \TBounce{cro_2}{}{cro_3}{.240} ;
\end{tikzpicture}
}
\caption{\label{fig:phage-lambda-cIcro-cro}%
  A portion of the PH model used in this section modeling the phage lambda immunity response.
  This part of the whole model only depicts the interactions towards $\cro$ from components
  $\cI$ and $\cro$.
}
\end{figure}

Let us first focus on the group of regulators $\{ \cro \}$.
Of course:
\[\forall \cro_i \in \PHl_{\cro}, \ctx_{\cro}^{\{ \cro \}}(\state{\cro_i}) = \{ \cro_i \}\]
and from \pref{eq:irB} and \eqref{eq:fp}:
\begin{align*}
  \irB_{\cro}^{\{ \cro \}}(\state{\cro_3}) &= \{ \cro_2 \} && \\
%     \text{(from \pref{eq:B-bounce})} \\
  \irB_{\cro}^{\{ \cro \}}(\state{\cro_2}) &= \{ \cro_1 \} && \text{and} &
    \Phi&(\cro_2) = \emptyset \\
%     \text{(from \pref{eq:B-no-bounce})} \\
  \irB_{\cro}^{\{ \cro \}}(\state{\cro_1}) &= \{ \cro_1 \} && \text{and} &
    \Phi&(\cro_1) = \emptyset \\
%     \text{(from \pref{eq:B-no-bounce})} \\
  \irB_{\cro}^{\{ \cro \}}(\state{\cro_0}) &= \{ \cro_0 \} && \text{and} &
    \state{\cI_2, \cro_0} \in \Phi&(\cro_0)
%     \text{(from \pref{eq:B-no-bounce})}
\end{align*}
We thus cannot infer any local influence from \pref{eq:edges-inference-auto}
by observing the group of regulators $\{ \cro \}$,
because $\Phi(\cro_1) = \Phi(\cro_2) = \emptyset$.
This is due to the fact that all processes of $\cI$ can hit $\cro_1$ and $\cro_2$;
these processes can therefore not be considered stable in any configuration.

Let us now focus instead on the group of regulators $\{ \cI \}$.
One again, we trivially have:
\[\forall \cI_i \in \PHl_{\cI}, \forall \cro_j \in \PHl_{\cro},
  \ctx_{\cro}^{\{ \cI \}}(\state{\cI_i, \cro_j}) = \{ \cI_i, \cro_j \}\]
Furthermore, from \pref{eq:irB}, we especially find:
\begin{align*}
%   \irB_{\cro}^{\{ \cI, \cro \}}(\state{\cI_0, \cro_0}) &= \{ \cro_1 \} &&
%   \text{(from \pref{eq:B-bounce})} \\
%   \irB_{\cro}^{\{ \cI, \cro \}}(\state{\cI_0, \cro_1}) &= \{ \cro_2 \} &&
%   \text{(from \pref{eq:B-bounce})} \\
  \irB_{\cro}^{\{ \cI \}}(\state{\cI_0, \cro_2}) &= \{ \cro_3 \} \\
%     \text{(from \pref{eq:B-bounce})} \\
  \irB_{\cro}^{\{ \cI \}}(\state{\cI_0, \cro_3}) &= \{ \cro_2 \}
%     \text{(from \pref{eq:B-bounce})}
\end{align*}
Therefore, \pref{eq:edges-inference} gives: $\cro \xrightarrow{3} \cro \in \hat{E}_-$.
% The configurations using the other processes of $\cro$ ($\cro_0$, $\cro_1$ and $\cro_2$)
% do not add more information on the influences towards $\N$.
Finally, as no more information is added by observing the other levels of $\cro$ and $\cI$,
and given \pref{pps:inference-IG}, it comes: $\GRNedgef{\cro}{-}{3}{\cro} \in E$.



\subsubsection{Conclusion on the phage lambda immunity control}

The IG inference of \pref{pps:inference-IG} applied to the whole Process Hitting modeling
the phage lambda immunity response produces the IG given in \pref{fig:phage-lambda-ig}.
This IG contains all edges included in the original model except the
self-influence on $\cI$ (namely, $\GRNedgef{\cI}{+}{2}{\cI} \notin E$);
this is due to the fact that an equivalent dynamics exists without this edge.

\begin{figure}[t]
\centering
\scalebox{1.2}{
\begin{tikzpicture}[grn]
  %\path[use as bounding box] (-1.3,-0.75) rectangle (3.5,1.5);
  \node[inner sep=0] (cI) at (0,2.5) {$\cI$};
  \node[inner sep=0] (cro) at (2.5,2.5) {$\cro$};
  \node[inner sep=0] (cII) at (0,0) {$\cII$};
  \node[inner sep=0] (N) at (2.5,0) {$\N$};
  \path[->]
    (cI) edge[bend right] node[elabel, above=-4pt] {$-2$} (cro)
          edge[bend right] node[elabel, left=-2pt] {$-2$} (cII)
          edge[pos=.85] node[elabel, above=-2pt] {$-1$} (N)
    (cro) edge[bend right] node[elabel, above=-4pt] {$-1$} (cI)
          edge[loop right] node[elabel, below=-2pt] {$-3$} (cro)
          edge[pos=.65] node[elabel, below=-2pt] {$-3$} (cII)
          edge node[elabel, right=-4pt] {$-2$} (N)
    (cII) edge[bend right] node[elabel, left=-3pt] {$+1$} (cI)
    (N) edge node[elabel, above=-4pt] {$+1$} (cII);
\end{tikzpicture}
}
\caption{\label{fig:phage-lambda-ig}%
  Result of the IG inference performed on the Process Hitting model
  of the phage lambda immunity response
  taken from \cite{thieffry_dynamical_1995}.
}
\end{figure}

Finally, we note that the parameters inference of \pref{pps:param_K} applied to the same model,
and using the output of the previous IG inference is conclusive for all parameters.
The complete parametrization produced is available in \pref{tb:phage-lambda-k}.

\begin{table}[t]
\begin{align*}
  K_{\cI, \emptyset} &= 2 &
  K_{\N, \emptyset} &= 1\\
  K_{\cI, \{\cro\}} &= 0&
  K_{\N, \{\cro\}} &= 0\\
  K_{\cI, \{\cII\}} &= 2&
  K_{\N, \{\cI\}} &= 0\\
  K_{\cI, \{\cII;cro\}} &= 2&
  K_{\N, \{\cI;cro\}} &= 0\\
  \\
  K_{\cII, \emptyset} &= 0&
  K_{\cro, \emptyset} &= 3\\
  K_{\cII, \{\cro\}} &= 0&
  K_{\cro, \{\cro\}} &= 2\\
  K_{\cII, \{\cI\}} &= 0&
  K_{\cro, \{\cI\}} &= 0\\
  K_{\cII, \{\cI;cro\}} &= 0&
  K_{\cro, \{\cI;cro\}} &= 2\\
  K_{\cII, \{\N\}} &= 1\\
  K_{\cII, \{\N;cro\}} &= 0\\
  K_{\cII, \{\N;cI\}} &= 0\\
  K_{\cII, \{\N;cI;cro\}} &= 0
\end{align*}
\caption{\label{tb:phage-lambda-k}%
  Result of the parameters inference performed on the Process Hitting model
  of the phage lambda immunity response
  taken from \cite{thieffry_dynamical_1995}.
}
\end{table}

We note two main differences between the parametrization obtained by our inference
and the one proposed in \cite{thieffry_dynamical_1995}:
\begin{itemize}
  \item The absence of the self-regulation of $\cI$ automatically reduces the size
    of the parametrization, but this does not modify the dynamics, as mentioned previously;
  \item Our inference returns $K_{\cro, \{\cI;cro\}} = 2$ instead of originally $0$;
    once again, this does not change the dynamics, and $1$ would also have been
    an acceptable value.
\end{itemize}
