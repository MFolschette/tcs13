% vim:spell spelllang=en
\subsection{Thomas' modeling}

\todo{Name Kaufman as the creator of BRNs.}

We concisely present Thomas' modeling of a Biological Regulatory Network (BRN) and its dynamics, merely inspired by
\cite{Richard06,BernotSemBRN}.
In order to enlarge the class of Thomas' models compatible with PH dynamics (w.r.t.~the presented
inference),
we propose the notion of unsigned edge modeling an interaction whose nature (activation or inhibition) is undetermined,
and we extend the classical parametrization formalism by setting parameters to intervals of values instead of single values.
We briefly discuss these additions.

\medskip

Thomas' formalism lies on two complementary descriptions of the system. First, the
\emph{Interaction Graph} (IG) models the structure of the system by defining the components'
mutual influences and the conditions of these influences.
The \emph{parametrization} then specifies the levels towards which tends a component when a given
configuration of its regulators applies.

The IG is composed of nodes ($a$, $b$, $c$, …) that represent components,
and edges ($\GRNedge{a}{s}{t}{b}$, …) labeled with a sign ($s$) and a threshold ($t$) that stand for regulations between these components (\pref{def:ig}).
The activity, concentration rate or presence of each component in a given state of the system is modeled by an abstract discrete value called \emph{expression level}.
The maximum expression level of a component $a$ is denoted $l_a$.
The sign of an edge denotes the kind of regulation it models: it can be positive ($+$), negative ($-$) or unsigned ($\uns$),
the latter meaning that the nature of the regulation cannot be reduced to a simple activation or inhibition.
Regarding the dynamics, an edge $\GRNedge{a}{s}{t}{b}$ states that $a$ influences the evolution of $b$ in a certain way when its expression level is above or equal to the threshold $t$;
when its expression level is strictly below this threshold, another kind of influence is expressed, which usually consists of the opposite influence.
%For a regulation to take place (activation or inhibition), the expression level of its head component has to be higher than its threshold;
%otherwise, the opposite influence is expressed.
The uniqueness of each regulation is forced, in order to make the following sections simpler.
The purpose of unsigned edges is discussed at the end of the current section.

\todo{Discuss about the meaning of signs (especially $+$ and $-$) and their link with parameters.
Especially, these signs introduce redundancy (and possibly contradiction) with the parameters.
\begin{itemize}
  \item We can discuss more about the choice we made here: the signs are only informative
    (it eases the read of the IG), but are used as a basis for the parameters enumeration;
  \item The reviewer suggests to remove signs, or only present them as an abstraction of the monotonicity assumption, given the levels and parameters of a BRN;
  \item It is also possible to add the monotonicity assumption in the definition of a BRN to avoid contradictions.
\end{itemize}}

\begin{definition}[Interaction Graph]
\label{def:ig}
An \emph{Interaction Graph} (IG) is a couple $\IG = (\Gamma, E)$ with $\Gamma$ the finite set of \emph{components},
and $E$ the finite set of \emph{regulations} between two nodes, labeled with a \emph{sign} and a \emph{threshold}:
$$E \DEF \{ \GRNedge{a}{s}{t}{b}, \ldots \mid a, b \in \Gamma \wedge s \in \{ +, -, \uns \} \wedge t \in \segm{1}{l_a}\} \enspace,$$
where a regulation from $a$ to $b$ is uniquely referenced:
$$\forall \GRNedge{a}{s}{t}{b} \in E, \forall \GRNedge{a}{s'}{t'}{b} \in E, s = s' \wedge t = t' \enspace.$$
\end{definition}

\todo{The notation $\GRNreg{b}$ is poorly chosen; should be replaced by $E^{-1}(b)$ for instance.}

Given this definition, we denote as a shortcut:
$E_s \DEF \{ \GRNedge{a}{s}{t}{b} \in E \}$ for $s \in \{ +, -, \uns \}$.
Furthermore, for all component $b \in \Gamma$, we denote $\GRNreg{b}$ the set of its regulators as defined in \pref{eq:grn-regulators}.
\begin{align}
\label{eq:grn-regulators}
  \GRNreg{b} \DEF \{ a\in\Gamma\mid \exists \GRNedge{a}{s}{t}{b} \in E \}
\end{align}
Then, for all component $a$ regulating $b$,
\ie if $\GRNedge{a}{s}{t}{b} \in E$,
we denote $\levels{a}{b}$ (resp.~$\ulevels{a}{b}$) the interval of expression levels of $a$ above (resp.~below) the threshold $t$ (\pref{def:levels}).
For all expression levels of $a$ that belong to $\levels{a}{b}$, $a$ is expected to have the influence corresponding to the sign $s$ on $b$;
for all expression levels belonging to $\ulevels{a}{b}$, the opposite influence is expected.
This definition holds because of the uniqueness of the edge $\GRNedge{a}{s}{t}{b}$.

\begin{definition}[Effective levels ($\levelsl$)]\label{def:levels}
If $\GRNedge{a}{s}{t}{b} \in E$, we define:
%Let $b \in \Gamma$ and $a \in \GRNreg{a}$; we define:
$$\levels{a}{b} \DEF \segm{t}{l_a} \quad \text{and} \quad \ulevels{a}{b} \DEF \segm{0}{t-1} \enspace.$$
\end{definition}

\begin{example}
\pref{fig:runningBRN}(left) represents an Interaction Graph $(\Gamma,E)$ where
$\Gamma = \{a, b, c\}$, with $l_a = 2$ and $l_b = l_c = 1$,
and:
\begin{align*}
  E_+ &= \{\GRNedgef{b}{+}{1}{a}, \GRNedgef{c}{+}{1}{a}\} &
  E_\uns &= \emptyset \\
  E_- &= \{\GRNedgef{a}{-}{2}{b}\}
\end{align*}
Hence:
\begin{align*}
  \GRNreg{a} &= \{ b, c \} &
  \GRNreg{b} &= \{ a \} \\
  \GRNreg{c} &= \emptyset
\end{align*}
We also have especially:
\begin{align*}
  \levels{a}{b} &= \segm{2}{2} & \ulevels{a}{b} &= \segm{0}{1}
\end{align*}
\end{example}

\begin{figure}[t]
\begin{minipage}{0.4\linewidth}
\centering
\scalebox{1.2}{
\begin{tikzpicture}[grn]
  \path[use as bounding box] (-0.3,-0.75) rectangle (2.5,1.5);
  \node[inner sep=0] (a) at (2,0) {a};
  \node[inner sep=0] (b) at (0,0) {b};
  \node[inner sep=0] (c) at (2,1.2) {c};
  \path[->]
    (b) edge[bend right] node[elabel, below=-2pt] {$+1$} (a)
    (c) edge node[elabel, right=-2pt] {$+1$} (a)
    (a) edge[bend right] node[elabel, above=-5pt] {$-2$} (b);
\end{tikzpicture}
}
\end{minipage}
\begin{minipage}{0.6\linewidth}
\centering
\begin{align*}
  K_{a,\emptyset} &= \segm{0}{0} & K_{b,\emptyset} &= \segm{0}{1} \\
  K_{a,\{b\}} &= \segm{1}{1} & K_{b,\{a\}} &= \segm{0}{0} \\
  K_{a,\{c\}} &= \segm{1}{1} &&\\
  K_{a,\{b,c\}} &= \segm{2}{2} & K_{c,\emptyset} &= \segm{0}{1}
\end{align*}
\end{minipage}
\caption{\label{fig:runningBRN}
  (left)
  IG example.
  Components are represented by nodes labeled with a name
  and regulations by edges labeled with their sign and threshold.
  For instance, the edge from $b$ to $a$ is labeled $+1$, which stands for: $\GRNedgef{b}{+}{1}{a}$.
  This means that if the expression level of $b$ is equal to (\ie above) 1, then $b$ activates $a$;
  otherwise, $b$ inhibits $a$.
  (right)
  Example parametrization on the left IG.
}
\end{figure}

A \emph{state} $s$ of an IG $(\Gamma, E)$ is an element in $\GRNstates \DEF \prod_{a \in \Gamma} \segm{0}{l_a}$.
$\GRNget{s}{a}$ refers to the level of component $a$ in $s$.
For each possible state, the set of \emph{resources} of a given component
is the set of regulators of this component whose expression level is above the threshold of the regulation (\pref{def:resources}).
In other words, in every state $s$, a regulator $b$ of a component $a$
is either a resource (if $\GRNget{s}{b} \in \levels{b}{a}$)
or not a resource (if $\GRNget{s}{b} \in \ulevels{b}{a}$).
%For each possible state, all regulators of a component $a$ can be divided into
%\emph{activators} and \emph{inhibitors}, given their type of interaction and expression level,
%referred to as the \emph{resources} of $a$ in this state (\pref{def:resources}).
Relying on this observation, the specificity of Thomas' approach lies in the use of discrete \emph{parameters} to represent the
focal level interval towards which the component will evolve in every configuration of the resources (\pref{def:param}).

\begin{definition}[Resources ($\GRNreslabel$)]\label{def:resources}
For a given component $a \in \Gamma$ and a state $s \in \GRNstates$,
the set of regulators of $a$ whose level in $s$ is above the related threshold %to regulate $a$
is called the set of \emph{resources} of $a$ in $s$ and is noted $\GRNres{a}{s}$:
$$\GRNres{a}{s} \DEF \{ b \in \GRNreg{a} \mid \GRNget{s}{b} \in \levels{b}{a} \}$$
\end{definition}

\begin{definition}[Parameter $K_{a,\omega}$ and Parametrization $K$]\label{def:param}
For a given component $a \in \Gamma$, and $\omega \subset \GRNreg{a}$ a set of regulators of $a$,
the \emph{parameter} $K_{a,\omega} = \segm{i}{j}$ is a non-empty interval with with $0 \leq i \leq j \leq l_a$.
The complete map $K$ of parameters on $\IG$ is called a \emph{parametrization} on $\IG$.
\end{definition}

%\todo{Discuss on the widespread use of discrete parameters + binary resources}
An IG and a parametrization make up a complete BRN.
The purpose of intervals as parameters is discussed at the end of this section.
Regarding the dynamics, an interval parameter $K_{a,\omega}$ is the set of values towards which $a$ will tend
in the states where its resources are exactly the regulators in $\omega$.
%as defined in the asynchronous dynamics of \pref{def:dynamics}.
%At last, \pref{def:dynamics} gives the asynchronous dynamics of a BRN using Thomas' parameters.
Indeed, from a given state $s$, a transition to another state $s'$ is possible provided that
exactly one component $a$ evolves of one expression level towards the closest value of the interval $K_{a,\GRNres{a}{s}}$,
as stated by the definition of the transition relation $\GRNtrans{s}{s'}$ (\pref{def:dynamics}).
However, $a$ cannot evolve if its expression level already belongs to the interval $K_{a,\GRNres{a}{s}}$.

\begin{definition}[Asynchronous dynamics ($\GRNtrans{}{}$)]\label{def:dynamics}
The dynamics of a BRN using Thomas' parameters is given by the transition relation $\GRNtrans{}{}\ \in \GRNstates \times \GRNstates$ defined by:
\begin{align*}
  \forall s, s' \in \GRNstates, \GRNtrans{s}{s'} &\Longleftrightarrow \exists a \in \Gamma,
  \GRNget{s}{a} \notin K_{a, \GRNres{a}{s}} \wedge \GRNget{s'}{a} = \GRNget{s}{a} + \delta^a(s) \\
    & \qquad\quad \wedge \forall b \in \Gamma, b \neq a \Rightarrow \GRNget{s}{b} = \GRNget{s'}{b}
\end{align*}
with: $\delta^a(s) = 
  \begin{cases}
    +1 & \text{if } \GRNget{s}{a} < K_{a, \GRNres{a}{s}} \\
    -1 & \text{if } \GRNget{s}{a} > K_{a, \GRNres{a}{s}} \\
  \end{cases}$
\end{definition}
The notations “$<$” and “$>$” comparing an integer and an interval were defined at the end of \pref{sec:intro}.



\begin{example}
\pref{fig:runningBRN}(right) gives a Parametrization on the IG of \pref{fig:runningBRN}(left).
In this BRN, the following transitions are possible given the semantics defined in \pref{def:dynamics}:
$$\GRNstate{a_0, b_1, c_1} \rightarrow \GRNstate{a_1, b_1, c_1} \rightarrow \GRNstate{a_2, b_1, c_1} \rightarrow
\GRNstate{a_2, b_0, c_1} \rightarrow \GRNstate{a_1, b_0, c_1} \enspace,$$
where $a_i$ denotes the component $a$ at level $i$.
This sequence of states ends in a steady state: no evolution is possible in $\GRNstate{a_1, b_0, c_1}$.
Due to the semantics of interval parameters and as $K_{b,\emptyset} = \segm{0}{1}$, no auto-regulation on $b$ is needed to prevent its evolution when $a$ is not at level $2$.
\end{example}



\paragraph{Interaction Graph of the dynamics}
From the dynamics specification of a BRN, one can infer back the IG that contain only the functional
regulations.
Let us define
$f^a(s) = s[a]$ if $s[a]\in K_{a, \GRNres{a}{s}}$
and
$f^a(s) = s[a]+\delta^a(s)$ otherwise.
A positive (resp. negative) interaction from $b$ to $a$ is inferred if there exists a state $s$ such
that increasing $b$ in $s$ would increase (resp. decrease) $f^a$; in other words, if
there exists a state $s$ so that:
$f^a(s\{b_i\}) < \text{(resp. $>$) } f^a(s\{b_{i+1}\})$, 
where $i < l_b$ and $s\{b_i\}$ denotes the state $s$ where the component $b$ is assigned to $b_i$.
Then, such an IG can be used to derive global properties on the dynamics
(e.g., \cite{Richard2010378,PR11-SASB}).

\paragraph{Discussion on maximum level restrictions}
\todo{The reviewer says that the discussion on the number of maximum levels has
already been fully developed in other articles related to Thomas and that it is not
necessary to develop this just to state that a PH may accept any number of levels.}
The maximum level $l_a$ of a component $a$ is usually chosen equal to the number of outgoing edges of $a$ in the IG, \ie the number of components that $a$ regulates.
Indeed, each discrete expression level in $\segm{0}{l_a}$ stands for an arbitrary set of concentration levels of the protein coded by $a$
and for which it has a certain influence on several other components that it regulates.
Allowing more expression levels makes several levels useless as they represent the same influence of $a$ than some other levels;
on the contrary, reducing the maximum level requires that several thresholds on outgoing edges of $a$ share the same value, which is biologically unlikely.
However, these considerations are usually taken into account when one has to directly build a model in this framework, but are not mandatory.
In the scope of this paper, we allow any number of levels for components, without considering the number of outgoing edges,
as the number of processes in a PH sort is not constrained in any way.
For instance, in the example of \pref{fig:runningBRN},
it would be more relevant to have $l_a = 1$ and $\GRNedgef{a}{-}{1}{b} \in E$, as $a$ has only one outgoing edge,
but the value $l_a = 2$ makes the example more interesting; this is why $\GRNedgef{a}{-}{2}{b} \in E$ can be arbitrarily chosen.

\paragraph{Discussion on unsigned edges}
\todo{Discuss the meaning of signs (cf. supra).
Also, the reviewer proposes to move this discussion before in the text,
and to define unsigned IGs before signed IGs as the latter is just a consequence of the monotonicity assumption.}
Unsigned edges allow to model regulations whose trend is more complex than a simple activation or inhibition.
It has to be noted that using unsigned edges in addition to the regular positive and negative ones does not add expressivity to the framework.
Indeed, signs have no impact on the parametrization (\pref{def:param}) or the dynamics (\pref{def:dynamics}).
They do have an impact later in this paper, when admissible parametrizations are enumerated (\pref{ssec:admissible-K}).
However, signs in an IG are useful to understand the trend of a regulation.
PH actions are very atomistic and the inference of the underlying BRN proposed in \pref{sec:infer-IG}
allows to find the global trends of regulations between components,
which may turn out to be difficult to perform by hand.
As signs have no direct influence on the semantics, unsigned edges can be convenient to model an IG with partial knowledge on some regulations.

\paragraph{Discussion on interval parameters}
Whereas most of the literature use single value parameters (e.g., $K_{a,\omega} = i$), we introduce
in this paper intervals as parameter values (e.g., $K_{a,\omega} = \segm{i}{j}$) for specifying
the discrete functions for components.
These interval parameters are mainly introduced for convenience, and it is worth noticing that they
do not add extra expressivity for modeling multivalued networks in the general case.
However, interval parameters are more expressive than Thomas' parameters with single value.
First, we remark that if  $K_{a,\omega} = \segm{i}{j}$ with $i < j$, then, necessarily there exists
a positive edge from $a$ to $a$ in the interaction graph of the dynamics.
Then, let us consider the case where $K_{a,\omega} = \segm{i}{i+2}$ (interval of three values), and
let $s$ be a state such that $\GRNres{a}{s} = \omega$.
From \pref{def:dynamics}, if $\GRNget{s}{a} \in K_{a,\omega}$, then $a$ cannot evolve in $s$;
therefore, $a$ has three (local) fixed points ($i$, $i+1$ and $i+2$).
This behavior cannot be reproduced with single value parameters, because it is impossible to distinguish three different sets of resources corresponding to the three involved states.
At most two steady-states can be modeled by using a self-regulation $\GRNedge{a}{s}{t}{a}$, allowing to distinguish two cases ($\GRNget{s}{a} < t$ and $\GRNget{s}{a} \geq t$), which is not enough.

