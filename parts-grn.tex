% vim:spell spelllang=en
\subsection{Thomas' modeling}
\label{ssec:thomas}

%\todo{Name Kaufman as the creator of BRNs.}

\modMM{
The principle of qualitative modeling was introduced as synchronous Boolean networks by Stuart Kauffman on the one hand \cite{kauffman1969metabolic}, asynchronous Ren\'e Thomas’ networks on the other \cite{Thomas73}. Both models have been considered of interest and led to numerous works. In the following, we choose to focus on  Thomas' modelling, with its asynchronous, unitary and multi-level semantics.}

\modMM{In this paragraph, we} concisely present Thomas' modeling of a Biological Regulatory Network (BRN) and its dynamics, merely inspired by
\cite{thomas1990biological, Richard06,BernotSemBRN}.
%In order to enlarge the class of Thomas' models compatible with PH dynamics (w.r.t.~the presented
%inference),
%we propose the notion of unsigned edge modeling an interaction whose nature (activation or inhibition) is undetermined,
%and we extend the classical parametrization formalism by setting parameters to intervals of values instead of single values.
%We briefly discuss these additions.

\medskip

Thomas' formalism lies on two complementary descriptions of the system. First, the
\emph{Interaction Graph} (IG) models the structure of the system by defining the components'
mutual influences and the conditions of these influences. \modMM{
Then t}he \emph{parametrization} \modMM{
removes the ambiguity when a component is targeted by (at least) two different influences. In other words, it} specifies the levels towards which a component tends when a given configuration of its regulators applies.

The IG is composed of nodes ($a$, $b$, $c$, …) that represent components,
and edges ($\GRNedge{a}{s}{t}{b}$, …) labeled with a sign ($s$) and a threshold ($t$) that stand for regulations between these components (\pref{def:ig}).
The activity, concentration rate or presence of each component in a given state of the system is modeled by an abstract discrete value called \emph{expression level}.
The maximum expression level of a component $a$ is denoted $l_a$.
The sign of an edge denotes the kind of regulation it models: it can be positive
($+$), negative ($-$) or \modLP{non-monotonous} ($\uns$),
the latter meaning that \modLP{the regulation have different sign depending on the context \cite{Sene13-TCS}}.
Regarding the dynamics, an edge $\GRNedge{a}{s}{t}{b}$ states that $a$ influences the evolution of $b$ in a certain way when its expression level is above or equal to the threshold $t$;
when its expression level is strictly below this threshold, another kind of influence is expressed, which usually consists of the opposite influence.

\modMM{The IG is not sufficient to unambiguously define the dynamics of the system. In particular, a component may be regulated by both a positive and negative interaction derived from two different components. To refine the dynamics, a parametrization had been added to the model \cite{snoussi1989qualitative}. This parametrization allows to define focal points, \ie it clarifies the level towards which a component evolves depending on the expression of its regulators.
Some of the information included in the IG and the parametrization may appear as slighltly redundant. Especially, if the single parametrization were enriched with additional constraints (e.g., monotonicity requiring an ordering on the value of parameters), then it may directly capture the signs. However, distinguishing the IG and parametrization allows to decouple the graph (which is generally one of the prior knowledge from the biologists) from the constraints on its dynamics. It opened the way to numerous studies (e.g., necessary condition on sustained oscillations) solely based on the structure of the IG (e.g., the analysis of positive or negative circuits).}

%For a regulation to take place (activation or inhibition), the expression level of its head component has to be higher than its threshold;
%otherwise, the opposite influence is expressed.
%The uniqueness of each regulation is forced, in order to make the following sections simpler.
%The purpose of unsigned edges is discussed at the end of the current section.

\begin{definition}[Interaction Graph]
\label{def:ig}
An \emph{Interaction Graph} (IG) is a couple $\IG = (\Gamma, E)$ with $\Gamma$ the finite set of \emph{components},
and $E$ the finite set of \emph{regulations} between two nodes, labeled with a \emph{sign} and a \emph{threshold}:
$$E \DEF \{ \GRNedge{a}{s}{t}{b}, \ldots \mid a, b \in \Gamma \wedge s \in \{ +, -, \uns \} \wedge t \in \segm{1}{l_a}\} \enspace,$$
where a regulation from $a$ to $b$ is uniquely referenced:
$$\forall \GRNedge{a}{s}{t}{b} \in E, \forall \GRNedge{a}{s'}{t'}{b} \in E, s = s' \wedge t = t' \enspace.$$
\end{definition}

Given this definition, we denote as a shortcut:
$E_s \DEF \{ \GRNedge{a}{s}{t}{b} \in E \}$ for $s \in \{ +, -, \uns \}$.
Furthermore, for all component $b \in \Gamma$, we denote $\GRNreg{b}$ the set of its regulators as defined in \pref{eq:grn-regulators}.
\begin{align}
\label{eq:grn-regulators}
  \GRNreg{b} \DEF \{ a\in\Gamma\mid \exists \GRNedge{a}{s}{t}{b} \in E \}
\end{align}
Then, for all component $a$ regulating $b$,
\ie if $\GRNedge{a}{s}{t}{b} \in E$,
we denote $\levels{a}{b}$ (resp.~$\ulevels{a}{b}$) the interval of expression levels of $a$ above (resp.~below) the threshold $t$ (\pref{def:levels}).
For all expression levels of $a$ that belong to $\levels{a}{b}$, $a$ is expected to have the influence corresponding to the sign $s$ on $b$;
for all expression levels belonging to $\ulevels{a}{b}$, the opposite influence is expected.
This definition holds because of the uniqueness of the edge $\GRNedge{a}{s}{t}{b}$.

\begin{definition}[Effective levels ($\levelsl$)]\label{def:levels}
If $\GRNedge{a}{s}{t}{b} \in E$, we define:
%Let $b \in \Gamma$ and $a \in \GRNreg{a}$; we define:
$$\levels{a}{b} \DEF \segm{t}{l_a} \quad \text{and} \quad \ulevels{a}{b} \DEF \segm{0}{t-1} \enspace.$$
\end{definition}

\begin{example}
\pref{fig:runningBRN}(left) represents an Interaction Graph $(\Gamma,E)$ where
$\Gamma = \{a, b, c\}$, with $l_a = 2$ and $l_b = l_c = 1$,
and:
\begin{align*}
  E_+ &= \{\GRNedgef{b}{+}{1}{a}, \GRNedgef{c}{+}{1}{a},
    \GRNedgef{b}{+}{1}{b}, \GRNedgef{c}{+}{1}{c}\} \\
  E_- &= \{\GRNedgef{a}{-}{2}{b}\} &&
    E_\uns = \emptyset \\
\end{align*}
Hence:
\begin{align*}
  \GRNreg{a} &= \{ b, c \} &
  \GRNreg{b} &= \{ a, b \} \\
  \GRNreg{c} &= \{ c \}
\end{align*}
We also have especially:
\begin{align*}
  \levels{a}{b} &= \segm{2}{2} & \ulevels{a}{b} &= \segm{0}{1}
\end{align*}
\end{example}

\begin{figure}[t]
\begin{minipage}{0.4\linewidth}
\centering
\scalebox{1.2}{
\begin{tikzpicture}[grn]
  \path[use as bounding box] (-0.5,-0.75) rectangle (3.2,1.7);
  \node[inner sep=0] (a) at (2,0) {a};
  \node[inner sep=0] (b) at (0,0) {b};
  \node[inner sep=0] (c) at (2,1.2) {c};
  \path[->]
    (b) edge[bend right] node[elabel, below=-2pt] {$+1$} (a)
    (c) edge node[elabel, right=-2pt] {$+1$} (a)
    (a) edge[bend right] node[elabel, above=-5pt] {$-2$} (b)
    (b) edge[loop, min distance=30pt] node[elabel, above=-5pt] {$+1$} (b)
    (c) edge[loop right, min distance=30pt] node[elabel, above=0pt, xshift=-10] {$+1$} (c);
\end{tikzpicture}
}
\end{minipage}
\begin{minipage}{0.6\linewidth}
\centering
\modMF{
\begin{align*}
  K_{a,\emptyset} &= 0 & K_{b,\emptyset} &= 0 \\
  K_{a,\{b\}} &= 1 & K_{b,\{a\}} &= 0 \\
  K_{a,\{c\}} &= 1 & K_{b,\{b\}} &= 1 \\
  K_{a,\{b,c\}} &= 2  & K_{b,\{a,b\}} &= 0 \\
  \\
  K_{c,\emptyset} &= 0 & K_{c,\{c\}} &= 1
\end{align*}
}
\end{minipage}
\caption{\label{fig:runningBRN}
  (left)
  IG example.
  Components are represented by nodes labeled with a name
  and regulations by edges labeled with their sign and threshold.
  For instance, the edge from $b$ to $a$ is labeled $+1$, which stands for: $\GRNedgef{b}{+}{1}{a}$.
  This means that if the expression level of $b$ is equal to (\ie above) 1, then $b$ activates $a$;
  otherwise, $b$ inhibits $a$.
  (right)
  Example parametrization on the left IG.
}
\end{figure}

A \emph{state} $s$ of an IG $(\Gamma, E)$ is an element in $\GRNstates \DEF \prod_{a \in \Gamma} \segm{0}{l_a}$.
$\GRNget{s}{a}$ refers to the level of component $a$ in $s$.
For each possible state, the set of \emph{resources} of a given component
is the set of regulators of this component whose expression level is above the threshold of the regulation (\pref{def:resources}).
In other words, in every state $s$, a regulator $b$ of a component $a$
is either a resource (if $\GRNget{s}{b} \in \levels{b}{a}$)
or not a resource (if $\GRNget{s}{b} \in \ulevels{b}{a}$).
%For each possible state, all regulators of a component $a$ can be divided into
%\emph{activators} and \emph{inhibitors}, given their type of interaction and expression level,
%referred to as the \emph{resources} of $a$ in this state (\pref{def:resources}).
Relying on this observation, the specificity of Thomas' approach lies in the use of discrete \emph{parameters} to represent the
focal level interval towards which the component will evolve in every configuration of the resources (\pref{def:param}).

\begin{definition}[Resources ($\GRNreslabel$)]\label{def:resources}
For a given component $a \in \Gamma$ and a state $s \in \GRNstates$,
the set of regulators of $a$ whose level in $s$ is above the related threshold %to regulate $a$
is called the set of \emph{resources} of $a$ in $s$ and is noted $\GRNres{a}{s}$:
$$\GRNres{a}{s} \DEF \{ b \in \GRNreg{a} \mid \GRNget{s}{b} \in \levels{b}{a} \}$$
\end{definition}

\begin{definition}[Parameter $K_{a,\omega}$ and Parametrization $K$]\label{def:param}
For a given component $a \in \Gamma$, and $\omega \subset \GRNreg{a}$ a set of regulators of $a$,
the \emph{parameter} \modMF{$K_{a,\omega} \in \segm{0}{l_a}$ is a non-negative integer}.
The complete map $K$ of parameters on $\IG$ is called a \emph{parametrization} on $\IG$.
\end{definition}

%\todo{Discuss on the widespread use of discrete parameters + binary resources}
An IG and a parametrization make up a complete BRN.
%The purpose of intervals as parameters is discussed at the end of this section.
%Regarding the dynamics, an interval parameter $K_{a,\omega}$ is the set of values towards which $a$ will tend
%in the states where its resources are exactly the regulators in $\omega$.
\modMF{Regarding the dynamics, a parameter $K_{a,\omega}$ is the value towards which $a$ will tend
in the states where its resources are exactly the regulators in $\omega$.}
%as defined in the asynchronous dynamics of \pref{def:dynamics}.
%At last, \pref{def:dynamics} gives the asynchronous dynamics of a BRN using Thomas' parameters.
Indeed, from a given state $s$, a transition to another state $s'$ is possible provided that
exactly one component $a$ evolves of one expression level \modMF{towards} $K_{a,\GRNres{a}{s}}$,
as stated by the definition of the transition relation $\GRNtrans{s}{s'}$ (\pref{def:dynamics}).
However, $a$ cannot evolve if its expression level already \modMF{equals}
the parameter $K_{a,\GRNres{a}{s}}$.

\begin{definition}[Asynchronous dynamics ($\GRNtrans{}{}$)]\label{def:dynamics}
The dynamics of a BRN using Thomas' parameters is given by the transition relation $\GRNtrans{}{}\ \in \GRNstates \times \GRNstates$ defined by:
\begin{align*}
  \forall s, s' \in \GRNstates, \GRNtrans{s}{s'} &\Longleftrightarrow \exists a \in \Gamma,
  \GRNget{s}{a} \neq K_{a, \GRNres{a}{s}} \wedge \GRNget{s'}{a} = \GRNget{s}{a} + \delta^a(s) \\
    & \qquad\quad \wedge \forall b \in \Gamma, b \neq a \Rightarrow \GRNget{s}{b} = \GRNget{s'}{b}
\end{align*}
with: $\delta^a(s) =
  \begin{cases}
    +1 & \text{if } \GRNget{s}{a} < K_{a, \GRNres{a}{s}} \\
    -1 & \text{if } \GRNget{s}{a} > K_{a, \GRNres{a}{s}} \\
  \end{cases}$
\end{definition}

\begin{example}
\pref{fig:runningBRN}(right) gives a Parametrization on the IG of \pref{fig:runningBRN}(left).
In this BRN, the following transitions are possible given the semantics defined in \pref{def:dynamics}:
$$\GRNstate{a_0, b_1, c_1} \rightarrow \GRNstate{a_1, b_1, c_1} \rightarrow \GRNstate{a_2, b_1, c_1} \rightarrow
\GRNstate{a_2, b_0, c_1} \rightarrow \GRNstate{a_1, b_0, c_1} \enspace,$$
where $a_i$ denotes the component $a$ at level $i$.
This sequence of states ends in a steady state: no evolution is possible in $\GRNstate{a_1, b_0, c_1}$.
\end{example}

\begin{remark}[Parameters equivalence]
\label{rem:K-equiv}
\modLP{%
In the scope of the asychronous dynamics (\pref{def:dynamics}), one can remark
that some parameters values can be equivalent, \ie the resulting dynamics is
the same.
It is notably the case with auto-regulations:
if $\GRNedge{a}{s}{t}{a}$,
then for any $\omega\subseteq \GRNreg{a}$,
two values $v$ and $v'$ of the parameter 
$K_{a,\omega\setminus \{a\}}$ are equivalent
if $v\geq t$ and $v'\geq t$.
Similary,
two values $v$ and $v'$ of the parameter
$K_{a,\omega\cup\{a\}}$ are equivalent 
if $v<t$ and $v'<t$.}
\end{remark}


\paragraph{Interaction Graph of the dynamics}
From the dynamics specification of a BRN, one can infer back the IG that contain only the functional
regulations.
Let us define
$f^a(s) = s[a]$ if $s[a] = K_{a, \GRNres{a}{s}}$
and
$f^a(s) = s[a]+\delta^a(s)$ otherwise.
A positive (resp. negative) interaction from $b$ to $a$ is inferred if there exists a state $s$ such
that increasing $b$ in $s$ would increase (resp. decrease) $f^a$; in other words, if
there exists a state $s$ so that:
$f^a(s\{b_i\}) < \text{(resp. $>$) } f^a(s\{b_{i+1}\})$, 
where $i < l_b$ and $s\{b_i\}$ denotes the state $s$ where the component $b$ is assigned to $b_i$.
Then, such an IG can be used to derive global properties on the dynamics
(e.g., \cite{Richard2010378,PR11-SASB}).

