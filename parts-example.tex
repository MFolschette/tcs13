\section{Examples}\label{sec:examples}

\modMF{%
This section aims at giving several applications of our work in order to
understand its results and range of applications.
First, \pref{ssec:phage-lambda} gives a detailed application of our IG inference method
by studying parts of a model of the phage lambda immunity control.
}%
% the functioning of our method
% and the implementation of the software developed for this work.
% First, \pref{ssec:phage-lambda} aims at giving a detailed example of the application of our method
% on parts of a model of the phage lambda immunity control.
%We then explain the use of the implementation and usage of the software developed for this work:
Then, \pref{ssec:appli} gives a practical application of our results on a biological model
of epithelial growth factor receptor taken from the literature.
Finally, \pref{ssec:cpu} gives data related to the results and computation times of our software
when applied to several large biological models.

\medskip

The inference \modMF{and enumeration} methods described in this paper have been implemented as part of
\textsc{Pint}\footnote{Available at \url{http://loicpauleve.name/pint}}, which gathers PH related tools.
Our implementation mainly consists in ASP programs that are solved using
Clingo\footnote{Available at \url{http://potassco.sourceforge.net}}.
The IG and parameter inferences can be performed using the following command:\\
  \hspace*{\parindent}\texttt{ph2thomas -i model.ph -{}-dot ig.dot}\\
where \texttt{model.ph} is the PH model file in \textsc{Pint} format,
and \texttt{ig.dot} is an output file to write the inferred IG in DOT format.
The (possibly partial) inferred parametrization will be returned on the standard output.
\modMF{%
Instead of the \texttt{-{}-dot ig.dot} option, it is possible to specify
an alternative IG to perform the parameters inference on
(instead of using the inferred IG) with the \texttt{-{}-ig model.ig} option,
where \texttt{model.ig} is an IG model file in \textsc{Pint} format.
The admissible parametrizations enumeration is performed by adding the \texttt{-{}-enumerate}
option to the command.
%or \texttt{-{}-enumerate} instead to obtain only simple parameters (intervals of cardinality $1$).
}%




\begin{example}
\begin{align*}
  \PHs &= \{ cI, cro, cII, N \} \\
  \PHl_{cI} &= \{ cI_0, cI_1, cI_2 \} &
  \PHl_{cro} &= \{ cro_0, cro_1, cro_2, cro_3 \} \\
  \PHl_{cII} &= \{ cII_0, cII_1 \} &
  \PHl_{N} &= \{ N_0, N_1 \} \\
\end{align*}
  % TODO : figure PH du modèle
\end{example}

\begin{example}
  The IG inference of \pref{pps:inference-IG} applied to the PH of \todo{ref}
  produces the IG given in \pref{fig:phage-lambda-ig}.
  This IG contains all edges included in the original model except the
  self-influence on $cI$ : $\GRNedgef{cI}{+}{2}{cI} \notin E$;
  this is due to the fact that an equivalent dynamics exists without this edge.
  \begin{figure}[t]
  \centering
  \scalebox{1.2}{
  \begin{tikzpicture}[grn]
    %\path[use as bounding box] (-1.3,-0.75) rectangle (3.5,1.5);
    \node[inner sep=0] (cI) at (0,2.5) {cI};
    \node[inner sep=0] (cro) at (2.5,2.5) {cro};
    \node[inner sep=0] (cII) at (0,0) {cII};
    \node[inner sep=0] (N) at (2.5,0) {N};
    \path[->]
      (cI) edge[bend right] node[elabel, above=-4pt] {$-2$} (cro)
           edge[bend right] node[elabel, left=-2pt] {$-2$} (cII)
           edge[pos=.85] node[elabel, above=-2pt] {$-1$} (N)
      (cro) edge[bend right] node[elabel, above=-4pt] {$-1$} (cI)
            edge[loop right] node[elabel, below=-2pt] {$-3$} (cro)
            edge[pos=.65] node[elabel, below=-2pt] {$-3$} (cII)
            edge node[elabel, right=-4pt] {$-2$} (N)
      (cII) edge[bend right] node[elabel, left=-3pt] {$+1$} (cI)
      (N) edge node[elabel, above=-4pt] {$+1$} (cII);
  \end{tikzpicture}
  }
  \caption{\label{fig:phage-lambda-ig}
    Result of the IG inference performed on the PH of \todo{ref}.
  }
  \end{figure}
\end{example}

\begin{example}
  The parameters inference of \pref{pps:param_K} applied to the PH of \todo{ref}
  is conclusive for all parameters.
  The complete parametrization produced is available in \pref{tb:phage-lambda-k}.
  \begin{table}[t]
  \begin{align*}
    K_{cI, \emptyset} &= \{2\} &
    K_{cI, \{cro\}} &= \{0\} \\
    K_{cI, \{cII\}} &= \{2\} &
    K_{cI, \{cII;cro\}} &= \{2\} \\
    K_{N, \emptyset} &= \{1\} &
    K_{N, \{cro\}} &= \{0\} \\
    K_{N, \{cI\}} &= \{0\} &
    K_{N, \{cI;cro\}} &= \{0\} \\
    K_{cII, \emptyset} &= \{0\} &
    K_{cII, \{cro\}} &= \{0\} \\
    K_{cII, \{cI\}} &= \{0\} &
    K_{cII, \{cI;cro\}} &= \{0\} \\
    K_{cII, \{N\}} &= \{1\} &
    K_{cII, \{N;cro\}} &= \{0\} \\
    K_{cII, \{N;cI\}} &= \{0\} &
    K_{cII, \{N;cI;cro\}} &= \{0\} \\
    K_{cro, \emptyset} &= \{3\} &
    K_{cro, \{cro\}} &= \{2\} \\
    K_{cro, \{cI\}} &= \{0\} &
    K_{cro, \{cI;cro\}} &= \{2\}
  \end{align*}
  \caption{\label{tb:phage-lambda-k}
    Result of the parameters inference performed on the PH of \todo{ref}.
  }
  \end{table}
\end{example}




\subsection{Biological application: the epithelial growth factor receptor}\label{ssec:appli}

This subsection focuses on the study of the epidermal growth factor (EGF) receptor model detailed in~\cite{Sahin09}.
This model is represented by an IG containing 20 components and 52 edges.
A protein named EGF, having no regulator, can be considered as the only input,
and a chain of reactions leads to the activation of protein pRB which is responsible for regulating the cell division,
therefore making it essential to prevent cancer development.

Three models are created from the original IG, with different levels of precision regarding the cooperation between components.
The translation from an IG into a Process Hitting is not detailed here as it was previously covered in~\cite{PMR10-TCSB}.
Model (1) represents a translation of the raw IG into Process Hitting, that is,
without any knowledge of the Boolean rules (and therefore the cooperations) of the components.
Model (2) implements some of the rules based on the results of several knockdown experiments.
Model (3) is the totally refined model with all cooperations implemented given the Boolean functions of all components.
Therefore, those three models can be considered as successive refinements of the original and most general one.
The results of the IG and parameters inferences are detailed in \pref{tb:egfr20}
and discussed in the following alongside with details about their construction.

\begin{table}[ht]
~\hfill%\begin{center}
  \begin{tabular}{r|l|l|m{2cm}|m{2.5cm}|m{1.5cm}}
    \textbf{Model} & $\mathbf{|E|}$ & $\mathbf{|K|}$ & \textbf{Inferred parameters} & \textbf{Possible\newline models} & \textbf{Fixed points}
  \\\hline\hline
    (1) & $52$ & $196$ & $20$ & $2^{176}\simeq 9.6\cdot10^{52}$ & $0$   % v1_0.ph
  \\\hline
    (2) & $51$ & $192$ & $98$ & $2^{94}\simeq 2.0\cdot10^{28}$ & $0$    % v2_1.ph
  \\\hline
    (3) & $51$ & $192$ & $192$ & $1$ & $3$                              % ori.ph
  \\\hline
  \end{tabular}
\hfill~%\end{center}
  \caption{%
  Results of the IG and parameters inference on three models
  derived from the EGF receptor model of~\cite{Sahin09}
  with different precisions in the definition of the cooperations.
  Model (1) contains no cooperations between the components.
  Several cooperations were included in model (2) under the form of 14 cooperative sorts
  and all of them were included in Model (3) under the form of 22 cooperative sorts.
  The second column gives the number of edges in the IG inferred with \pref{pps:inference-IG}
  (the number of nodes is always the number of components in the model, that is, 20).
  The third column gives the number of parameters in the model (given the IG),
  the fourth column gives the number of parameters that could be inferred using \pref{pps:param_K},
  and the fifth column consequently gives the number of compatible models with the studied PH model,
  which exponentially depends on the number of parameters that could not be inferred.
  Finally, the last column gives the number of fixed points in the PH model,
  computed with another existing PH tool provided with \textsc{Pint}.
  }
  \label{tb:egfr20}
\end{table}

Model (1) encompasses only sole interactions between components, that is,
independent activations or inhibitions of a component on another given the regulations specified in the original IG.
Therefore, the IG inferred from model (1) is the same as the IG used to create the model,
with one additional positive auto-edge on the only input EGF (which is due to its absence of regulators).
The only parameters that could be inferred are the parameters for the extreme cases of regulation
(all activators present and all inhibitors absent, and the opposite).
This first model therefore abstracts a large number of Thomas models as a lot of parameters are left undecided.

In order to build model (2), 14 cooperative sorts were added in order to model the Boolean functions of several components
(consisting of AND and OR operands).
To do so, the following components were noticed due to their importance in the chain of reactions:
CDK4, CDK6, CycD1, ER$\alpha$ and c-MYC.
Indeed, knockdown experiments have been conducted in~\cite{Sahin09}
and the results showed that knocking down these components lead to an important decrease in the production of pRB.
We therefore concluded that these components were involved in other components' Boolean functions
in a way that the knockdown of the former was sufficient to prevent the production of the latter (which is typical of AND operands).
In order to reproduce such requirements, the Boolean functions of their successors,
that is CDK4, CDK6, prB, p21, p27, IGF1R, MYC, CycD1 and CycE1,
were modeled as cooperative sorts, if needed.
In theory, 9 cooperative sorts would have sufficed, but the chaining of cooperative sorts described
in \pref{ssec:PH} was used to reduce the number of processes in each cooperative sort.
As a result, the added cooperations allowed to infer about half the parameters;
however, the number of possible Thomas models that can be inferred from this PH is still significant
because of the numerous remaining unknown parameters.
Furthermore, we note that the inferred IG contains one edge less than the original IG. This is due to the fact that
one of the Boolean functions could in fact be simplified in a way that a component did not appear anymore in it.
No edge have therefore been inferred by our method in this case.

Finally, model (3) was build using all the Boolean functions provided in~\cite{Sahin09}.
These functions take the form of 22 cooperative sorts into the model in order to match the desired behavior of the system.
As all cooperations are fully defined in this model, all the parameters are inferred and only one Thomas model can be derived.
We note also that this PH model is the only one containing at least one fixed point.
In fact, the three found steady states include the two states that correspond to a complete propagation of the input signal,
that is, in the case where EGF is active and in the case where it is not.
The two other models contain no fixed point because some cooperations are not fully defined,
leading to oscillations that are a consequence of the nondeterministic behavior.



\subsection{Computation times on several large models}\label{ssec:cpu}

The current implementation can successfully handle large PH models of BRNs found in the literature such as:
\begin{itemize}
  \item the EGF receptor model from~\cite{Sahin09} with 20 components presented in the last
    subsection\footnote{All models mentioned in this section are available as examples distributed with \textsc{Pint}.},
  \item a T cell receptor model described as an IG in~\cite{Klamt06}, which contains 40 components and 14 cooperative sorts.
\end{itemize}
For each model, IG and parameters inferences are performed together in less than a second
on a standard desktop computer.

Bigger models related to the aforementioned systems were also tested with our implementation:
\begin{itemize}
  \item a model of the T cell receptor with 94 components, described in~\cite{SaezRodriguez2007},
  \item a model of the EGF receptor with 104 components, described in~\cite{Samaga2009}.
\end{itemize}
These two models were obtained in a previous work by an automatic translation from the CellNetAnalyzer~\cite{klamt2007structural} formalism.

The composition of all models and the results of the inferences are summarized in \pref{tb:computation}.
%We note that due to a very high number of parameters, no parameters inference could be performed on the $94$ and $104$ component models.
%These models would therefore be more efficiently studied as PH than as BRNs.
%Finally, we note that the complexity of the method is exponential in the number of regulators of one
%component and linear in the number of components.

\begin{table}[ht]
\begin{center}
  \begin{tabular}{r|c|c|c|c|c}
    \textbf{Model} & $\mathbf{|\Gamma|}$ & $\mathbf{|\PHs \setminus \Gamma|}$ & \textbf{$\Delta t$ IG} & \textbf{$\Delta t$ K} & $\mathbf{|K|}$
\\\hline\hline
    EGF receptor \cite{Sahin09} & $20$ & $22$ & <1s & <1s & 192
\\\hline
    T cell receptor \cite{Klamt06} & $40$ & $14$ & <1s & <1s & 143
\\\hline
    T cell receptor \cite{SaezRodriguez2007} & $94$ & $39$ & \modMF{100s} & \modMF{<1s} & \modMF{578}
\\\hline
    EGF receptor \cite{Samaga2009} & $104$ & $89$ & \modMF{200s} & \modMF{2.5s} & \modMF{27496}
  \end{tabular}
\end{center}
\caption{%
  Computation times and several pieces of information related to the IG and parametrization inferences of four biological models.
  The second column gives the number of components of each model and
  the third column gives the number of cooperative sorts used to model joint actions.
  The fourth (resp.~fifth) column gives the computation times of the IG inference (resp.~the parametrization inference).
  The last column gives the number of parameters in each model.
  }
\label{tb:computation}
\end{table}
