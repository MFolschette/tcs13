\section{Conclusion and Discussion}

This work establishes the abstraction relationship between PH and Thomas' approaches for
qualitative BRN modeling.
The PH allows an abstract representation of BRNs dynamics (allowing incomplete knowledge on the
cooperation between components) that cannot be exactly represented in Ren\'e Thomas' formalism by a
single instance of BRN parametrization.
This motivates the concretization of PH models into a set of compatible Thomas' models in order to benefit
of the complementary advantages of these two formal frameworks.

We first propose an original inference of the Interaction Graph (IG) from a BRN
having its dynamics specified in the PH framework.
An IG gives a compact abstract representation of the influence of the components between each
others.
Then, based on a prior inference of Ren\'e Thomas' parametrization for BRNs from a PH model, we
delimit the set of admissible Thomas' parametrizations that are compatible with the PH dynamics,
and give arguments for their correctness.
A parametrization is compatible with the PH if its dynamics (in terms of possible transitions) is included in the PH dynamics.
The enumeration of such parametrizations is efficiently tackled using Answer Set Programming.
We illustrate the overall method with several results
on \modMF{both small and} large biological models.

Several extensions of the presented work are now to be considered.
First, the link between successively refined models of a system could be formally studied.
Indeed, it is convenient to refine a PH model by removing actions or adding cooperations;
the study and formalization of such process would allow to predict behavioral changes and lead to more accurate models.
Second, the inference of BRN multiplexes \cite{BernotMultiplexes} may be of practical interest 
as they allow to implicitly reduce the possible parametrizations by making cooperations appear
in the IG.
Because of its atomicity, the PH allows to specify a range of cooperations that cannot be
completely captured by a single instance of BRN multiplexes, then encouraging the inference of a set
of compatible ones.
Finally, in order to improve the performances in the IG inference, we will consider projection operations on
the PH structure to undo cooperations between components and reduce the cardinality of
configurations to explore by making the interactions independent.

\paragraph{Acknowledgement}
This work was partially supported by the Fondation Centrale Initiatives and
the French National Agency for Research (ANR-10-BLANC-0218 BioTempo project).

